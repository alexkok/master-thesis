% % % % % % % % % % % % % % % % % % % % % % % % % % % % % % % % % % % % %
% Chapter: Conclusion
% % % % % % % % % % % % % % % % % % % % % % % % % % % % % % % % % % % % %
\chapter{Conclusion}
\label{chp:conclusion}
% Summary
In this thesis, we have shown a way how the generator can be tested by using property-based testing. This is done by generating tests based on the \textit{Rebel} specification and making use of the generator to generate the tests. The \textit{Rebel} specification is build up based on a set of defined properties of \textit{Rebel}. These property definitions were not defined earlier, thus we defined many expected properties in \textit{Rebel}.\\
\\
With test framework we have found some bugs in the generated system that were unknown before. This proves that this approach already worked to identify some problems in the generator that were not known before. Additionally, we contributed to an open-source library called \textit{Squants}, by issuing two reports of bugs that existed in the library.\\
\\
% Research questions
% Answer main research question
To answer the main research question, we defined and answered the three sub research questions, which have been discussed in \autoref{cpt:discussion}. The main research question was as follows:
\begin{quote}
\rqMain
\end{quote}
A \textit{Rebel} specification is created from the defined properties. Next, the existing generator is being used to generate the generated system and to translate the properties (in the \textit{Rebel} specification) to test cases. When running the test framework, we found some errors by using the generated system. Additionally, we were able to detect a compilation error and incorrectly translated formulas.\\
\\
We conclude that, by using this approach, the generator can be checked on whether it satisfies the defined properties. The generated system is used to determine the result.

% % % % % % % % % % % % % % % % % % % % % % % % % % % % % % % % % % % % %
% Section: Future work
\section{Future work}
% Multi VM testing, specify sync properties, can test with multi JVM testing etc. Sources > future work

\subsection*{Complete property definitions}
\pinfo{Full definitions for Rebel}
In this thesis, we defined some properties on the \textit{Rebel} language,
which we consider to hold when using \textit{Rebel}. This list of property
definitions on \textit{Rebel} is not complete, there can be many more properties
and there are other types available in \textit{Rebel} which we did not cover.
These should be added to have a complete set of property definitions for the
\textit{Rebel} language. This is left as future work. When additional properties
are known, these properties can be added to the test framework such that these
will also be tested on the generator.

\subsection*{Values generation by using solvers}
\pinfo{Current value generation limitations}
The current value generation works for the properties that we defined throughout this thesis. When additional properties are being added, the value generator might or might not have to be updated. This depends on what preconditions have to be parsed for that property and whether the current implementation supports these operations. As described, the current implementation has some limitations. It does not support expressions that are using different operators or that use expressions on both the left-hand and right-hand side (expressions here means non-literals or variables, but a combination of operators and literals/variables).\\
\\
\pinfo{Other tools possible}
Another way how the value generator could work is to use some tools to determine the random values. As we have discussed in \autoref{cpt:experiment3}, the SMT solver would theoratically be an option. But in our attempts, this resulted in having the same input values every time the test framework is run, meaning that the randomness of the values is lost in the first part. To counter this problem, other solvers might be more useful to implement this behaviour, such as \textit{SageMath}~\cite{siteSageMath2017}. \textit{SageMath} can parse expressions and determine the conditions to which each variable in the expression should hold. When using this, it would mean that the value generator has to be modified such that it integrates with this tool. Another approach could be used for this too, such as concolic testing~\cite{sen2006cute} to determine the input values randomly. % Maybe there are some QuickCheck ports that do this already for some languages?

\subsection*{Multiple generators}
\pinfo{Only one, support for more could be added}
Throughout this thesis, we have only tested the Scala/Akka generator which is
developed by \textit{ING}. Since there are more generators available, the test
framework can be improved such that it is
compatible with the other systems that can be generated by using one of the
other generators that are available within \textit{ING}. By doing this, the same
property definitions can be tested on different kind of generated systems. This
can be used to detect inequalities among the generated system.

\subsection*{Mutation testing}
A way to evaluate the effectiveness of the test framework would be to use
mutation testing. The mutation coverage could be used to measure how
effective the test framework would be (the number of mutants created and the
number of killed mutants). Unfortunately, there is currently a limited support
for mutation testing for \textit{Scala} systems, as it cannot meaningfully
mutate \textit{Scala} code~\cite{siteSbtPit2017}.\\
\\
\textit{Scala} compiles to \textit{Java} bytecode, mutating on bytecode level
could be used to solve the problem of limited support. However, this could lead
to false-positives, meaning that some mutants might not be relevant in that the
modifications would not affect the implementation. This should then be taken
into account.

\subsection*{Other components}
The current setup of the test framework is intended to be used to test a certain
component of the generated system. Namely, the \textit{Rebel} types and the
operations among these. More properties can be added to check every type that
\textit{Rebel} supports. The test framework could also be extended such that it
can also test other components of the system. Such as the sync block
definitions, defining actions that should happen synchronously. Or performance
measures when interacting with the data in the system, which uses a database
implementation. Currently, such components are not being tested, while these
components are also important for the bank. This is left as future work.\\
\\
Although the test framework can be extended to also test other components of
the generated system, it might not be possible to check each and every component
by using this approach. For example, it is probably not able to check how
multiple generated systems would integrate with each other and if this is done
correctly. The generated system is configured to work with multiple
distributions of the system, but this is not being tested yet.

% Ultimately, as last
\subsection*{Properties based on type checker}
\pinfo{Generating properties based on type checker}
\textit{Rebel} contains a type checker, which is able to determine exactly
which operations are supported by using certain combinations of operators and
types. This could be used to automatically generate the property specifications
in \textit{Rebel} based on whether a type supports certain operations that are
required for the property. For example, if \textit{Percentage} supports
addition, the properties for \textit{additivity}, \textit{associativeAddition}
and \textit{commutativeAddition} can be generated based on this rule.\\
\\
Considering that a map
of all operations can be created and a map of all existing types, each
combination can be tested against the type checker. If the type checker an
expression as correct, the event definition for the property can
be generated. Although this might result in many definitions that are being
tested, with a possible overlap between them, it would be a way to automatically
generate the properties that should be tested. This would assume that the type
checker does the right thing, which would be a threat for this approach. This
might also cause more compilation errors in the generated system. Nevertheless,
it can make it more dynamic when a new type could be added to \textit{Rebel}.
