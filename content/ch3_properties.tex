% % % % % % % % % % % % % % % % % % % % % % % % % % % % % % % % % % % % %
% Chapter: Property definitions
% % % % % % % % % % % % % % % % % % % % % % % % % % % % % % % % % % % % %
\chapter{Properties of Rebel}
\label{cpt:properties}
% Definition: An Axiom is a mathematical statement that is assumed to be true.
% ( http://www.aaaknow.com/lessonFull.php?slug=propsCommAssoc )
% Sources:
% Axioms in algebra — where did they come from?
% http://www.jstor.org.proxy.uba.uva.nl:2048/stable/pdf/27956337.pdf?refreqid=excelsior%3A623b1c85f79f8cd18bc9d09560339b9b
% (simple formulas, history, and explanations)
% A perspective on the algebra of logic
% http://www-tandfonline-com.proxy.uba.uva.nl:2048/doi/pdf/10.2989/16073606.2011.622856?needAccess=true
% (More formulas, deeper explanations)
% Might contain more properties

% Another source was (but not a paper or so)
% http://www.aaamath.com/ac22.htm

% Another good source (book)
% Calculus, Vol. 1: One-Variable Calculus, with an Introduction to Linear Algebra 2nd Edition

\pinfo{Specific types in Rebel}
Rebel introduces specific types, like \textit{Iban} and \textit{Money} and
allows operations on those~\cite{stoel2016solving}. But what are the expected
properties of the generator? In this chapter we will try to answer the first
research question:
\begin{description}
	\item [~~~~RQ 1:] \rqOne
\end{description}
To answer this question, we first describe a way how we can determine the
properties. Followed by the property defintions that will be used troughout this
thesis, with a motivation why these properties are expected to hold.

%\todo[inline]{Percentage? Not yet, defining those requires properties again. And running the test suit again, evaluate again, etc.\\
%Describe what exactly is Percentage if we define it in a specification, which rules does it inherit?\\
%Percentage (expected): 2/more decimals, however its a precise amount. Calculating with it is expected in a result of the real value.\\
%Percentage (current): As it is defined now: full number, it can exceed 100. However it does not support decimals yet.}

% % % % % % % % % % % % % % % % % % % % % % % % % % % % % % % % % % % % %
% Section: Determining the properties
\section{Determining the properties}
\pinfo{Properties not defined for Rebel. We must define these}
Currently, there are no definitions available of what the properties are of each
type and operation in \textit{Rebel}. Due to the missing definitions of these
types, it means that we first have to define what the the expected properties
on these types are and substantiate these. Only then we can determine whether
the generator is working as expected with these properties. As there are many
operations available among the available types in \textit{Rebel}, we are not
able to define all the properties that exist in the \textit{Rebel} language. As
there can be countless of properties and combinations among the different types
that \textit{Rebel} supports. During this thesis we will focus on the
\textit{Money} type, considering this is the most important type for a bank that
has the highest priority to be implemented correctly.\\
\\
\pinfo{Types and axioms}
For types like integers the axioms of algebra can be used to determine whether
the implementation in the generated system is correct. These are most likely
translated to integers in the generated system too, with perhaps the expectation
that these have the same properties in \textit{Scala}. However, it is not
possible to rely on the \textit{Integer} definition of a specific language, as
another generator might generate a system in another language, or might
implement it differently in the same language. Would that mean that the
properties on that other language should now hold on the \textit{Integer} type?
Well, as this is not defined for \textit{Rebel}, this is unknown. In this
chapter we will define properties that are expected to hold on the
\textit{Money} type in \textit{Rebel}. The properties that we define are based
on the known axioms in
algebra~\cite{baumgart1961axioms,raftery2011perspective,apostol2007calculus}.
We provide an explanation of why a certain property should also hold in these
cases.\\
\\
\pinfo{Amount, not floating-point arithmetic}
The \textit{Money} type can be seen as a currency with an amount value. The
amount of a \textit{Money} value can have multiple decimals depending on the
currency. Thus, the amount can be seen as a floating number. Does this mean that
it inherits the computation properties of Floating-Point Arithmetic, as defined
in the IEEE standards 754 or 854? Since the Rebel is intended to be a formal
specification language for banking products, we don't expect that the described
problems with this arithmetic are intended to exist on the Money type.
Considering that that a high volume flows within a bank in terms of
\textit{Money}, using the Floating-Point Arithmetic properties can result in the
known precision, overflow and underflow errors as described
in~\cite{goldberg1991every}. Such errors should be avoided when using the
\textit{Money} type. The author of \cite{fowler2002patterns} also describes that
the intend of the \textit{Money} type is to avoid this:
\begin{quote}
\todo{Use quotes instead of "" (latex-advice)}
	"You should absolutely avoid any kind of floating point type, as that will
	introduce the kind of rounding problems that Money is intended to avoid."
    -- Martin Fowler \cite{fowler2002patterns}
		%[p. 462] % Of pdf, ebook thing doesn't have page numbers
\end{quote}

\pinfo{Money operations, expecting division problem}
The author of~\cite{fowler2002patterns} also describes the operations that can
be done with the \textit{Money} type, which are: +, -, *, allocate, $<$, $>$,
$\leq$, $\geq$ and =. Where the allocate method is used instead of the division (/) operation. This is due to the division problem, requiring a number to be rounded off at a certain time. For example, when splitting 1 EUR with 3 people, everyone would receive 33 cents, but what is done with the last cent that is left? This is the problem that is being solved by the allocate method, which describes the ratio \todo{Check in source if 'ratio' is correct} on where the last cent would go to in this case. The allocate method is a thing that \textit{Rebel} does not have, instead it just allows the use of the division operator. Because of this, we expect the division problem to occur while running the test framework. The amount of a \textit{Money} value is often rounded when it is being represented to the user, as it could have many decimals. The representation of the \textit{Money} value is up to the business on how this is done, as there are multiple factors influencing this. Instead, we only focus on the internal value that is used when operating with the \textit{Money} type.\\
\\
\pinfo{Not between different currencies}
It is unsupported to use these operations with \textit{Money} values when using
values that are of different currencies. This is due to the exchange rates
between currencies, which can vary and are not implemented yet.\\
\\
\pinfo{Thus: precise value, rounding only for division}
We say that the amount of a \textit{Money} value in \textit{Rebel} should hold the exact value as if we would calculate the same expression by ourselves. Meaning that the precision errors would have to be prevented. An exception on this is when the result would be in the form of a fraction, for example 1/3. As this results in a value which we have to round up sometime. To fix this, we say that in this case we use x decimals when calculating, without rounding the x+1th decimal. When using properties with division, this should be taken into account in case the system fails on these tests. \todo{Define X, is this also 60 decimals?}

% % % % % % % % % % % % % % % % % % % % % % % % % % % % % % % % % % % % %
% Section: Property definitions
\section{Property definitions}
\pinfo{Properties based on known axioms, but has restrictions}
The properties that we define are based on the known axioms in algebra. Although, not every property can be used. For example, it isn't possible to multiply two \textit{Money} types with each other to support the multiplicative property for example. Instead we can only multiply \textit{Money} with other types, such as \textit{Integer} and \textit{Percentage}, resulting in multiple property definitions when using different types. In the following sections we motivate the properties that we use during the project.
% % % % % % % % % % % % %
% Subsection: Reflexivity
\subsection*{Reflexivity}
% Table
\FloatBarrier
\begin{table}[!ht]
\centering
\begin{tabular}{lll}
\hline
                        \textbf{Formula} & \textbf{Property name} & \textbf{Variable (Type)} \\ \hline
\rowcolor[HTML]{EFEFEF} x == x           & reflexiveEquality      & x: Money                 \\
                        x $\leq$ x       & reflexiveInequalityLET & x: Money                 \\
\rowcolor[HTML]{EFEFEF} x $\geq$ x       & reflexiveInequalityGET & x: Money                 \\ \hline
\end{tabular}
\caption{Reflexivity on \textit{Money}}
\label{tbl:ch4_money_reflexivity}
\end{table}
\FloatBarrier
% End table
% Reasoning
\pinfo{Equality, currency and amount}
The reflexive property means a relation of a type with itself \cite{raftery2011perspective}. An instance of type \textit{Money} should be equal to itself. Taking both the currency and the amount into account. The inequality relations \textit{smaller or equal to} and \textit{greater or equal to} should hold too. As we can compare \textit{Money} variables and defined equality in the first row of the table.

% % % % % % % % % % % % %
% Subsection: Symmetry
\subsection*{Symmetry}
% Table
\FloatBarrier
\begin{table}[!ht]
\centering
\begin{tabular}{lll}
\hline
                        \textbf{Formula}         & \textbf{Property name} & \textbf{Variable (Type)} \\ \hline
\rowcolor[HTML]{EFEFEF} x == y $\implies$ y == x & symmetric              & x: Money                 \\
\rowcolor[HTML]{EFEFEF}                          &                        & y: Money                 \\ \hline
\end{tabular}
\caption{Symmetry on \textit{Money}}
\label{tbl:ch4_money_symmetry}
\end{table}
\FloatBarrier
% End table
% Reasoning
\pinfo{Equality, currency and amount}
Reflexivity already described equality on \textit{Money} when used on the same variable. When two different variables are used, the order should not matter and thus it should work in both ways. Which is known as the symmetric property \cite{raftery2011perspective}.

% % % % % % % % % % % % %
% Subsection: Antisymmetry
\subsection*{Antisymmetry}
% Table
\FloatBarrier
\begin{table}[!ht]
\centering
\begin{tabular}{lll}
\hline
                        \textbf{Formula}                             & \textbf{Property name} & \textbf{Variable (Type)} \\ \hline
\rowcolor[HTML]{EFEFEF} x $\leq$ y \&\& y $\leq$ x $\implies$ x == y & antisymmetryLET        & x: Money                 \\
\rowcolor[HTML]{EFEFEF}                                              &                        & y: Money                 \\
                        x $\geq$ y \&\& y $\geq$ x $\implies$ x == y & antisymmetryGET        & x: Money                 \\
                                                                     &                        & y: Money                 \\ \hline
\end{tabular}
\caption{Antisymmetry on \textit{Money}}
\label{tbl:ch4_money_antisymmetry}
\end{table}
\FloatBarrier
% End table
% Reasoning
The antisymmetric relation describes that whenever there is a relation from x to y and a relation from y to x, then x and y should be equal. The lower or equal then and greater or equal then relations fit in this category, as shown in \autoref{tbl:ch4_money_antisymmetry}. We can use these operations on the \textit{Money} type when both x and y use the same currency. This antisymmetric relation is also expected to hold, as \textit{Money} values should be equal when they are of the same currency and hold the same amount.

% % % % % % % % % % % % %
% Subsection: Commutativity
\subsection*{Commutativity}
\label{ssct:4_commutativity}
% Table
\FloatBarrier
\begin{table}[!ht]
\centering
\begin{tabular}{lll}
\hline
                        \textbf{Formula} & \textbf{Property name}               & \textbf{Variable (Type)} \\ \hline
\rowcolor[HTML]{EFEFEF} x + y == y + x   & commutativeAddition                  & x: Money                 \\
\rowcolor[HTML]{EFEFEF}                  &                                      & y: Money                 \\
                        x * y == y * x   & commutativeMultiplicationInteger1    & x: Integer               \\
                                         &                                      & y: Money                 \\
\rowcolor[HTML]{EFEFEF} x * y == y * x   & commutativeMultiplicationInteger2    & x: Money                 \\
\rowcolor[HTML]{EFEFEF}                  &                                      & y: Integer               \\
                        x * y == y * x   & commutativeMultiplicationPercentage1 & x: Percentage            \\
                                         &                                      & y: Money                 \\
\rowcolor[HTML]{EFEFEF} x * y == y * x   & commutativeMultiplicationpercentage2 & x: Money                 \\
\rowcolor[HTML]{EFEFEF}                  &                                      & y: Percentage            \\ \hline
\end{tabular}
\caption{Commutativity on \textit{Money}}
\label{tbl:ch4_money_commutativity}
\end{table}
\FloatBarrier
% End table
% Reasoning
These properties are based on the commutative law \cite{baumgart1961axioms}. The result of an addition or multiplication does not vary when swapping the input variables. Because of the \textit{Money} type, we can only do addition on \textit{Money} values with other \textit{Money} values. For multiplication, there is no known value for multiplying two \textit{Money} variables. It is possible to multiply it by an \textit{Integer} or \textit{Percentage}. Also in this case, the order shouldn't matter if we would put the \textit{Money} value as first input parameter to multiplication or the other way around.

% % % % % % % % % % % % %
% Subsection: Anticommutativity
\subsection*{Anticommutativity}
% Table
\FloatBarrier
\begin{table}[!ht]
\centering
\begin{tabular}{lll}
\hline
                        \textbf{Formula}  & \textbf{Property name} & \textbf{Variable (Type)} \\ \hline
\rowcolor[HTML]{EFEFEF} x - y == -(y - x) & anticommutativity      & x: Money                 \\
\rowcolor[HTML]{EFEFEF}                   &                        & y: Money                 \\ \hline
\end{tabular}
\caption{Anticommutativity on \textit{Money}}
\label{tbl:ch4_money_anticommutativity}
\end{table}
\FloatBarrier
% End table
% Reasoning
In \autoref{ssct:4_commutativity} we described the commutative properties. Note that the operations only use addition and multiplication on this property. Subtraction is a operation that is anticommutative as swapping the order of the two argumates is negating the result. The anticommutative property thus negates the result of swapping the two arguments, intending to result in the actual value again, as shown in \autoref{tbl:ch4_money_anticommutativity}.

% % % % % % % % % % % % %
% Subsection: Transitivity
\subsection*{Transitivity}
% Table
\FloatBarrier
\begin{table}[!ht]
\centering
\begin{tabular}{lll}
\hline
                         \textbf{Formula}                                 & \textbf{Property name}  & \textbf{Variable (Type)} \\ \hline
\rowcolor[HTML]{EFEFEF}  x == y \&\& y == z $\implies$ x == z             & transitiveEquality      & x: Money                 \\
\rowcolor[HTML]{EFEFEF}                                                   &                         & y: Money                 \\
\rowcolor[HTML]{EFEFEF}                                                   &                         & z: Money                 \\
                         x $<$ y \&\& y $<$ z $\implies$ x $<$ z          & transitiveInequalityLT  & x: Money                 \\
                                                                          &                         & y: Money                 \\
                                                                          &                         & z: Money                 \\
\rowcolor[HTML]{EFEFEF}  x $>$ y \&\& y $>$ z $\implies$ x $>$ z          & transitiveInequalityGT  & x: Money                 \\
\rowcolor[HTML]{EFEFEF}                                                   &                         & y: Money                 \\
\rowcolor[HTML]{EFEFEF}                                                   &                         & z: Money                 \\
                         x $\leq$ y \&\& y $\leq$ z $\implies$ x $\leq$ z & transitiveInequalityLET & x: Money                 \\
                                                                          &                         & y: Money                 \\
                                                                          &                         & z: Money                 \\
\rowcolor[HTML]{EFEFEF}  x $\geq$ y \&\& y $\geq$ z $\implies$ x $\geq$ z & transitiveInequalityGET & x: Money                 \\
\rowcolor[HTML]{EFEFEF}                                                   &                         & y: Money                 \\
\rowcolor[HTML]{EFEFEF}                                                   &                         & z: Money                 \\ \hline
\end{tabular}
\caption{Transitivity on \textit{Money}}
\label{tbl:ch4_money_transitivity}
\end{table}
\FloatBarrier
% End table
% Reasoning
Operations can be done on the \textit{Money} types. The transitive properties \cite{raftery2011perspective} on the (in)equality operators should still hold on the \textit{Money} type as we can still compare the \textit{Money} values. It is important to note that either the currency of the values should be the same, or the conversion rate should be taken into account with these operations.

% % % % % % % % % % % % %
% Subsection: Associativity
\subsection*{Associativity}
\label{ssct:4_associativity}
% Table
\FloatBarrier
\begin{table}[!ht]
\centering
\begin{tabular}{lll}
\hline
                        \textbf{Formula}           & \textbf{Property name}               & \textbf{Variable (Type)} \\ \hline
\rowcolor[HTML]{EFEFEF} (x + y) + z == x + (y + z) & associativeAddition                  & x: Money                 \\
\rowcolor[HTML]{EFEFEF}                            &                                      & y: Money                 \\
\rowcolor[HTML]{EFEFEF}                            &                                      & z: Money                 \\
                        (x * y) * z == x * (y * z) & associativeMultiplicationInteger1    & x: Integer               \\
                                                   &                                      & y: Integer               \\
                                                   &                                      & z: Money                 \\
\rowcolor[HTML]{EFEFEF} (x * y) * z == x * (y * z) & associativeMultiplicationInteger2    & x: Money                 \\
\rowcolor[HTML]{EFEFEF}                            &                                      & y: Integer               \\
\rowcolor[HTML]{EFEFEF}                            &                                      & z: Integer               \\
                        (x * y) * z == x * (y * z) & associativeMultiplicationPercentage1 & x: Money                 \\
                                                   &                                      & y: Percentage            \\
                                                   &                                      & z: Integer               \\
\rowcolor[HTML]{EFEFEF} (x * y) * z == x * (y * z) & associativeMultiplicationpercentage2 & x: Integer               \\
\rowcolor[HTML]{EFEFEF}                            &                                      & y: Money                 \\
\rowcolor[HTML]{EFEFEF}                            &                                      & z: Percentage            \\ \hline
\end{tabular}
\caption{Associativity on \textit{Money}}
\label{tbl:ch4_money_associativity}
\end{table}
\FloatBarrier
% End table
% Reasoning
The law of associativity is known on addition and multiplication \cite{baumgart1961axioms}. It defines that the order in which certain operations are done, does not affect the result of the whole expression. As described in \autoref{ssct:4_commutativity} it is not possible to operate with multiplication with only \textit{Money} types. However, in \textit{Rebel} the same properties should hold when using different types, as shown in \autoref{tbl:ch4_money_associativity}.

% % % % % % % % % % % % %
% Subsection: Non-associativity
\subsection*{Non-associativity}
% Table
\FloatBarrier
\begin{table}[!ht]
\centering
\begin{tabular}{lll}
\hline
                        \textbf{Formula}           & \textbf{Property name} & \textbf{Variable (Type)} \\ \hline
\rowcolor[HTML]{EFEFEF} (x - y) - z != x - (y - z) & nonassociativity       & x: Money                 \\
\rowcolor[HTML]{EFEFEF}                            &                        & y: Money                 \\
\rowcolor[HTML]{EFEFEF}                            &                        & z: Money                 \\ \hline
\end{tabular}
\caption{Non-associativity on \textit{Money}}
\label{tbl:ch4_money_nonassociativity}
\end{table}
\FloatBarrier
% End table
% Reasoning
% in tegenstelling tot steen is geen steen stoep
In contrast to associativity (\autoref{ssct:4_associativity}), non-associativity described that the order does affect the result of the whole expression. As we can see in \autoref{tbl:ch4_money_nonassociativity} subtraction is a relation where this property holds. An exception to this would be when each argument is zero.

% % % % % % % % % % % % %
% Subsection: Distributivity
\subsection*{Distributivity}
% Table
\FloatBarrier
\begin{table}[!ht]
\centering
\begin{tabular}{lll}
\hline
                        \textbf{Formula}             & \textbf{Property name}  & \textbf{Variable (Type)} \\ \hline
\rowcolor[HTML]{EFEFEF} x * (y + z) == x * y + x * z & distributiveInteger1    & x: Money                 \\
\rowcolor[HTML]{EFEFEF}                              &                         & y: Integer               \\
\rowcolor[HTML]{EFEFEF}                              &                         & z: Integer               \\
                        (y + z) * x == y * x + z * x & distributiveInteger2    & x: Integer               \\
                                                     &                         & y: Money                 \\
                                                     &                         & z: Money                 \\
\rowcolor[HTML]{EFEFEF} x * (y + z) == x * y + x * z & distributivePercentage1 & x: Percentage            \\
\rowcolor[HTML]{EFEFEF}                              &                         & y: Money                 \\
\rowcolor[HTML]{EFEFEF}                              &                         & z: Money                 \\
                        (y + z) * x == y * x + z * x & distributivePercentage2 & x: Percentage            \\
                                                     &                         & y: Money                 \\
                                                     &                         & z: Money                 \\ \hline
\end{tabular}
\caption{Distributivity on \textit{Money}}
\label{tbl:ch4_money_distributivity}
\end{table}
\FloatBarrier
% End table
% Reasoning
The law of distributivity is another well-known law \cite{baumgart1961axioms}. Unlike associativity, the order does matter here when using different operations. These operations can be used on \textit{Money} and since we can see \textit{Money} as a number, this property is also expected on this type. Note that it is not possible to multiply \textit{Money} types with each other, so the variable types are an important part in these properties as described in \autoref{tbl:ch4_money_distributivity}.

% % % % % % % % % % % % %
% Subsection: Identity
\subsection*{Identity}
% Table
\FloatBarrier
\begin{table}[!ht]
\centering
\begin{tabular}{lll}
\hline
                        \textbf{Formula} & \textbf{Property name}  & \textbf{Variable (Type)} \\ \hline
\rowcolor[HTML]{EFEFEF} x + 0 == x       & additiveIdentity1       & x: Money                 \\
						0 + x == x       & additiveIdentity2       & x: Money                 \\
\rowcolor[HTML]{EFEFEF} x * 1 == x       & multiplicativeIdentity1 & x: Money                 \\
                        1 * x == x       & multiplicativeIdentity2 & x: Money                 \\ \hline
\end{tabular}
\caption{Identity on \textit{Money}}
\label{tbl:ch4_money_identity}
\end{table}
\FloatBarrier
% End table
% Reasoning
The identity relation describes a function that returns the same value as the value that was given as input. For additive this entails the addition of zero to the input value and for multiplicative this entails multiplying the value by 1. Also the commutative property holds here, as the order does not matter in which this function is applied. Since it is not possible to just add 0 to a \textit{Money} value, the 0 showed in \autoref{tbl:ch4_money_identity} must be defined in a \textit{Money} format. Thus it must have the same currency as the parameter, with the amount of 0. For multiplication the \textit{Integer} type can be used.

% % % % % % % % % % % % %
% Subsection: Inverse
\subsection*{Inverse}
% Table
\FloatBarrier
\begin{table}[!ht]
\centering
\begin{tabular}{lll}
\hline
                        \textbf{Formula} & \textbf{Property name} & \textbf{Variable (Type)} \\ \hline
\rowcolor[HTML]{EFEFEF} x + (-x) == 0    & additiveInverse1       & x: Money                 \\
                        (-x) + x == 0    & additiveInverse2       & x: Money                 \\ \hline
\end{tabular}
\caption{Inverse on \textit{Money}}
\label{tbl:ch4_money_inverse}
\end{table}
\FloatBarrier
% End table
% Reasoning
The inverse relation describes for additivity that using addition with the input parameter and the negative of the input parameter, results in the value zero. Note that the operation is used on the \textit{Money} type, so the expected value is 0 with the same currency as the currency of the input parameter. Although the inverse relation could also be used with multiplication and division (defined as \code{x*(1/x) == 0}), it is not possible to use this definition in this project. As we cannot divide something with the \textit{Money} type, which is why we only define the inverse relation using addition on \textit{Money}.

% % % % % % % % % % % % %
% Subsection: Additivity
\subsection*{Additivity}
% Table
\FloatBarrier
\begin{table}[!ht]
\centering
\begin{tabular}{lll}
\hline
                        \textbf{Formula}                             & \textbf{Property name} & \textbf{Variable (Type)} \\ \hline
\rowcolor[HTML]{EFEFEF} x == y $\implies$ x + z == y + z             & additive               & x: Money                 \\
\rowcolor[HTML]{EFEFEF}                                              &                        & y: Money                 \\
\rowcolor[HTML]{EFEFEF}                                              &                        & z: Money                 \\
                        x == y \&\& z == a $\implies$ x + z == y + a & additive4params        & x: Money                 \\
                                                                     &                        & y: Money                 \\
                                                                     &                        & z: Money                 \\
                                                                     &                        & a: Money                 \\ \hline
\end{tabular}
\caption{Additivity on \textit{Money}}
\label{tbl:ch4_money_additivity}
\end{table}
\FloatBarrier
% End table
% Reasoning
Addition was earlier mentioned for the Commutativity (\autoref{ssct:4_commutativity}) and Associativity (\autoref{ssct:4_associativity}) properties. The properties mentioned here extend these by defining properties that are true when the input values are equal. When using the addition operator such that the resulting values on both sides remain the same, as shown in \autoref{tbl:ch4_money_symmetry}, it should not break the equality property on the resulting values.

% % % % % % % % % % % % %
% Subsection: Property of Zero
\subsection*{Property of Zero}
% Table
\FloatBarrier
\begin{table}[!ht]
\centering
\begin{tabular}{lll}
\hline
                        \textbf{Formula} & \textbf{Property name}     & \textbf{Variable (Type)} \\ \hline
\rowcolor[HTML]{EFEFEF} x * 0 == 0       & multiplicativeZeroProperty1 & x: Money                 \\
                        0 * x == 0       & multiplicativeZeroProperty2 & x: Money                 \\ \hline
\end{tabular}
\caption{Property of Zero on \textit{Money}}
\label{tbl:ch4_money_propertyzero}
\end{table}
\FloatBarrier
% End table
% Reasoning
The property of zero on multiplication states that if something is multiplied by zero, the result will always be zero. Since Rebel allows the use of multiplication on the \textit{Money} type, it's possible to multiply it by 0. Since the value of a \textit{Money} variable is based on a decimal number, this property states that the value will be exactly 0 (or 0.00 in the representation of a \textit{Money} value). But it should not contain any decimal number\unsure{Not sure if decimal is right here? - Fraction}. \todo{Note the order doesnt matter}

% % % % % % % % % % % % %
% Subsection: Division
\subsection*{Division}
% Table
\FloatBarrier
\begin{table}[!ht]
\centering
\begin{tabular}{lll}
\hline
                        \textbf{Formula}                 & \textbf{Property name} & \textbf{Variable (Type)} \\ \hline
\rowcolor[HTML]{EFEFEF} x * y == z $\implies$ x == z / y & division1              & x: Money                 \\
\rowcolor[HTML]{EFEFEF}                                  &                        & y: Integer               \\
\rowcolor[HTML]{EFEFEF}                                  &                        & y: Money                 \\
                        x == z * y $\implies$ x / y == z & division2              & x: Money                 \\
                                                         &                        & y: Integer               \\
                                                         &                        & y: Money                 \\ \hline
\end{tabular}
\caption{Division on \textit{Money}}
\label{tbl:ch4_money_division}
\end{table}
\FloatBarrier
% End table
% Reasoning
When using division with the \textit{Money} type, it is not possible to use a \textit{Money} value as denominator. However, a Money type can be dived by an Integer, thus we can define the division properties by using both the \textit{Money} and \textit{Integer} type. Note that the denominator cannot be zero, as division by zero is not possible. % Maybe theres a source for division by zero? :)

% % % % % % % % % % % % %
% Subsection: Trichotomy
\subsection*{Trichotomy}
% Table
\FloatBarrier
\begin{table}[!ht]
\centering
\begin{tabular}{lll}
\hline
                        \textbf{Formula}             & \textbf{Property name} & \textbf{Variable (Type)} \\ \hline
\rowcolor[HTML]{EFEFEF} x $<$ y || x == y || x $>$ y & trichotomy             & x: Money                 \\
\rowcolor[HTML]{EFEFEF}                              &                        & y: Money                 \\ \hline
\end{tabular}
\caption{Trichotomy on \textit{Money}}
\label{tbl:ch4_money_trichotomy}
\end{table}
\FloatBarrier
% End table
% Reasoning
The law of trichotomy defines that for every pair of arbitrary real numbers, exactly one of the relations $<$, ==, $>$ holds. We can define a property for this on money, as shown in \autoref{tbl:ch4_money_trichotomy}.

% % % % % % % % % % % % % % % % % % % % % % % % % % % % % % % % % % % % %
% Section: Analysis
\section{Analysis}
In this chapter, we focused on the properties of the custom types in \textit{Rebel}. The research question lead as follows:\rqOne\\
\\
Since there were no definition available of the types in \textit{Rebel}, it was required to define the properties in detail. Describing what the expected behaviour is when operating with the types and which operations are allowed on each type. In this chapter we have defined a set of properties that are expected to hold with Rebel.\\
\\
The properties are based on the axioms in algebra, but the requirements of \textit{Rebel} have been taken into account when it comes to using certain operations with certain types. These properties can be used to test the generator. However, it is not true that there are no bugs in the generator in case every property holds. It does mean that the test framework haven't found an error in the properties that it checks the generator on. Many properties can be defined within a language, the properties defined here are certainly not all the properties in \textit{Rebel}.

% % % % % % % % % % % % % % % % % % % % % % % % % % % % % % % % % % % % %
% Section: Threats to validity
\section{Threats to validity}
We defined a set of properties that are expected to hold in \textit{Rebel}. This can be seen as an incomplete set, as many more properties can hold and can be expected when using \textit{Rebel}. The set of properties we defined are aimed to cover many operations when using the \textit{Money} type, as this can be considered the most important type for a bank. This leads to some threats to validity.

% Lack of properties / missing properties
\subsection*{Lack of properties}
The properties that we have defined are based on the Axioms of Algebra. Many axioms exist, which brings a threat to this approach such that we may have missed certain properties that can trigger even more bugs. However, due to the current setup of the test framework, additional properties can easily be added by just appending these properties to the \textit{Rebel} specifications.

% Other components
\subsection*{Other components}
The current setup is intended to be used to test a certain component of the SUT. Namely, the types and the operations among these. The focus was to test (some of) the custom types that are available in \textit{Rebel}, the \textit{Money} type has been worked with thoroughly. More specifications and properties can be added to check every type that \textit{Rebel} supports. However, this only allows to reason about a specific component in the SUT. The test framework could be extended such that it can also test other components of the system. Such as the sync block definitions, defining actions that should happen synchronously. Or performance measures when interacting with the data in the system, which uses a database implementation. Currently, such components are not being tested, while these components are also important for a bank. This is left as future work.\\
\\
Although the test framework can be extended to also test other components of the SUT, it will not be possible to check all the components by using this approach. A generated system, for example, exposes its actions via a Rest API, which should be used to interact with the system. The tests that are done by using this approach do not check the implementation of this interface. Also, it is not able to check how, for example, multiple generated systems would integrate with each other and if this is done correctly.

% Invalid definitions
\subsection*{Invalid definitions}
We provided properties that should hold on \textit{Rebel} types. We already discussed the lack of properties, but it could be the case that our definitions are invalid or might be renewed in the future such that some properties are invalid. When this is the case, the properties could be adjusted to fit the updated definitions in this case.

% Substituting properties
\subsection*{Substituting properties}
In this chapter we defined many properties. Some of the defined properties might be overlapping each other, it might be possible to substitute the amount of properties that have to be defined. However, the purpose of each property on what it tests is different. So one should be carefull when determining some property unneccesary such that it can be substituted. For this thesis we used some well-known properties and defined those for the \textit{Rebel} types.
