% % % % % % % % % % % % % % % % % % % % % % % % % % % % % % % % % % % % %
% Chapter: Property definitions
% % % % % % % % % % % % % % % % % % % % % % % % % % % % % % % % % % % % %
\chapter{Property definitions} 
\label{cpt:4_properties}
% Definition: An Axiom is a mathematical statement that is assumed to be true.
% ( http://www.aaaknow.com/lessonFull.php?slug=propsCommAssoc )
% Sources:
% Axioms in algebra — where did they come from?
% http://www.jstor.org.proxy.uba.uva.nl:2048/stable/pdf/27956337.pdf?refreqid=excelsior%3A623b1c85f79f8cd18bc9d09560339b9b
% (simple formulas, history, and explanations)
% A perspective on the algebra of logic 
% http://www-tandfonline-com.proxy.uba.uva.nl:2048/doi/pdf/10.2989/16073606.2011.622856?needAccess=true
% (More formulas, deeper explanations) 
% Might contain more properties

% Another source was (but not a paper or so)
% http://www.aaamath.com/ac22.htm

% Another good source (book)
% Calculus, Vol. 1: One-Variable Calculus, with an Introduction to Linear Algebra 2nd Edition

\pinfo{Types and axioms}
For types like integers the axioms can be used to determine whether the implementation in the generated system is correct. These are most likely translated to integers in the generated system too, with the expectation that these have the same properties in Scala. For these types the known axioms could be used. However, it depends on what \textit{Rebel} defines as \textit{Integer}.\\
\\
\pinfo{Custom types, no properties yet. Focusing on Rebel types initially}
\textit{Rebel} has built-in types that are specifically designed for it's domain, the banking domain. Types like \textit{Money}, \textit{Percentage} and \textit{Iban} do not have a specification of what properties these have exactly. In this chapter we will define properties for these types, which we can then use to generate the test suite. The properties that we define are based on the known axioms in algebra \cite{baumgart1961axioms,raftery2011perspective,apostol2007calculus}. We provide a short explanation of why a certain property should also hold in these cases. We will focus on the custom types that are introduced in \textit{Rebel}.
%\todo[inline]{Percentage? Not yet, defining those requires properties again. And running the test suit again, evaluate again, etc.\\
%Describe what exactly is Percentage if we define it in a specification, which rules does it inherit?\\
%Percentage (expected): 2/more decimals, however its a precise amount. Calculating with it is expected in a result of the real value.\\
%Percentage (current): As it is defined now: full number, it can exceed 100. However it does not support decimals yet.}

% % % % % % % % % % % % % % % % % % % % % % % % % % % % % % % % % % % % %
% Section: Money specification
\section{Money specification}
\pinfo{Money definition, operations. And rounding up to business}
The Money type is a built-in type in Rebel, but there is no exact definition of it yet. It consists of a currency and an amount value, which is similar to the pattern that Fowler describes for a \textit{Money} type \cite{fowler2002patterns}. The author also describes the operations that can be done with the \textit{Money} type, which are: +, -, *, allocate, $<$, $>$, $\leq$, $\geq$ and =. However, Rebel also allows the division (/) operation on the \textit{Money} type. \improve{Allocate is division, but with the problem covered. Explain that instead of Rebel supporting division as additional. Describing this also shows we have read the part in Fowlers book.} It is unsupported to use these operations with \textit{Money} values using different currencies. This is due to the exchange rates between currencies, which can vary and are not implemented yet. The amount of a \textit{Money} value is often rounded when it is being represented to the user, as it could have many decimals. The representation of the \textit{Money} value is up to the business on how this is done, as there are multiple factors influencing this. Instead, we only focus on the internal value that is used when operating with it.\\
\\
\pinfo{Amount, not floating-point arithmetic}
Since the amount of a \textit{Money} value contains multiple decimals, it can be seen as a floating number. Does this mean that it inherits the computation properties of Floating-Point Arithmetic, as defined in the IEEE standards 754 or 854? Since the Rebel is a formal specification language for banking products, we don't expect that the described problems with this arithmetic are intended to exist on the Money type. Considering that that a high volume flows within a bank in terms of Money, using the Arithmetic properties can result in the known precision, overflow and underflow errors as described in \cite{goldberg1991every}. The precision errors for instance, should be avoided when using the \textit{Money} type. The author of \cite{fowler2002patterns} also describes that the intend of the \textit{Money} type is to avoid this:
\begin{quote}
\todo{Use quotes instead of "" (latex-advice)}
	"You should absolutely avoid any kind of floating point type, as that will introduce the kind of rounding problems that Money is intended to avoid." 
    -- Martin Fowler \cite{fowler2002patterns} %[p. 462] % Of pdf, ebook thing doesn't have page numbers
\end{quote}
\pinfo{Conclusion: Precise value, rounding only for division}
We say that the amount of a \textit{Money} value in Rebel should hold the exact value as if we would calculate the same expression by ourselves. Meaning that the precision errors would have to be prevented. An exception on this is when the result would be in the form of a fraction, for example 1/3. As this results in a value which we have to round up sometime. To fix this, we say that in this case we use x decimals when calculating, without rounding the x+1th decimal. When using properties with division, this should be taken into account in case the system fails on these tests. \todo{Define X, is this also 60 decimals?}\\
\\
\pinfo{Properties based on known axioms, but has restrictions}
The properties that we define on the \textit{Money} type of Rebel are based on the axioms of integers. It isn't possible to multiply two \textit{Money} types with each other to support the multiplicative property for example. Instead we can only multiply \textit{Money} with other types, such as \textit{Integer} and \textit{Percentage}, resulting in multiple property definitions when using different types. In the following sections we motivate the properties that we use during the project.
% % % % % % % % % % % % %
% Subsection: Reflexivity
\subsection{Reflexivity}
% Table
\begin{table}[h!]
\centering
\begin{tabular}{|lll|}
\hline
                        \textbf{Formula} & \textbf{Property name} & \textbf{Variable (Type)} \\ \hline
\rowcolor[HTML]{EFEFEF} x == x           & reflexiveEquality      & x: Money                 \\
                        x $\leq$ x       & reflexiveInequalityLET & x: Money                 \\
\rowcolor[HTML]{EFEFEF} x $\geq$ x       & reflexiveInequalityGET & x: Money                 \\ \hline
\end{tabular}
\caption{Reflexivity on \textit{Money}}
\label{tbl:ch4_money_reflexivity}
\end{table}
% Reasoning
\pinfo{Equality, currency and amount}
The reflexive property means a relation of a type with itself \cite{raftery2011perspective}. An instance of type \textit{Money} should be equal to itself. Taking both the currency and the amount into account. The inequality relations \textit{smaller or equal to} and \textit{greater or equal to} should hold too. As we can compare \textit{Money} variables and defined equality in the first row of the table.

% % % % % % % % % % % % %
% Subsection: Symmetry
\subsection{Symmetry}
% Table
\begin{table}[h!]
\centering
\begin{tabular}{|lll|}
\hline
                        \textbf{Formula}         & \textbf{Property name} & \textbf{Variable (Type)} \\ \hline
\rowcolor[HTML]{EFEFEF} x == y $\implies$ y == x & symmetric              & x: Money                 \\
\rowcolor[HTML]{EFEFEF}                          &                        & y: Money                 \\ \hline
\end{tabular}
\caption{Symmetry on \textit{Money}}
\label{tbl:ch4_money_symmetry}
\end{table}
% Reasoning
\pinfo{Equality, currency and amount}
Reflexivity already described equality on \textit{Money} when used on the same variable. When two different variables are used, the order should not matter and thus it should work in both ways. Which is known as the symmetric property \cite{raftery2011perspective}.

% % % % % % % % % % % % %
% Subsection: Antisymmetry
\subsection{Antisymmetry}
% Table
\begin{table}[h!]
\centering
\begin{tabular}{|lll|}
\hline
                        \textbf{Formula}                             & \textbf{Property name} & \textbf{Variable (Type)} \\ \hline
\rowcolor[HTML]{EFEFEF} x $\leq$ y \&\& y $\leq$ x $\implies$ x == y & antisymmetryLET        & x: Money                 \\
\rowcolor[HTML]{EFEFEF}                                              &                        & y: Money                 \\ 
                        x $\geq$ y \&\& y $\geq$ x $\implies$ x == y & antisymmetryGET        & x: Money                 \\
                                                                     &                        & y: Money                 \\ \hline
\end{tabular}
\caption{Antisymmetry on \textit{Money}}
\label{tbl:ch4_money_antisymmetry}
\end{table}
% Reasoning
The antisymmetric relation describes that whenever there is a relation from x to y and a relation from y to x, then x and y should be equal. The lower or equal then and greater or equal then relations fit in this category, as shown in \autoref{tbl:ch4_money_antisymmetry}. We can use these operations on the \textit{Money} type when both x and y use the same currency. This antisymmetric relation is also expected to hold, as \textit{Money} values should be equal when they are of the same currency and hold the same amount.

% % % % % % % % % % % % %
% Subsection: Transitivity
\subsection{Transitivity}
% Table
\begin{table}[h!]
\centering
\begin{tabular}{|lll|}
\hline
                         \textbf{Formula}                                 & \textbf{Property name}  & \textbf{Variable (Type)} \\ \hline
\rowcolor[HTML]{EFEFEF}  x == y \&\& y == z $\implies$ x == z             & transitiveEquality      & x: Money                 \\
\rowcolor[HTML]{EFEFEF}                                                   &                         & y: Money                 \\
\rowcolor[HTML]{EFEFEF}                                                   &                         & z: Money                 \\
                         x $<$ y \&\& y $<$ z $\implies$ x $<$ z          & transitiveInequalityLT  & x: Money                 \\
                                                                          &                         & y: Money                 \\
                                                                          &                         & z: Money                 \\
\rowcolor[HTML]{EFEFEF}  x $>$ y \&\& y $>$ z $\implies$ x $>$ z          & transitiveInequalityGT  & x: Money                 \\
\rowcolor[HTML]{EFEFEF}                                                   &                         & y: Money                 \\
\rowcolor[HTML]{EFEFEF}                                                   &                         & z: Money                 \\ 
                         x $\leq$ y \&\& y $\leq$ z $\implies$ x $\leq$ z & transitiveInequalityLET & x: Money                 \\
                                                                          &                         & y: Money                 \\
                                                                          &                         & z: Money                 \\
\rowcolor[HTML]{EFEFEF}  x $\geq$ y \&\& y $\geq$ z $\implies$ x $\geq$ z & transitiveInequalityGET & x: Money                 \\
\rowcolor[HTML]{EFEFEF}                                                   &                         & y: Money                 \\
\rowcolor[HTML]{EFEFEF}                                                   &                         & z: Money                 \\ \hline
\end{tabular}
\caption{Transitivity on \textit{Money}}
\label{tbl:ch4_money_transitivity}
\end{table}
% Reasoning
Operations can be done on the \textit{Money} types. The transitive properties \cite{raftery2011perspective} on the (in)equality operators should still hold on the \textit{Money} type as we can still compare the \textit{Money} values. It is important to note that either the currency of the values should be the same, or the conversion rate should be taken into account with these operations. 

% % % % % % % % % % % % %
% Subsection: Commutativity
\subsection{Commutativity}
\label{ssct:4_commutativity}
% Table
\begin{table}[h!]
\centering
\begin{tabular}{|lll|}
\hline
                        \textbf{Formula} & \textbf{Property name}               & \textbf{Variable (Type)} \\ \hline
\rowcolor[HTML]{EFEFEF} x + y == y + x   & commutativeAddition                  & x: Money                 \\
\rowcolor[HTML]{EFEFEF}                  &                                      & y: Money                 \\
                        x * y == y * x   & commutativeMultiplicationInteger1    & x: Integer               \\
                                         &                                      & y: Money                 \\
\rowcolor[HTML]{EFEFEF} x * y == y * x   & commutativeMultiplicationInteger2    & x: Money                 \\
\rowcolor[HTML]{EFEFEF}                  &                                      & y: Integer               \\
                        x * y == y * x   & commutativeMultiplicationPercentage1 & x: Percentage            \\
                                         &                                      & y: Money                 \\
\rowcolor[HTML]{EFEFEF} x * y == y * x   & commutativeMultiplicationpercentage2 & x: Money                 \\
\rowcolor[HTML]{EFEFEF}                  &                                      & y: Percentage            \\ \hline
\end{tabular}
\caption{Commutativity on \textit{Money}}
\label{tbl:ch4_money_commutativity}
\end{table}
% Reasoning
These properties are based on the commutative law \cite{baumgart1961axioms}. The result of an addition or multiplication does not vary when swapping the input variables. Because of the \textit{Money} type, we can only do addition on \textit{Money} values with other \textit{Money} values. For multiplication, there is no known value for multiplying two \textit{Money} variables. It is possible to multiply it by an \textit{Integer} or \textit{Percentage}. Also in this case, the order shouldn't matter if we would put the \textit{Money} value as first input parameter to multiplication or the other way around.

% % % % % % % % % % % % %
% Subsection: Anticommutativity
\subsection{Anticommutativity}
% Table
\begin{table}[h!]
\centering
\begin{tabular}{|lll|}
\hline
                        \textbf{Formula}  & \textbf{Property name} & \textbf{Variable (Type)} \\ \hline
\rowcolor[HTML]{EFEFEF} x - y == -(y - x) & anticommutativity      & x: Money                 \\
\rowcolor[HTML]{EFEFEF}                   &                        & y: Money                 \\ \hline
\end{tabular}
\caption{Anticommutativity on \textit{Money}}
\label{tbl:ch4_money_anticommutativity}
\end{table}
% Reasoning
In \autoref{ssct:4_commutativity} we described the commutative properties. Note that the operations only use addition and multiplication on this property. Subtraction is a operation that is anticommutative as swapping the order of the two argumates is negating the result. The anticommutative property thus negates the result of swapping the two arguments, intending to result in the actual value again, as shown in \autoref{tbl:ch4_money_anticommutativity}.

% % % % % % % % % % % % %
% Subsection: Associativity
\subsection{Associativity}
\label{ssct:4_associativity}
% Table
\begin{table}[h!]
\centering
\begin{tabular}{|lll|}
\hline
                        \textbf{Formula}           & \textbf{Property name}               & \textbf{Variable (Type)} \\ \hline
\rowcolor[HTML]{EFEFEF} (x + y) + z == x + (y + z) & associativeAddition                  & x: Money                 \\
\rowcolor[HTML]{EFEFEF}                            &                                      & y: Money                 \\
\rowcolor[HTML]{EFEFEF}                            &                                      & z: Money                 \\
                        (x * y) * z == x * (y * z) & associativeMultiplicationInteger1    & x: Integer               \\
                                                   &                                      & y: Integer               \\
                                                   &                                      & z: Money                 \\
\rowcolor[HTML]{EFEFEF} (x * y) * z == x * (y * z) & associativeMultiplicationInteger2    & x: Money                 \\
\rowcolor[HTML]{EFEFEF}                            &                                      & y: Integer               \\
\rowcolor[HTML]{EFEFEF}                            &                                      & z: Integer               \\
                        (x * y) * z == x * (y * z) & associativeMultiplicationPercentage1 & x: Money                 \\
                                                   &                                      & y: Percentage            \\
                                                   &                                      & z: Integer               \\
\rowcolor[HTML]{EFEFEF} (x * y) * z == x * (y * z) & associativeMultiplicationpercentage2 & x: Integer               \\
\rowcolor[HTML]{EFEFEF}                            &                                      & y: Money                 \\
\rowcolor[HTML]{EFEFEF}                            &                                      & z: Percentage            \\ \hline
\end{tabular}
\caption{Associativity on \textit{Money}}
\label{tbl:ch4_money_associativity}
\end{table}
% Reasoning
The law of associativity is known on addition and multiplication \cite{baumgart1961axioms}. It defines that the order in which certain operations are done, does not affect the result of the whole expression. As described in \autoref{ssct:4_commutativity} it is not possible to operate with multiplication with only \textit{Money} types. However, in Rebel the same properties should hold when using different types, as shown in \autoref{tbl:ch4_money_associativity}.

% % % % % % % % % % % % %
% Subsection: Non-associativity
\subsection{Non-associativity}
% Table
\begin{table}[h!]
\centering
\begin{tabular}{|lll|}
\hline
                        \textbf{Formula}           & \textbf{Property name} & \textbf{Variable (Type)} \\ \hline
\rowcolor[HTML]{EFEFEF} (x - y) - z != x - (y - z) & nonassociativity       & x: Money                 \\
\rowcolor[HTML]{EFEFEF}                            &                        & y: Money                 \\ \hline
\end{tabular}
\caption{Non-associativity on \textit{Money}}
\label{tbl:ch4_money_nonassociativity}
\end{table}
% Reasoning
% in tegenstelling tot steen is geen steen stoep
In contrast to associativity (\autoref{ssct:4_associativity}), non-associativity described that the order does affect the result of the whole expression. As we can see in \autoref{tbl:ch4_money_nonassociativity} subtraction is a relation where this property holds. An exception to this would be when each argument is zero.

% % % % % % % % % % % % %
% Subsection: Distributivity
\subsection{Distributivity}
% Table
\begin{table}[h!]
\centering
\begin{tabular}{|lll|}
\hline
                        \textbf{Formula}             & \textbf{Property name}  & \textbf{Variable (Type)} \\ \hline
\rowcolor[HTML]{EFEFEF} x * (y + z) == x * y + x * z & distributiveInteger1    & x: Money                 \\
\rowcolor[HTML]{EFEFEF}                              &                         & y: Integer               \\
\rowcolor[HTML]{EFEFEF}                              &                         & z: Integer               \\
                        (y + z) * x == y * x + z * x & distributiveInteger2    & x: Integer               \\
                                                     &                         & y: Money                 \\
                                                     &                         & z: Money                 \\
\rowcolor[HTML]{EFEFEF} x * (y + z) == x * y + x * z & distributivePercentage1 & x: Percentage            \\
\rowcolor[HTML]{EFEFEF}                              &                         & y: Money                 \\
\rowcolor[HTML]{EFEFEF}                              &                         & z: Money                 \\
                        (y + z) * x == y * x + z * x & distributivePercentage2 & x: Percentage            \\
                                                     &                         & y: Money                 \\
                                                     &                         & z: Money                 \\ \hline
\end{tabular}
\caption{Distributivity on \textit{Money}}
\label{tbl:ch4_money_distributivity}
\end{table}
% Reasoning
The law of distributivity is another well-known law \cite{baumgart1961axioms}. Unlike associativity, the order does matter here when using different operations. These operations can be used on \textit{Money} and since we can see \textit{Money} as a number, this property is also expected on this type. Note that it is not possible to multiply \textit{Money} types with each other, so the variable types are an important part in these properties as described in \autoref{tbl:ch4_money_distributivity}.

% % % % % % % % % % % % %
% Subsection: Identity
\subsection{Identity}
% Table
\begin{table}[h!]
\centering
\begin{tabular}{|lll|}
\hline
                        \textbf{Formula} & \textbf{Property name}  & \textbf{Variable (Type)} \\ \hline
\rowcolor[HTML]{EFEFEF} x + 0 == x       & additiveIdentity1       & x: Money                 \\
						0 + x == x       & additiveIdentity2       & x: Money                 \\
\rowcolor[HTML]{EFEFEF} x * 1 == x       & multiplicativeIdentity1 & x: Money                 \\
                        1 * x == x       & multiplicativeIdentity2 & x: Money                 \\ \hline
\end{tabular}
\caption{Identity on \textit{Money}}
\label{tbl:ch4_money_identity}
\end{table}
% Reasoning
The identity relation describes a function that returns the same value as the value that was given as input. For additive this entails the addition of zero to the input value and for multiplicative this entails multiplying the value by 1. Also the commutative property holds here, as the order does not matter in which this function is applied. Since it is not possible to just add 0 to a \textit{Money} value, the 0 showed in \autoref{tbl:ch4_money_identity} must be defined in a \textit{Money} format. Thus it must have the same currency as the parameter, with the amount of 0. For multiplication the \textit{Integer} type can be used.

% % % % % % % % % % % % %
% Subsection: Inverse
\subsection{Inverse}
% Table
\begin{table}[h!]
\centering
\begin{tabular}{|lll|}
\hline
                        \textbf{Formula} & \textbf{Property name} & \textbf{Variable (Type)} \\ \hline
\rowcolor[HTML]{EFEFEF} x + (-x) == 0    & additiveInverse1       & x: Money                 \\
                        (-x) + x == 0    & additiveInverse2       & x: Money                 \\ \hline
\end{tabular}
\caption{Inverse on \textit{Money}}
\label{tbl:ch4_money_inverse}
\end{table}
% Reasoning
The inverse relation describes for additivity that using addition with the input parameter and the negative of the input parameter, results in the value zero. Note that the operation is used on the \textit{Money} type, so the expected value is 0 with the same currency as the currency of the input parameter. Although the inverse relation could also be used with multiplication and division (defined as \code{x*(1/x) == 0}), it is not possible to use this definition in this project. As we cannot divide something with the \textit{Money} type, which is why we only define the inverse relation using addition on \textit{Money}.

% % % % % % % % % % % % %
% Subsection: Additivity
\subsection{Additivity}
% Table
\begin{table}[h!]
\centering
\begin{tabular}{|lll|}
\hline
                        \textbf{Formula}                             & \textbf{Property name} & \textbf{Variable (Type)} \\ \hline
\rowcolor[HTML]{EFEFEF} x == y $\implies$ x + z == y + z             & additive               & x: Money                 \\
\rowcolor[HTML]{EFEFEF}                                              &                        & y: Money                 \\
\rowcolor[HTML]{EFEFEF}                                              &                        & z: Money                 \\
                        x == y \&\& z == a $\implies$ x + z == y + a & additive4params        & x: Money                 \\
                                                                     &                        & y: Money                 \\
                                                                     &                        & z: Money                 \\ 
                                                                     &                        & a: Money                 \\ \hline
\end{tabular}
\caption{Additivity on \textit{Money}}
\label{tbl:ch4_money_additivity}
\end{table}
% Reasoning
Addition was earlier mentioned for the Commutativity (\autoref{ssct:4_commutativity}) and Associativity (\autoref{ssct:4_associativity}) properties. The properties mentioned here extend these by defining properties that are true when the input values are equal. When using the addition operator such that the resulting values on both sides remain the same, as shown in \autoref{tbl:ch4_money_symmetry}, it should not break the equality property on the resulting values.

% % % % % % % % % % % % %
% Subsection: Property of Zero
\subsection{Property of Zero}
% Table
\begin{table}[h!]
\centering
\begin{tabular}{|lll|}
\hline
                        \textbf{Formula} & \textbf{Property name}     & \textbf{Variable (Type)} \\ \hline
\rowcolor[HTML]{EFEFEF} x * 0 == 0       & multiplicativeZeroProperty1 & x: Money                 \\
                        0 * x == 0       & multiplicativeZeroProperty2 & x: Money                 \\ \hline
\end{tabular}
\caption{Property of Zero on \textit{Money}}
\label{tbl:ch4_money_propertyzero}
\end{table}
% Reasoning
The property of zero on multiplication states that if something is multiplied by zero, the result will always be zero. Since Rebel allows the use of multiplication on the \textit{Money} type, it's possible to multiply it by 0. Since the value of a \textit{Money} variable is based on a decimal number, this property states that the value will be exactly 0 (or 0.00 in the representation of a \textit{Money} value). But it should not contain any decimal number\unsure{Not sure if decimal is right here? - Fraction}. \todo{Note the order doesnt matter}

% % % % % % % % % % % % %
% Subsection: Division
\subsection{Division}
% Table
\begin{table}[h!]
\centering
\begin{tabular}{|lll|}
\hline
                        \textbf{Formula}                 & \textbf{Property name} & \textbf{Variable (Type)} \\ \hline
\rowcolor[HTML]{EFEFEF} x * y == z $\implies$ x == z / y & division1              & x: Money                 \\
\rowcolor[HTML]{EFEFEF}                                  &                        & y: Integer               \\ 
\rowcolor[HTML]{EFEFEF}                                  &                        & y: Money                 \\ 
                        x == z * y $\implies$ x / y == z & division2              & x: Money                 \\
                                                         &                        & y: Integer               \\
                                                         &                        & y: Money                 \\ \hline
\end{tabular}
\caption{Division on \textit{Money}}
\label{tbl:ch4_money_division}
\end{table}
% Reasoning
When using division with the \textit{Money} type, it is not possible to use a \textit{Money} value as denominator. However, a Money type can be dived by an Integer, thus we can define the division properties by using both the \textit{Money} and \textit{Integer} type. Note that the denominator cannot be zero, as division by zero is not possible. % Maybe theres a source for division by zero? :)

% % % % % % % % % % % % %
% Subsection: Trichotomy
\subsection{Trichotomy}
% Table
\begin{table}[h!]
\centering
\begin{tabular}{|lll|}
\hline
                        \textbf{Formula}             & \textbf{Property name} & \textbf{Variable (Type)} \\ \hline
\rowcolor[HTML]{EFEFEF} x $<$ y || x == y || x $>$ y & trichotomy             & x: Money                 \\
\rowcolor[HTML]{EFEFEF}                              &                        & y: Money                 \\ \hline
\end{tabular}
\caption{Trichotomy on \textit{Money}}
\label{tbl:ch4_money_trichotomy}
\end{table}
% Reasoning
The law of trichotomy defines that for every pair of arbitrary real numbers, exactly one of the relations $<$, ==, $>$ holds. We can define a property for this on money, as shown in \autoref{tbl:ch4_money_trichotomy}.



% Old content

%\change{Axiom is a specification of a property.}
%The properties that we test are based on the properties of Axioms in Algebra. We assume that any expression that Rebel allows in its syntax, should be a valid expression. \change{In syntax more is allowed. Syntax is first. Then type checker and then its valid. Maybe reasoning might adds more...} Rebel supports multiple types for variables, where some types can be used in conjunction with others. The property definitions can depend on the combination of types too.\change{Type system, just a term for it, Jargon. Also check other papers how these define it} For example when we have a value x of type `Money` and value y and z of type `Percentage`, then it is not the case that x*(y*z) equals (x*y)*z. But this property would hold if all the values x, y and z were of type `Integer`.
%\todo{Tabular representation with samples?}
% % % % % % % % % % % % % % % % % % % % % % % % % % % % % % % % % % % % %
% Section: Reflexitivity
%\section{Reflexitivity}
%A type is equal to itself

% % % % % % % % % % % % % % % % % % % % % % % % % % % % % % % % % % % % %
% Section: Associative addition and multiplication
%\section{Associative addition and multiplication}
%The expected result when operating with money types is that the total amount remains the same despite of the order. This is also a property in the Axioms of Algebra on Integers, which we can relate to the Money type. For example, if we first receive 50 euros, and then 40 euros, we have a total of 90 euros. But its also expected that we have 90 euros in case we first receive the 40 euros and then the 50 euros. Although this holds with the addition of money, it doesn't hold for subtraction. Furthermore we cannot multiply money with itself. We can however multiply between Money and an Integer, or with Money and Percentage. For these cases we also expect that the result of money multiplied by an int will be the same as multiplying the int with the money.\change{Why 'int' here? Kevin: read as Rebel int and Integer as scala int...}\\
%\\
% % % % % % % % % % % % % % % % % % % % % % % % % % % % % % % % % % % % %
% Section: Distributive multiplication
%\section{Distributive multiplication}
%etc...
