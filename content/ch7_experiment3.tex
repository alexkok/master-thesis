% % % % % % % % % % % % % % % % % % % % % % % % % % % % % % % % % % % % %
% Chapter: Experiment 3
% % % % % % % % % % % % % % % % % % % % % % % % % % % % % % % % % % % % %
\chapter{Experiment 3: Improving the value generation}
\label{cpt:experiment3}
In the second experiment (\autoref{cpt:experiment2}) it was found that the
property definition for using division with the \textit{Monye} type wasn't clear
enough in that it couldn't be satisfied the way how it was specified. In this
experiment we aim to improve this. We update the existing definition such that
it can now be correctly.\\
\\
Another result from the second experiment was that the value generation was not
very dynamic when new properties were being added. It should be possible to add
additional properties to the specification, which are then being tested
automatically by the test framework. To do this, the value generator will be
updated too, such that it uses the preconditions in the event definitions to
determine the parameter values. Additional properties can then be added to test
the generator.

% - Updating properties, more for division
% - Also adding rounding definition. Now clearer and division actually being tested
% - Improving dynamicallity
% ...

% % % % % % % % % % % % % % % % % % % % % % % % % % % % % % % % % % % % %
% Section: Method
\section{Method}
The property definition of division will be updated for this experiment, because
the first definition did not take the division problem into account. There was
also no definition for rounding the value, as the value was expected to hold the
exact value. We will update the definition by using a round function. By using
this, the fixed properties can be defined which can then be tested by the test
framework. These are fixed because these property definitions turned out to be
incomplete or incorrect in the second experiment (\autoref{cpt:experiment2}).\\
\\
Besides updating the property definitions that were using division, additional
property definitions are being added to test more of the generator. However, as
we have seen in the second experiment, the values generator was not very
dynamic. Which resulted in that it had to be modified in case an implicative
property is being added to the specification. This is the second thing that
should be improved, such that additional (implicative) properties do not require
modifications to the values generator.

% % % % % % % % % % % % %
% Subsection: Updating property definitions
\subsection{Updating property definitions}
\todo{Source for division - rounding?}
\todo{Rephrase}
\pinfo{Round method}
Initially there were only 2 property definitions that were using division, which both are being updated. In order to check for the values such that the property definition can be used, we implement a \code{round()} method in the \textit{Rebel} specification. This \textit{round} method is intended to return a value which can be used to define the expected behaviour when using division with the \textit{Money} type.\\
\\
\pinfo{Defining, updating and adding more}
In addition to updating the existing properties that are using division, more properties can be added. For example, properties of inequality when using division, as the ones that were defined for
division only used equality. Also other properties like the subtraction property can
be added and more definitions for multiplication and additivity
can be added to the existing list of property definitions that we defined for \textit{Rebel}. The
additional property definitions for division, multiplication, additivity and
subtraction fall under the ``Properties of equality and inequality'' category
and are defined in \autoref{ssct:properties_definitions_additionalproperties},
along with the updated definitions for the existing property definitions that
use division.\\
\\
\pinfo{Round method implementation + why}
The properties using division are now using the \code{round()} method in its d
efinition. \textit{Rebel} does not provide a way to round a value, which is why
we need to define the function in the specification. In \textit{Rebel} a
function is defined as an expression that is being executed whenever the
function is being called. Unfortunately, there is no way in \textit{Rebel} to
define the \textit{Scala} implementation of this \textit{round} function. As a
workaround, we define the implementation as a \textit{String} and modify the
generator such that the function's implementation is the content of the
\textit{String} (removing the quotes). The \textit{round} method rounds the
\textit{Money} value to a maximum of 4 decimals. The fifth decimal is being
rounded by using ``HALF_UP'' technique
\todo{Maybe a source or so? Or common knowledge?}. The function implementation
in the \textit{Rebel} specification is shown in
\autoref{lst:experiment3_rebel_round_implementation}.
% Listing
\FloatBarrier
\begin{sourcecode}[!ht]
\begin{lstlisting}[language=Rebel]
function round(money: Money): Money =
    "money.currency(money.amount.setScale(4, RoundingMode.HALF_UP))";
\end{lstlisting}
\caption{The updated event definition of the \textit{Symmetric} property}
\label{lst:experiment3_rebel_round_implementation}
\end{sourcecode}
\FloatBarrier
% End listing



% % % % % % % % % % % % %
% Improving dynamicallity
\subsection{Improving dynamicallity}
...

% - Values generation could be done different: Tried SMT< didnt work > why?
% - Generating a limited version by ourselves. Current properties can be covered, and more can be added that are supported too.
% - Describe how
% - Describe rounding too (modified generator for that)
% - Describe limitations
...

% Dynamicallity
\subsection*{Dynamicallity}
\pinfo{Not very dynamic}
The implicative properties are now being tested such that the condition of the
if-clause is being satisfied. However, the values generator that is being used
for this is not very dynamic. As it simply checks for the event name and throws
an exception in case this is not defined for the property yet. This means that
adding new property definitions to the test framework requires a modification to
the value generator in case of an implicative property. This makes the test
framework less dynamic when adding new properties that should be tested on the
generator.\\
\\
\pinfo{Fix: using the preconditions}
To fix this, it would be better to generate the values by interpreting the
preconditions such that random values can be determined based on a certain
condition. Since the \textit{Rebel} toolchain already makes use of a bounded
model checker to check a specification, this could be used to simply translate
an expression and retrieve values for which the condition holds.\\
\\
\pinfo{Checked using Z3, but returns same values all the time}
We have looked into this, by using the \textit{Z3} solver. However, the solver
always returns the same number when executing it multiple times. Which means
that the 100 values that we would ask from the generator, will be exactly the
same. A workaround would be to then add the number that was received earlier as
another constraint, such that 100 unique values are being retrieved. But the
problem still remains, as executing the same script multiple times results in
the same values. When changing the seed of the random generator that is being
used, it will return different values. In order to make the test framework
execute this behaviour, the value generator has to be changed to integrate with
the solver. Additionally, this could have a huge effect on time increase that
the test framework needs to successfully finish.\\
\\






% % % % % % % % % % % % % % % % % % % % % % % % % % % % % % % % % % % % %
% Section: Results
\section{Results}
% - Run, tests succeed (hopefully)
...

% % % % % % % % % % % % % % % % % % % % % % % % % % % % % % % % % % % % %
% Section: Analysis
\section{Analysis}
% - Test succeeding provided rounding
% - Case of rounding down, it fails (bigger number shows difference)
% - > Thus probably division does this too usualy. FLooring causes difference
% - Adding properties didn't require a change in the value generator anymore
% ?- Describe extra failing cases in case there were more
...

% % % % % % % % % % % % % % % % % % % % % % % % % % % % % % % % % % % % %
% Section: Evaluation
\section{Evaluation}
% - Same coverage
% - No extra bugs found (or if we do with new properties, describe those)
...

% % % % % % % % % % % % % % % % % % % % % % % % % % % % % % % % % % % % %
% Section: Threats to validity
\section{Threats to validity}
% - Implementation error in our generator. However, we generate random values every run. (unlike Z3)
% - Existing approaches are available too, we only checked the Z3. But SageMath or others are possibilities too. However, this requires us to make such a tool compatible with the test framework. Our own implementation has shown to work for the properties we have defined. In case of more complex properties, it might be more useful to use another tool to determine these values.
...


%
%\pinfo{Other possibilities, future work}
%There are other solvers available too, or other methods to generate values that
%match the condition. It would be useful to make the test framework more dynamic
%when such properties are being used.
