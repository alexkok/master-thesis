% % % % % % % % % % % % % % % % % % % % % % % % % % % % % % % % % % % % %
% Chapter: Discussion
% % % % % % % % % % % % % % % % % % % % % % % % % % % % % % % % % % % % %
\chapter{Discussion}
\label{cpt:discussion}
In this chapter we discuss the research questions.

% % % % % % % % % % % % % % % % % % % % % % % % % % % % % % % % % % % % %
% RQ 1: Which properties
\section{RQ 1: \rqOne{}}
There were no existing properties defined for \textit{Rebel} that were expected
to hold. We have defined many properties for \textit{Rebel}. With the focus on
the \textit{Money} type, which is considered the most important type for a
bank.\\
\\
Many properties have been defined, but are these all the properties? Is each
definition correct? Some properties might be considered incorrect or overlapping
with others. We have seen this with the defintion of \textit{division} that was
incorrect initially, which was encountered in the second experiment. In the
third experiment the property definitions for \textit{division} have been
updated. Additionally, extra properties were added to the list for the third
experiment, \textit{additivity}, \textit{subtraction} and
\textit{multiplication}. This showed how additional properties could be added to
the test framework to test the generator. In case of incorrect, missing or
overlapping property definitions, the current set of property definitions could
be updated as we have done with the third experiment. There can be many
combinations among the different types of \textit{Rebel} and the supported
operators. Therefore, the set of properties that we have defined in
\autoref{cpt:properties} is not the complete set of expected properties in
\textit{Rebel}.\\
\\
Each property definition is defined as an expression, the test framework is
currently limited to checking the generator on these kind of properties. We can also
think of other properties that should hold in \textit{Rebel}. For example, the
behaviour of sync block: what are the properties of the sync block in
\textit{Rebel}? Can these be described in a \textit{Rebel} specification? If so,
the current test framework would not support that definition because it can
currently only cope with expression-based properties.

% % % % % % % % % % % % % % % % % % % % % % % % % % % % % % % % % % % % %
% RQ 2: How we test
\section{RQ 2: \rqTwo{}}
We have described a way how the generator can be tested by using property-based
testing. In order to check the generator, a \textit{Rebel} specification is
created containing the property definitions. This specification is then used by
the test framework to generate tests and run these tests against the generated
system.\\
\\
The initial version of the test framework used random values to test each property. The first experiment showed that this doesn't work for the implicative properties. The if-clause of these implicative properties were not being triggered when using random values. In the second experiment the test framework was improved. The input values where determined such that the condition of the implicative property was being satisfied. Additionally, these values were being mutated by a random operation, to keep the randomness of the input values. This improvement to the test framework revealed some incorrect property definitions. The implementation of how the input values were determined was not very dynamic. Because every time an implicative property was being added to the specification, an update was required to the test framework.\\
\\
In the third experiment the input values were determined in a more dynamic way, by using the preconditions to determine the input values. This requires a property definition to define its preconditions. As a result, it was not required to update the test framework when implicative properties were being added. This is shown in the third experiment.\\
\\
We evaluated the test framework in each experiment by looking at the test coverage and the number of bugs that have been detected. The coverage details showed that in the first experiment the if-clause of the implicative properties was not being triggered. Which lead to an improved version in the second and third experiment. Furthermore the branch coverage reported a consistent 50\% over the second and third experiment, which is caused by the implicative properties. The else-clause is not being tested, but that is also not the intention of that property.\\
\\
The overall test coverage for each experiment was roughly the same (87.80\%, 88.87\% and 88.28\%). This shows that despite of the improvements made to the test framework, the test coverage did not change. This is caused by the fact that each experiment is different:
\begin{itemize}
  \item The first experiment used random input values for each property. The else-clause of the implicative properties were defined as \code{true}, resulting that each implicative property was considered correct.
  \item In the second experiment the else-clause of the implicative properties were set to \code{false}. The input values are generated such that these satisfy the condition of the if-clause. These values are also being mutated when running the test, to keep the actual values random.
  \item In the third experiment some incorrectly defined properties were updated and additional properties have been added.
\end{itemize}
The test coverage of the first experiment would be considerably lower in case the else-clause of the implicative properties would be \code{false}. This would result in that every test concerning an implicative property would fail. Although the result of this would then be that there are many false-positives, because it just returns \textit{false} in its definition.\\
\\
The test coverage over the second and third experiment are roughly the same (only 0.59\% difference). But the third experiment tested 17 more properties than the second experiment. This shows that adding additional properties does not have much impact on the test coverage. The additional properties are being checked automatically by the test framework because these properties are defined in the \textit{Rebel} specification.

% % % % % % % % % % % % % % % % % % % % % % % % % % % % % % % % % % % % %
% RQ 3: Number and kind of bugs
\section{RQ 3: \rqThree{}}
Multiple bugs were found using property-based testing to check the generator.
The generator failed to satisfy a total of 9 properties that we have defined.
Some properties triggered different kind of bugs. The bugs that were found can
be separated into the following categories:
\begin{description}
  \item[~~~~Compilation errors:] Errors that make the generated system unable to compile, resulting that the generated system cannot be used.
  \item[~~~~Overflow/underflow errors:] Errors happening because of a limit that has been reached on specific types.
  \item[~~~~Precision errors:] Errors causing an unexpected outcome value when being calculated.
\end{description}

\subsection*{Compilation errors}
\pinfo{Compilation error, Squants issue}
In the second experiment, the test framework was initially being terminated because of a
compilation error. Although one assumption was that the generated system should
be able to compile, another assumption was that the specification was
consistent. The specification containing all the properties is consistent,
as \textit{Rebel} did not report any syntactic or semantic errors with the type
checker. The test framework is thus able to find such compilation errors. However,
there can be many more compilation errors for which we do not check, which is
also out of the scope of this thesis. The cause of this error was actually
caused by an implementation error in an open-source library that the generated
system used, called \textit{Squants}~\cite{siteSquants2017}. To fix this, we
created a \textit{Github}
issue\footnote{https://github.com/typelevel/squants/issues/281} describing the
problem. So that this can be fixed in the next release of the library.

\subsection*{Overflow/underflow errors}
\pinfo{Overflow/underflow errors, discussion and unclear definition}
The overflow/underflow errors are caused because of the use of the
\textit{Integer} type. On one hand, this could be prevented by checking the
operations beforehand for overflow errors. On the other hand, this could be the
expected behaviour when an \textit{Integer} is being used in \textit{Rebel}. As
\textit{Integers} are known have such limits that are also dependent on the
platform the application is run~\cite{wang2009intscope}. However, in
\textit{Rebel} there is currently no other type that can be used to hold a
bigger number. For example in \textit{Java} there is \textit{Long} for a larger
number, or \textit{BigDecimal} for even bigger numbers. This would mean that
\textit{Rebel} does not support big numbers, or that a custom type must be used
for this. Considering that \textit{Rebel} does not provide another type for
bigger numbers, the \textit{Integer} type is considered to also hold bigger
numbers. Since the specification is about banking products and it probably could
happen that a big number is needed. After all, we cannot know this for sure, as
\textit{Rebel} does not provide a specification yet of each of type in
\textit{Rebel}.

\subsection*{Precision errors}
\pinfo{Precision errors, Squants issue}
As we have seen, the \textit{Money} precision errors both occurred when using
\textit{Percentage} values as well as when using \textit{Integer} values to
operate with the \textit{Money} type. Since we were able to reproduce the issue
in a clean \textit{REPL} environment, the problem existed in the open-source
library, called \textit{Squants}, that was used for the \textit{Money} type. In
order to solve this problem, we created an issue on
\textit{Github}\footnote{https://github.com/typelevel/squants/issues/265}
related to the precision problems on the \textit{Money} type. A contributor
responded and fixed the issue within a day, the change will be included in the
next version of the library (1.4). So it is required to update this library in
order to let these tests in our test suite pass.

% Advice?
% To think of: Where there some things encountered during the project, which might require attention, but haven't or won't be thoroughly tested with this approach. Or do some additions cover these?
% > Squants
%	- Some bugs, maybe there are more in there
% 	- Snapshot release unavailable for some time? (1.4.0-SNAPSHOT was not reachable, had to compile it ourselves)
