% % % % % % % % % % % % % % % % % % % % % % % % % % % % % % % % % % % % %
% Chapter: Discussion
% % % % % % % % % % % % % % % % % % % % % % % % % % % % % % % % % % % % %
\chapter{Discussion}
\label{cpt:discussion}
In this chapter we discuss the research questions.

% % % % % % % % % % % % % % % % % % % % % % % % % % % % % % % % % % % % %
% RQ 1: Which properties
\section{RQ 1: \rqOne{}}
\textit{Rebel} did not have any definitions of which properties are expected to
hold on a specification. Since we are using property-based testing to check the
generator, it was required to define the expected properties first. The
definitions of each property can be found in \autoref{cpt:properties}.\\
\\
Many properties were defined during this thesis, but these are certainly not
all the properties that exist for \textit{Rebel}. We focused on the
\textit{Money} type, which is considered the most important type for a bank.
Additional properties can always be added to the test framework.

% % % % % % % % % % % % % % % % % % % % % % % % % % % % % % % % % % % % %
% RQ 2: How to test
\section{RQ 2: \rqTwo{}}
We have described a way how the generator could be tested by using
property-based testing. In order to check the generator, a \textit{Rebel}
specification was required. The generated system was used to determine the
results, allowing to reason about the generator. The test setup could be divided
into the following phases:
\begin{enumerate}
  \item \tfPhaseOne{}
  \item \tfPhaseTwo{}
  \item \tfPhaseThree{}
  \item \tfPhaseFour{}
  \item \tfPhaseFive{}
\end{enumerate}
The first phase had to be done manually. For the other phases, we introduced
our test framework to automate these steps in order to detect the bugs as
automatically as possible.

% % % % % % % % % % % % % % % % % % % % % % % % % % % % % % % % % % % % %
% RQ 3: Number and kind of bugs
\section{RQ 3: \rqThree{}}
Multiple bugs were found using property-based testing to check the generator.
The generator failed to satisfy a total of 9 properties that we have defined.
Some properties triggered different kind of bugs. The bugs that were found can
be separated into the following categories:
\begin{description}
  \item[~~~~Compilation errors:] Errors that make the generated system unable to compile, and thus it cannot be used.
  \item[~~~~Overflow/underflow errors:] Errors happening because of a limit that has been reached on specific types.
  \item[~~~~Precision errors:] Errors causing an unexpected outcome value when being calculated.
\end{description}
\todo{Add division}
\subsection*{Compilation errors}
\pinfo{Compilation error, Squants issue}
The fact that the test framework initially was being terminated was because of a
compilation error. Although one assumption was that the generated system should
be able to compile, another assumption we made was that the specification was
consistent. The specification we created for all the properties is consistent,
as \textit{Rebel} did not report any syntactic or semantic errors with the type
checker. The test framework is thus able to find such compilation errors. However,
there can be many more compilation errors for which we do not check, which is
also out of the scope of this thesis. The cause of this error was actually
caused by an implementation error in an open-source library that the generated
system used, called \textit{Squants}~\cite{siteSquants2017}. To fix this, we
created a \textit{Github}
issue\footnote{https://github.com/typelevel/squants/issues/281} describing the
problem. So that this can be fixed in the next release of the library.

\subsection*{Overflow/underflow errors}
\pinfo{Overflow/underflow errors, discussion and unclear definition}
The overflow/underflow errors are caused because of the use of the
\textit{Integer} type. On one hand, this could be prevented by checking the
operations beforehand for overflow errors. On the other hand, this could be the
expected behaviour when an \textit{Integer} is being used in \textit{Rebel}. As
\textit{Integers} are known have such limits that are also dependent on the
platform the application is run~\cite{wang2009intscope}. However, in
\textit{Rebel} there is currently no other type that can be used to hold a
bigger number. For example in \textit{Java} there is \textit{Long} for a larger
number, or \textit{BigDecimal} for even bigger numbers. This would mean that
\textit{Rebel} does not support big numbers, or that a custom type must be used
for this. Considering that \textit{Rebel} does not provide another type for
bigger numbers, the \textit{Integer} type is considered to also hold bigger
numbers. Since the specification is about banking products and it probably could
happen that a big number is needed. After all, we cannot know this for sure, as
\textit{Rebel} does not provide a specification yet of each of type in
\textit{Rebel}.

\subsection*{Precision errors}
\pinfo{Precision errors, Squants issue}
As we have seen, the \textit{Money} precision errors both occurred when using
\textit{Percentage} values as well as when using \textit{Integer} values to
operate with the \textit{Money} type. Since we were able to reproduce the issue
in a clean \textit{REPL} environment, the problem existed in the open-source
library, called \textit{Squants}, that was used for the \textit{Money} type. In
order to solve this problem, we created an issue on
\textit{Github}\footnote{https://github.com/typelevel/squants/issues/265}
related to the precision problems on the \textit{Money} type. A contributor
responded and fixed the issue within a day, the change will be included in the
next version of the library (1.4). So it is required to update this library in
order to let these tests in our test suite pass.

\todo{Add division}

% Rounding errors
% And division problem!

% Advice?
% To think of: Where there some things encountered during the project, which might require attention, but haven't or won't be thoroughly tested with this approach. Or do some additions cover these?
% > Squants
%	- Some bugs, maybe there are more in there
% 	- Snapshot release unavailable for some time? (1.4.0-SNAPSHOT was not reachable, had to compile it ourselves)
