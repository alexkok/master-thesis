% % % % % % % % % % % % % % % % % % % % % % % % % % % % % % % % % % % % %
% Abstract
\abstract{
  \textit{Rebel} is a domain specific language focused on the banking industry. Banking products can be specified in \textit{Rebel}. The tool chain for \textit{Rebel} can be used to check, simulate and visualize \textit{Rebel} specifications to reason about the specified banking product. The tool chain for \textit{Rebel} also provides some generators. These generators can generate a system based on \textit{Rebel} specifications. These generated systems provide an \textit{API} to work with the specified banking products.\\
  \\
  Although the generated system is based on the \textit{Rebel} specification(s), the generated code is not being checked against the \textit{Rebel} specification(s). This means that the generated system is perhaps working as expected. In this thesis we aim to improve this, by automatically testing the generated code using property-based testing and the \textit{Rebel} language itself.\\
  \\
  First we define the expected properties of \textit{Rebel}, using the existing axioms of algebra as inspiration. Then we write these properties down as ``events'' in a new \textit{Rebel} specification. We introduce our test framework, to automatically test this \textit{Rebel} specification on the generated code. This is done as follows: the test framework uses the existing generator in the \textit{Rebel} tool chain to generate the system for this \textit{Rebel} specification. It then generates a test suite containing test cases for each ``event'' in this \textit{Rebel} specification. As last, it runs the test suite against the generated system.\\
  \\
  Using this approach we identified some problems in the generated code that were unknown before. We have found precision errors, overflow/underflow errors and a compilation error. However, the number and the kind of bugs that can be found with this approach depend on the properties that were defined in the first place. We conclude that, by using this approach, the semantics of the generated code can be checked automatically on whether it satisfies the defined properties. The set of defined properties in this thesis is not complete for \textit{Rebel}. Therefore, it is not the case that all the generated code is tested, but only the generated code concerning the defined properties is being tested.
}
