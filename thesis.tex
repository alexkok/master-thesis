\documentclass{uvamscse}
\usepackage{placeins}

\usepackage{listings}
\usepackage{xcolor}

\lstdefinelanguage{sdf}{%
  numbers=none,
  morekeywords={module,imports,exports,sorts,context,lexical,free,syntax,==,=,+,-,left,cons,prefer,avoid,bracket},
  columns=flexible,
  morestring=[b]",
  basicstyle=\footnotesize\mdseries,
  literate={->}{{\,\,$\to$\,\,}}1
}

\definecolor{colorRascalKeyword}{RGB}{106,63,96}
\definecolor{colorRascalString}{RGB}{176,133,155}
%\definecolor{colorRascalString}{RGB}{2,112,10}
\definecolor{colorRascalComment}{RGB}{150,152,150}

\lstdefinelanguage{rascal}{
  morekeywords={syntax, keyword, lexical, break, continue, finally, private, fail, filter, if, tag, extend, append, non-assoc, assoc, test, anno, layout, data, join, it, bracket, in, import, all, solve, try, catch, notin, else, insert, switch, return, case, while, throws, visit, for, assert, default, map, alias, any, module, mod, public, one, throw, start, value, loc, node, num, type, bag, int, rat, rel, lrel, real, tuple, str, bool, void, datetime, set, map, list},
  morestring=[b]",
  basicstyle=\footnotesize\mdseries,
  keywordstyle=\color{colorRascalKeyword}\bfseries\rmfamily,
  stringstyle=\color{colorRascalString}\ttfamily,
  commentstyle=\color{colorRascalComment}\ttfamily,
  morecomment=[l][\color{colorRascalComment}]{\//}
}

% Same colors in Rebel (Rascal colors)
\lstdefinelanguage{rebel}{
  morekeywords={module, import, specification, fields, events, event, lifeCycle, initial, final, new, this, preconditions, postconditions},
  morestring=[b]",
  basicstyle=\footnotesize\mdseries,
  keywordstyle=\color{colorRascalKeyword}\bfseries\rmfamily,
  stringstyle=\color{colorRascalString}\ttfamily,
  commentstyle=\color{colorRascalComment}\ttfamily,
  morecomment=[l][\color{colorRascalComment}]{\//},
  morecomment=[l][\color{colorRascalComment}]{@}
}

% Source: https://tex.stackexchange.com/questions/47175/scala-support-in-listings-package
% Missing colors
\lstdefinelanguage{scala}{
  morekeywords={abstract,case,catch,class,def,%
    do,else,extends,false,final,finally,%
    for,if,implicit,import,match,mixin,%
    new,null,object,override,package,%
    private,protected,requires,return,sealed,%
    super,this,throw,trait,true,try,%
    type,val,var,while,with,yield},
  otherkeywords={=>,<-,<\%,<:,>:,\#,@},
  sensitive=true,
  morecomment=[l]{//},
  morecomment=[n]{/*}{*/},
  morestring=[b]",
  morestring=[b]',
  morestring=[b]"""
}

\lstdefinelanguage{log}{
  %morekeywords={},
  %morecomment=[l]{//},
  %morestring=[b]",
}

\lstset{%
  frame=none,
  xleftmargin=2pt,
  stepnumber=1,
  numbers=left,
  numbersep=7pt,
  numberstyle=\ttfamily\scriptsize\color[gray]{0.3},
  belowcaptionskip=\bigskipamount,
  captionpos=b,
  tabsize=2,
  emphstyle={\bf},
  stringstyle=\itshape,
  showspaces=false,
  keywordstyle=\bfseries\rmfamily,
  columns=flexible,
  basicstyle=\small\ttfamily,
  showstringspaces=false,
  breaklines=true, % sets automatic line breaking
}


% % % % % % % % % % % % % % % % % % % % % % % % % % % % % % % %
% New commands
\newcommand{\cmd}[1]{\texttt{$\backslash$#1}}
\definecolor{codegray}{gray}{0.9}
\newcommand{\code}[1]{\colorbox{codegray}{\texttt{#1}}}
%\newcommand{\code}[1]{\texttt{\leavevmode\unskip $`$#1$`$}}
%\PackageWarning{TODO:}{Remove todos on final!!!}
%\newcommand\todo[1]{{\marginpar{\textcolor{red}{$-$ TODO:\\#1}}}}
%\renewcommand\todo[1]{} % Turns off all TODO's
\newcommand\pinfo[1]{{\marginpar{\begin{flushleft}\textcolor{blue}{|\\#1}\end{flushleft}}}} %  Left align in \pinfo's with "flushleft"
\renewcommand\pinfo[1]{} % Turns off all \pinfo's

% % % % % % % % % % % % % % % % % % % % % % % % % % % % % % % %
% Customization
\usepackage{fancyhdr} % Fancy header package
\pagestyle{fancy} % Enable
\fancyhf{} % Clear
\fancyhead[L]{\leftmark} % Fancy header
\fancyfoot[C]{\thepage} % Leave page numbers on footer

\def\chapterautorefname{Chapter} % "Chapter X" instead of "SS X" when referring to chapters
\def\subsectionautorefname{Subsection} % "Subsection X" instead of "SS X" when referring to subsections
\renewcommand{\arraystretch}{1.3} % Vertical spacing in tables
\usepackage[labelfont=bf]{caption} % Make figure and table tags bold too

% % % % % % % % % % % % % % % % % % % % % % % % % % % % % % % %
% Front matters

\title{ATUR: Automated Testing Using Rebel}
% \coverpic[100pt]{figures/terminal.png}
%\subtitle{Automatically testing a generated system using its input language}
\subtitle{Testing a generator using its input language and output system}
% \date{Spring 2014}

\author{Alex Kok}
\authemail{alexkok08@gmail.com}
\supervisor{Jurgen J. Vinju}
\hostsupervisor{Jorryt-Jan Dijkstra}
\host{ING, \url{http://www.ing.nl}}

% % % % % % % % % % % % % % % % % % % % % % % % % % % % % % % %
% Content
% Research questions

% RQ Main \ How can we automatically test the implementation of the generated program?

  %\item Which of the existing approaches best fits our situation to determine the test cases? \info{(Which properties of Rebel should be tested and how are these defined?)}
 % RQ 1 \item How can we automatically generate the tests, such that it can test a part of the generated system?
  % - Properties
  % - Define in Rebel > Translate to Scala with generator > Scala tests > Run
 % RQ 2 \item What kind of errors can we find using this approach?
  % - Implementation of operations, such as equality, addition, multiplication, etc.
  % - Precision errors? (is basically the above?)
  % RQ 3 \item How can we extend our test suite to find more implementation errors?
  % - Add sync block
  % - Even more properties, but should be useful ones

% Old one
%\def \rqMain{ How can we automatically test the implementation of a component in the generated system against the Rebel specification, to check whether the implementation works as expected? }
% New
% Or something like:
% - Wat zegt het gegenereerde systeem over de kwaliteit van de generator mbt Rebel types?
% -
\def \rqMain{
	How can the generator in the \textit{Rebel} toolchain be tested automatically, by using the generated system, to check whether the implementation works as expected?
}
% Old one: Too vague, PBT wasn't mentioned, so 'automatically' was also unclear
% \def \rqOne{ How can we automatically generate the tests, such that it can test a component the generated system? }
% New
\def \rqOne{
	Which properties are expected to hold on the semantics of the generated code?
}
% Old one: zoom in, # of bugs and which type?
%\def \rqTwo{ What kind of errors can we find using this approach? }
% New
\def \rqTwo{
	How can we test each property as automatically as possible to find bugs in the generator?
}
% Old one: future work question...
%\def \rqThree{ How can we extend the test suite to find more implementation errors in the generated system? }
\def \rqThree{
	What kind of bugs can be found using this approach and how many?
}

% % % % % % % % % % % % % % % % % % % % % % % % % % % % % % % % % % % % %
% Abstract
\abstract{
  \textit{Rebel} is a domain specific language focused on the banking industry. Banking products can be specified in \textit{Rebel}. The tool chain for \textit{Rebel} can be used to check, simulate and visualize \textit{Rebel} specifications to reason about the specified banking product. The tool chain for \textit{Rebel} also provides some generators. These generators can generate a system based on \textit{Rebel} specifications. These generated systems provide an \textit{API} to work with the specified banking products.\\
  \\
  Although the generated system is based on the \textit{Rebel} specification(s), the generated code is not being checked against the \textit{Rebel} specification(s). This means that the generated system is perhaps working as expected. In this thesis we aim to improve this, by automatically testing the generated code using property-based testing and the \textit{Rebel} language itself.\\
  \\
  First we define the expected properties of \textit{Rebel}, using the existing axioms of algebra as inspiration. Then we write these properties down as ``events'' in a new \textit{Rebel} specification. We introduce our test framework, to automatically test this \textit{Rebel} specification on the generated code. This is done as follows: the test framework uses the existing generator in the \textit{Rebel} tool chain to generate the system for this \textit{Rebel} specification. It then generates a test suite containing test cases for each ``event'' in this \textit{Rebel} specification. As last, it runs the test suite against the generated system.\\
  \\
  With this approach we identified some problems in the generated code that were unknown before. We have found precision errors, overflow/underflow errors and a compilation error. However, the number and the kind of bugs that can be found with this approach depend on the properties that were defined in the first place. We conclude that, by using this approach, the semantics of the generated code can be checked automatically on whether it satisfies the defined properties. The set of defined properties in this thesis is not complete for \textit{Rebel}. Therefore, it is not the case that all the generated code is tested, but only the generated code concerning the defined properties is being tested.
}

% % % % % % % % % % % % % % % % % % % % % % % % % % % % % % % % % % % % %
% Chapter: Introduction
% % % % % % % % % % % % % % % % % % % % % % % % % % % % % % % % % % % % %
\chapter{Introduction}
\label{chp:intro}
\pinfo{Short about Rebel}
Large systems often suffer from domain knowledge that is implicit, incomplete, out of date or ambiguous definitions. This is what \textit{Rebel} aims to solve \cite{stoel2016solving}. The toolchain of \textit{Rebel} can be used to check, simulate and visualize the specifications, allowing to reason about the final product \cite{stoelcase}. Checking is done based on bounded model checking by using the Z3 solver.\\
\\
\pinfo{Aim of project}
Generators are being used to generates a system from the Rebel specifications. The generated system provides an API interface in order to work with the specified product and handles the database connectivity. However, the implementation of the generated program is not checked against the specifications, meaning that the generated program is perhaps not doing what it is supposed to do according to its specifications. The aim of this project is to improve this, by automatically testing the generated program against Rebel specifications.

% % % % % % % % % % % % % % % % % % % % % % % % % % % % % % % % % % % % %
% Section: Initial study
% // Don't think it's needed? Rebel papers?
% // It's in background too, about Rebel and what Rebel does.

% % % % % % % % % % % % % % % % % % % % % % % % % % % % % % % % % % % % %
% Section: Problem statement
\section{Problem statement}
\pinfo{Translation from generator and why automatically}
From the \textit{Rebel} specifications, a system can be generated by the generator. However, neither the generator, nor the generated program is being tested against the specification. Thus it could be that the generated system doesn't work according to what is specified in the \textit{Rebel} specification. Although the generator should translate everything correctly, we cannot assume that it actually does translate it correctly for each case and that the implementation works as expected.\\
\\
Currently, there are no tests for the generator or the generated system, during the development of the generator the results are being checked manually. Testing is a major cost factor in Software Development, with test automation being proposed as one of the solutions to reduce these costs \cite{ramler2006economic}. We aim for an approach such that much of the testing is automated to reduce the time (and costs) needed for testing certain components of the generated system.\\
\\
The main research question is as follows:
\begin{quote}
  \rqMain
\end{quote}
We investigate the following solution: generating tests based on a \textit{Rebel} specification and then run these tests against the generated system.\\
\\
% - Property based testing is a possible path, which we will do.
\pinfo{Possibility: PBT + short explanation}
Property-based testing is an approach to validate whether an implementation satisfies its specification~\cite{fink1997property}. It describes what the program should or should not do. As~\cite{fink1997property} describes: ``Property-based testing validates that the final product is free of specific flaws.''. With property-based testing, a property is being defined which should hold on the system. Next, the property is being tested for a certain amount of tries, using different input values to check whether the property holds. In case the property doesn't hold, it will result in a failure, reporting that there is a case in which the property doesn't hold. Indicating that a bug in the system has been triggered.\\
\\
\pinfo{Doing it too, worked already on x, x and x}
Property-based testing has already shown a success in earlier studies~\cite{fink1997property,claessen2011quickcheck,arts2006testing}, detecting errors in a system that were not known before. In this thesis we will use property-based testing to check the generator, using the generated system to check whether the properties hold.\\
\\
% - Hypothesis? We will find bugs in the generated system using this approach. Although errors could have been found (and fixed) already by manual testing/checking and code reviews,  we expect to detect yet unknown bugs in the generated system.
\pinfo{Hypothesis, we will detect}
We hypothesize that there are yet unknown bugs in the generator, resulting in that the generated system does not work as expected. By using property-based testing we expect to detect bugs in the generator.\\
\\
To answer the main research question, we will first answer the following research questions:
\begin{description}
\item[~~~~RQ 1:] \rqOne
\item[~~~~RQ 2:] \rqTwo
\item[~~~~RQ 3:] \rqThree
\end{description}

% - Shortly describe how this is done, and how it would find bugs, test mechanics explain it in more detail, image in test mechanics.
\pinfo{In short: how. Details are in CH3}
\todo{Is this needed here? Maybe remove or place somewhere else?}
The generator takes a \textit{Rebel} specification as input. Which contains the properties that are expected to hold in the generated system. Next, \textit{Scala} tests are being generated based on the properties, using the existing generator to translate the expressions used in the properties. These tests will be run against the generated system, to check whether each property holds. In case a test of a failing test, a bug has been found. There are multiple generators available within ING. Throughout this thesis, we will use the most mature generator, which is the Scala/Akka generator. This generator is often used for other experiments too.\\
\\
\pinfo{Assumptions}
In order to run the test suit, we assume that the generated system can be compiled and that it can be run. Furthermore, the specification which was used to generate the system should be syntactically and semantically correct. Which means that the \textit{Rebel} type checker should not report any error about the specification.\\
\\
\pinfo{Not detecting everything, but checking properties}
The test framework generates a test suite that can be run against the generated system. In case the test suite finishes without errors, it means that it did not found any bugs and that the generator satisfies the properties that were tested. This doesn't mean that there are no bugs in the generated system, instead, it means that our test suite was not able to find errors in the properties that it checks for. The generated system will probably still contain bugs which are not detected by using the test framework. In this case, improving the test framework might extend the number of bugs that it can find.

% % % % % % % % % % % % % % % % % % % % % % % % % % % % % % % % % % % % %
% Section: Research method
\section{Research method}
% In short, describe steps:
% // (Steps also interesting for mechanics chapter)
% 1 Defining properties, define what is unknown yet and what it actually means.
% 2 Describe how tests can be generated out of this. From these properties to Scala tests. (Properties > Rebel events > Scala tests per event, with random values)
% 3 Describe how we measure our experiments
% 4 Evaluate the results. Can we reason about it? What have we covered and also what have we not covered?
\pinfo{First defining properties, then small example}
We will start off with defining the properties that are expected to hold on the generator. Then we describe how these properties can be tested on the generator, using one property to demonstrate the working of the test framework. We can then run the tests suit against the generated system and check if this method actually works to detect bugs.\\
\\
\pinfo{Generate, run, evaluate results and improve again}
Next, we generate tests for each of these properties and run these against the generated system. After running the test suit, the result is being evaluated. When one or more tests are failing, a bug is found. However, we need to investigate the failing case such that we discover what the actual bug is. After evaluating we improve our test generation and continue to evaluate the results again.

% % % % % % % % % % % % % % % % % % % % % % % % % % % % % % % % % % % % %
% Section: Contribution
\section{Contribution}
% - Definitions (properties)
% Before, in conclusion:
% We provided a definition of some types in \textit{Rebel} and described properties that should hold when using operations with these types. Since there were no clear definitions available yet of what these types exactly do, a contribution has been made to the \textit{Rebel} project, such that there are now some definitions and properties available. This also helps when reasoning about the implementation, as there is now a definition of what is expected. Although these definitions might change over time, it allows to reason back to an earlier starting point.\\
We provide definitions of the properties that are expected to hold when using the \textit{Rebel} language and its toolchain. Allowing to reason about the generator and whether the implementation satisfies these properties. There are no property definitions of \textit{Rebel} yet, so we provide a starting point for this with the focus on the important properties.\\
\\
% - A way to automatically test the generator and generated system using Rebel, defining properties as events
% Before, in conclusion:
% We have shown a way to automatically test the generator using the \textit{Rebel} language and its toolchain, with the addition of our test framework. This is done by defining the properties that are expected to hold in the first case. Which can be translated to a \textit{Rebel} specification such that a system can be generated from it. Additionally using the generator when generating the test cases, so that the translation of expressions in the generator will be checked too. The different kind of properties should be able to detect unexpected behaviour.\\
The defined properties are being used to check the generator. We come with a solution that uses the input that is required for the generator and the output of the generator to check the generator. This process should be as automatic as possible, such that it doesn't require much time to make use of it. Our test framework will combine the required steps to detect bugs as automatically as possible.\\
\\
% - Found number of bugs in generated system
% Before, in conclusion:
%A number of bugs have been found in the generated system, precision errors, overflow/underflow errors and even a compile error. These bugs are now known issues in this project, while these were unknown before. This shows us that this approach already worked in such a way that it is able to detect bugs in the generated system. It can be extended to detect even more bugs. With some modifications, it is also possible to use this approach with other generated systems, such that it can detect inequalities among these.\\
The generator that is being used is expected to contain bugs. Therefore, we expect to detect, yet unknown, bugs in the generator. These bugs are then known bugs and can be fixed. The bugs found will then also indicate what kind of bugs we can detect in a generator by using this approach.

% - Squants issues
% Before, in conclusion:
%In this approach, we identified two bugs, which were due to an open-source library used in the generated system, called \textit{Squants}. We reported these two issues, such that these can be fixed in a later version of the library.

% % % % % % % % % % % % % % % % % % % % % % % % % % % % % % % % % % % % %
% Section: Related work
\section{Related work}
% // Doesn't this have to be at the end?
% - Testing of (extended) finite state machines

% % % % % % % % % % % % % % % % % % % % % % % % % % % % % % % % % % % % %
% Subsection: Random testing
\subsection{Random testing}
\pinfo{Describe random testing (FDRT)}
Random testing is a technique in which random values are being used as input for the test cases. \textit{QuickCheck} \cite{claessen2011quickcheck} and \textit{Randoop} \cite{pacheco2007randoop} are examples of random testing techniques. These differ in how they automatically test systems and what actually is being tested. QuickCheck is based on property-based testing, which we also apply in this project. \textit{Randoop} on the other hand is based on feedback-directed random testing.\\
\\
\pinfo{Describe FDRT}
With feedback-directed random testing, random tests are generated which will immediately be run. The result of earlier test attempts can affect the next test that is being generated, which can be seen as feedback for the next generated test. This allows each test case to 'learn' from earlier attempts and to create unique tests.\\
\\
\pinfo{About Randoop, shortly how it works}
\textit{Randoop} is build for Java projects and checks some built-in specifications of Java that can't be checked by the compiler. The test cases are simple unit tests, consisting of a unique sequence of methods due to the feedback of earlier attempts. The method sequences are unique because it also checks whether the same case has already been checked. Since there can be unlimited sequences of methods to test, the test suite will be terminated after a defined timeout. Next, the result is determined and failing cases are being reported, although when using this approach it cannot determine whether the whole system is correct according to the Java specifications. Instead, it just wasn't able to find a case for which it fails. This is a useful approach to generate unique tests, but its goal is to check systems built in Java and thus is not compatible with the semantics of Rebel. It works with calling the Java objects and using methods on those, while the generated system consists of mainly states and events. When using an approach like \textit{Randoop} with random input values, it cannot be known whether the result of a specific transition was expected to succeed or fail.\\
% Can add DART here, if needed
\\
\pinfo{Our case, why existing approaches can't be used}
%  Background contains info about quickcheck and randoop. Repeat limitations and describe how we do it
Approaches like \textit{QuickCheck} \cite{claessen2011quickcheck} and \textit{Randoop} \cite{pacheco2007randoop} enforce the system under test to be written in a specific language (\textit{Haskell} for \textit{QuickCheck}, \textit{Java} for \textit{Randoop}). For \textit{QuickCheck} there are alternative solutions for other languages. In our case, we need to use \textit{Rebel} when generating test cases, such that we test the generator when generating the tests. Another reason why we can't use a method like \textit{Randoop} is that \textit{Randoop} strictly checks for \textit{Java} properties, which are not in line with the \textit{Rebel} language.

% % % % % % % % % % % % % % % % % % % % % % % % % % % % % % % % % % % % %
% Section: Outline
\section{Outline}
\todo[inline]{Write later, when outline is more final.}
% 1. Describing background
% 2. Defining properties
% 3. Test mechanics
% 4. Experiment 1
% 5. Experiment 2
% 6. Conclusion
% 7. Threats of validity
% 8. Future work?

% % % % % % % % % % % % % % % % % % % % % % % % % % % % % % % % % % % % %
% Chapter: Background and context
% % % % % % % % % % % % % % % % % % % % % % % % % % % % % % % % % % % % %
\chapter{Background and context}
\label{chp:background_and_context}
...
% % % % % % % % % % % % % % % % % % % % % % % % % % % % % % % % % % % % %
% Section: Rascal
%\section{Rascal}
% // Not explaining Rascal for now
% - Rascal is a Meta-progrmaming language which is used throughout the project.
% - Rebel is written in Rascal.\\
% - The generators are also written in Rascal.\\
% - Our test generator is also written in Rascal. However, the resulting tests are written in Scala.

% % % % % % % % % % % % % % % % % % % % % % % % % % % % % % % % % % % % %
% Section: Rebel
\section{Rebel}
\pinfo{About Rebel}
Rebel is a domain specific language that focuses on the banking industry \cite{stoel2016solving}. Banking products can be specified in the language, with the use of types like \textit{Money} and \textit{Iban}. The tool chain of Rebel allows to check, visualize and simulate the specified banking product. For checking and simulation an efficient, state-of-the-art SMT solver is being used called \textit{Z3}, which is developed by Microsoft \cite{de2008z3}. \textit{Rebel} is written in Rascal and is developed by the \textit{ING} in corporation with the CWI. Currently, the tool chain of Rebel is also written in \textit{Rascal}.\\
\\
\pinfo{What Rebel exactly checks}
Checking a Rebel specification is based on Bounded Model Checking \cite{stoel2016solving}. The \textit{Rebel} specification is being translated to SMT constraints, next, the \textit{Z3} solver is being used to check whether the specification is inconsistent. An inconsistent specification means that a counter example has been found (a trace is found for which an invariant doesn't hold). It is bounded since it only checks if a counter example can be found within a certain amount of steps. Besides checking the specification, the specification can also be simulated. For simulation, the SMT solver is also being used to determine whether a transition can happen. After successfully checking the specification, meaning that no counter examples could be found, the result is still that the specification 'might' be valid. As the checking method is bounded, it stops at a certain point (bounded, for example, maximum depth of traces used for checking). This means that there can be a long or untested trace such that a counter example can exist.\\
\\
\pinfo{Generating system from specification}
From the Rebel specification, a system can be generated by using a generator which is developed by \textit{ING}. The generators are also written in \textit{Rascal}. This requires a specification that is consistent and that does not trigger errors by the type checker. Although the specification is being checked by using bounded model checking, the generated system is not being checked against the specification. Unfortunately, it is not possible to also use the bounded model checker to test the generator or the the generated system. As the generated system is written in Scala, while the checker only supports checking a \textit{Rebel} specification.

% % % % % % % % % % % % % % % % % % % % % % % % % % % % % % % % % % % % %
% Section: The Scala/Akka generator
\section{The Scala/Akka generator}
% Generator -> Scala, Akka, test config missing in prototype, thus we add that ourselves. Using Doubles for Percentage. Might expect problems on here. Known problems with Doubles etc.
\pinfo{Different generators, we use Scala/Akka generator}
There are multiple generators developed within \textit{ING}. The generators are different in that the resulting product is written in a different language or uses a different implementation, like database or messaging layer. Each generator is written in Rascal. Rebel defines the states and the transitions between the states in the \textit{lifecycle} block, which can be seen as the Finite State Machine definition of the specified product. The Scala/Akka generator is one of these generators. It generates a system that is written in \textit{Scala}, uses \textit{Akka}~\cite{siteAkka2017} as actor system and uses Cassandra~\cite{siteCassandra2016} as database. A resulting system of this generator is also tested thoroughly with performance tests, to reason about how well this system performs with its architecture. However, this does not check the implementation of the generated system against the \textit{Rebel} specification. This generator is often used within experiments inside ING and is considered the most mature generator among the currently developed generators. Because of this, it's interesting if we encounter yet unknown problems in the generator itself or the resulting system. Throughout this project, we will only make use of this generator.\\
\\
\pinfo{Short explanation on implementation generated system + why}
A specification can be defined in terms of a Labeled Transition System~\cite{stoel2016solving}%\todo{SRC-Rebel-Reference on smt check chapter. Do we need that source? ("As introduced by Keller 1876") Just citing Rebel is probably OK too}
, containing the states, the data fields and the transitions between the states along with the pre- and post conditions of the transitions. The generated system is based on these states and transitions defined in \textit{Rebel}, resulting in a system that works like a Labeled Transition System. Thus the generated program also implements it like states and transitions between them. An instance of a banking product can have fields and is in a specific state. In order to go to another state, a transition can be done which might have pre- or postconditions. In case there are pre- or postconditions, these have to be satisfied in order to successfully complete a transition.\\
\\
\pinfo{Adding test configuration, but not relevant for the project}
Although this generator is the most mature and often used in experiments within ING, it is still used as a prototype. The resulting system is thus not production ready, as this requires some more actions. One of these is that the resulting system should have tests which test the generated system. The generated system doesn't contain anything that's related to testing yet. So to make use of the testing libraries in Scala, we will need to add the test dependencies to the build file of the project and add a configuration file for Akka. This is done when we initialize the test suit and can be found in the source. However, we will not cover these settings in detail, as these are not relevant for this project.\\
\\
\pinfo{Diff types: known/expected problems. We have to specify expected properties}
\textit{Rebel} introduces custom types, such as \textit{Iban}, \textit{Percentage} and \textit{Money}, these types are not supported natively in Scala. For these cases, a library or own implementation is used. An example is the \textit{Money} type, which is available in the Squants \cite{siteSquants2017} library. The generated system uses this library to deal with the Money type and its operations. Another example is the \textit{Percentage} type, which is simply translated by calling a method \code{percentage()}. However, the return type of that method is a \textit{Double}, which is is a type using Floating-Point Arithmetic, which is known to have precision loss errors \cite{goldberg1991every}. In \textit{Rebel} the \textit{Percentage} type is actually defined as a whole number, so precision errors are probably not expected when using Rebel. In \textit{Rascal}, there are \textit{real} values, which basically always should contain the value as we expect it, without rounding errors. Since \textit{Rebel} is written in \textit{Rascal}, we could reason that the expected behaviour is to what \textit{Rascal} does, which is providing real values without precision loss\footnote{The \textit{real} type in \textit{Rascal} uses a precision of 60 decimals when expressions cannot be expressed, for example, 1/3.}. In order to conclude that the generated program is doing something incorrectly, we have to specify what properties are expected in \textit{Rebel} during this project.


% % % % % % % % % % % % % % % % % % % % % % % % % % % % % % % % % % % % %
% Section: Property based testing
\section{Property based testing}
\pinfo{About property based testing}
With property based testing properties are defined and being tested. It uses random values as input and checks whether the defined property holds. After a certain amount of succeeding cases the test succeeds and the next property is being checked.\\
\\
\pinfo{About QuickCheck}
A well-known tool which is based on property based testing is
\textit{QuickCheck}, which is written for Haskell \cite{claessen2011quickcheck}.
It tests the properties automatically by using random input values. For each property \textit{QuickCheck} tries to find counter examples, which are a set of values for which the desired property does not hold. If 100 test cases are succeeding in a row, it goes on to the next property. In case it found a counter example, it will try to minimize the values to try to report the edge case of the failure. However, one might have properties that only hold under certain conditions. For this \textit{QuickCheck} allows using preconditions. Although, this doesn't work well for every case as QuickCheck will just generate new pairs of values in case the precondition didn't hold for the generated set of values. An example of this is where 2 of the input values have to be equal, the chance that this happens with random values is rather low. \textit{QuickCheck} will try to generate new values each time, with a maximum of 1000 tries by default. In case this maximum is reached, it reports the case as "untested" and continues to the next property. Note that these values of 100 and 1000 are the default values, these can be adjusted when needed.\\
\\
\pinfo{About the ports of it}
% Different ports, for example FortressCheck
Due to the effectiveness of \textit{QuickCheck}, many ports for other languages were written. Most ports implement the basics of \textit{QuickCheck}, additionally, each port could have added extra features. Examples of some ports are FortressCheck (for Fortran) \cite{kang2011fortresscheck} and ScalaCheck (for Scala) \cite{siteScalaCheck2015}. FortressCheck supports polymorphic types and, unlike QuickCheck, heavily uses reflection for its value generation to solve certain problems with polymorphic constructs. Although Scala supports polymorphism, ScalaCheck does not use reflection to test this \cite{kang2011fortresscheck}.\\
\\
\pinfo{No QuickCheck for Rebel, not using ScalaCheck}
There is no \textit{QuickCheck} implementation for \textit{Rebel}. As the generated system is written in \textit{Scala}, ScalaCheck might be applicable to us. However, this would result in using \textit{ScalaCheck} as a black box, implementing the random functionality ourselves makes sure we know what is going on. Thus resulting in a white box implementation for our test framework.\\
\\
\pinfo{Custom modifications}
We can also modify it to our needs when we want to improve our test suite. One of the things that we might want to improve is to generate values under certain conditions. For example, if a property only holds under a certain condition, the chance that random values satisfy the condition can be very low. Resulting in a test case that wouldn't do anything most of the time. An example of such a property is \textit{Transitivity} (\code{x == y $\implies$ y == x}).%, which defines that y should be equal to x iff x is equal to y.
Additionally, we might also need to slightly interact with other components, such as the messaging layer, that the generated system uses. This could make the implementation more complicated when using \textit{ScalaCheck}.

% % % % % % % % % % % % % % % % % % % % % % % % % % % % % % % % % % % % %
% Section: Terminology
\section{Terminology}
\pinfo{Confusing terminologies and abstractions}
In this thesis there are some levels of abstraction, the terminology used throughout this theses can cause confusion. Words like "specification", "properties" and "tests" generally can have a different meaning depending on the context. In this section we describe the confusing terminologies and abstractions in detail.

\begin{description}
\item[Specification]\hfill\\
In this thesis we use the word “specification” exclusively to indicate banking products described in the \textit{Rebel} language. This includes the state machines (life cycle), pre and post-conditions and logical invariants.

\item[Properties]\hfill\\
We use the word “property” to describe semantic properties of the \textit{Rebel} language. The set of properties we introduce in this thesis can be seen as a partial “specification” of the semantics of Rebel, but we do not use this word to avoid confusion. We stick with “properties”.

\item[Implicative property]\hfill\\
A "property" that uses the implication ($\implies$) operator in it's definition.

\item[Generator(s)]\hfill\\
The generator(s) that can be used to generate a system based on a "specification". When using the term "the generator", we refer to the Scala/Akka generator which we use throughout this thesis.

\item[Test framework]\hfill\\
The test framework that was developed during this thesis. Which builds the specification, generates a system from the specification by using the generator, generates the test suite and runs the test suit against the generated system.

\item[System Under Test (SUT)]\hfill\\
The system against which the tests are being run. This system is generated by using a generator. Also referred to as "the generated system" in the context of generating the "system under test".

\item[Tests]\hfill\\
A generated test by the "test framework", intended to check whether a certain "property" holds.

\item[Test suite]\hfill\\
The collection of generated test cases, along with its configurations which can be run to test the generated system. Note that the test suite initially doesn't exist. Instead, it is being generated and added to "the generated system" when we run the "test framework".

\item[Events]\hfill\\
The event definitions in a "specification", these can be seen as transitions in a Labeled Transition System.

\item[Scala Build Tool (SBT)]\hfill\\
The tool that is used to compile and run "the generated system". Also used to run the "generated test suite".

%item[Invariant(s):] The invariant of a Rebel specification defines conditions that should always hold during the lifecycle of an instance of the product.

\end{description}

% % % % % % % % % % % % % % % % % % % % % % % % % % % % % % % % % % % % %
% Chapter: Property definitions
% % % % % % % % % % % % % % % % % % % % % % % % % % % % % % % % % % % % %
\chapter{Property definitions} 
\label{cpt:4_properties}
% Definition: An Axiom is a mathematical statement that is assumed to be true.
% ( http://www.aaaknow.com/lessonFull.php?slug=propsCommAssoc )
% Sources:
% Axioms in algebra — where did they come from?
% http://www.jstor.org.proxy.uba.uva.nl:2048/stable/pdf/27956337.pdf?refreqid=excelsior%3A623b1c85f79f8cd18bc9d09560339b9b
% (simple formulas, history, and explanations)
% A perspective on the algebra of logic 
% http://www-tandfonline-com.proxy.uba.uva.nl:2048/doi/pdf/10.2989/16073606.2011.622856?needAccess=true
% (More formulas, deeper explanations) 
% Might contain more properties

% Another source was (but not a paper or so)
% http://www.aaamath.com/ac22.htm

% Another good source (book)
% Calculus, Vol. 1: One-Variable Calculus, with an Introduction to Linear Algebra 2nd Edition

\pinfo{Types and axioms}
For types like integers the axioms can be used to determine whether the implementation in the generated system is correct. These are most likely translated to integers in the generated system too, with the expectation that these have the same properties in Scala. For these types the known axioms could be used. However, it depends on what \textit{Rebel} defines as \textit{Integer}.\\
\\
\pinfo{Custom types, no properties yet. Focusing on Rebel types initially}
\textit{Rebel} has built-in types that are specifically designed for it's domain, the banking domain. Types like \textit{Money}, \textit{Percentage} and \textit{Iban} do not have a specification of what properties these have exactly. In this chapter we will define properties for these types, which we can then use to generate the test suite. The properties that we define are based on the known axioms in algebra \cite{baumgart1961axioms,raftery2011perspective,apostol2007calculus}. We provide a short explanation of why a certain property should also hold in these cases. We will focus on the custom types that are introduced in \textit{Rebel}.
%\todo[inline]{Percentage? Not yet, defining those requires properties again. And running the test suit again, evaluate again, etc.\\
%Describe what exactly is Percentage if we define it in a specification, which rules does it inherit?\\
%Percentage (expected): 2/more decimals, however its a precise amount. Calculating with it is expected in a result of the real value.\\
%Percentage (current): As it is defined now: full number, it can exceed 100. However it does not support decimals yet.}

% % % % % % % % % % % % % % % % % % % % % % % % % % % % % % % % % % % % %
% Section: Money specification
\section{Money specification}
\pinfo{Money definition, operations. And rounding up to business}
The Money type is a built-in type in Rebel, but there is no exact definition of it yet. It consists of a currency and an amount value, which is similar to the pattern that Fowler describes for a \textit{Money} type \cite{fowler2002patterns}. The author also describes the operations that can be done with the \textit{Money} type, which are: +, -, *, allocate, $<$, $>$, $\leq$, $\geq$ and =. However, Rebel also allows the division (/) operation on the \textit{Money} type. \improve{Allocate is division, but with the problem covered. Explain that instead of Rebel supporting division as additional. Describing this also shows we have read the part in Fowlers book.} It is unsupported to use these operations with \textit{Money} values using different currencies. This is due to the exchange rates between currencies, which can vary and are not implemented yet. The amount of a \textit{Money} value is often rounded when it is being represented to the user, as it could have many decimals. The representation of the \textit{Money} value is up to the business on how this is done, as there are multiple factors influencing this. Instead, we only focus on the internal value that is used when operating with it.\\
\\
\pinfo{Amount, not floating-point arithmetic}
Since the amount of a \textit{Money} value contains multiple decimals, it can be seen as a floating number. Does this mean that it inherits the computation properties of Floating-Point Arithmetic, as defined in the IEEE standards 754 or 854? Since the Rebel is a formal specification language for banking products, we don't expect that the described problems with this arithmetic are intended to exist on the Money type. Considering that that a high volume flows within a bank in terms of Money, using the Arithmetic properties can result in the known precision, overflow and underflow errors as described in \cite{goldberg1991every}. The precision errors for instance, should be avoided when using the \textit{Money} type. The author of \cite{fowler2002patterns} also describes that the intend of the \textit{Money} type is to avoid this:
\begin{quote}
\todo{Use quotes instead of "" (latex-advice)}
	"You should absolutely avoid any kind of floating point type, as that will introduce the kind of rounding problems that Money is intended to avoid." 
    -- Martin Fowler \cite{fowler2002patterns} %[p. 462] % Of pdf, ebook thing doesn't have page numbers
\end{quote}
\pinfo{Conclusion: Precise value, rounding only for division}
We say that the amount of a \textit{Money} value in Rebel should hold the exact value as if we would calculate the same expression by ourselves. Meaning that the precision errors would have to be prevented. An exception on this is when the result would be in the form of a fraction, for example 1/3. As this results in a value which we have to round up sometime. To fix this, we say that in this case we use x decimals when calculating, without rounding the x+1th decimal. When using properties with division, this should be taken into account in case the system fails on these tests. \todo{Define X, is this also 60 decimals?}\\
\\
\pinfo{Properties based on known axioms, but has restrictions}
The properties that we define on the \textit{Money} type of Rebel are based on the axioms of integers. It isn't possible to multiply two \textit{Money} types with each other to support the multiplicative property for example. Instead we can only multiply \textit{Money} with other types, such as \textit{Integer} and \textit{Percentage}, resulting in multiple property definitions when using different types. In the following sections we motivate the properties that we use during the project.
% % % % % % % % % % % % %
% Subsection: Reflexivity
\subsection{Reflexivity}
% Table
\begin{table}[h!]
\centering
\begin{tabular}{|lll|}
\hline
                        \textbf{Formula} & \textbf{Property name} & \textbf{Variable (Type)} \\ \hline
\rowcolor[HTML]{EFEFEF} x == x           & reflexiveEquality      & x: Money                 \\
                        x $\leq$ x       & reflexiveInequalityLET & x: Money                 \\
\rowcolor[HTML]{EFEFEF} x $\geq$ x       & reflexiveInequalityGET & x: Money                 \\ \hline
\end{tabular}
\caption{Reflexivity on \textit{Money}}
\label{tbl:ch4_money_reflexivity}
\end{table}
% Reasoning
\pinfo{Equality, currency and amount}
The reflexive property means a relation of a type with itself \cite{raftery2011perspective}. An instance of type \textit{Money} should be equal to itself. Taking both the currency and the amount into account. The inequality relations \textit{smaller or equal to} and \textit{greater or equal to} should hold too. As we can compare \textit{Money} variables and defined equality in the first row of the table.

% % % % % % % % % % % % %
% Subsection: Symmetry
\subsection{Symmetry}
% Table
\begin{table}[h!]
\centering
\begin{tabular}{|lll|}
\hline
                        \textbf{Formula}         & \textbf{Property name} & \textbf{Variable (Type)} \\ \hline
\rowcolor[HTML]{EFEFEF} x == y $\implies$ y == x & symmetric              & x: Money                 \\
\rowcolor[HTML]{EFEFEF}                          &                        & y: Money                 \\ \hline
\end{tabular}
\caption{Symmetry on \textit{Money}}
\label{tbl:ch4_money_symmetry}
\end{table}
% Reasoning
\pinfo{Equality, currency and amount}
Reflexivity already described equality on \textit{Money} when used on the same variable. When two different variables are used, the order should not matter and thus it should work in both ways. Which is known as the symmetric property \cite{raftery2011perspective}.

% % % % % % % % % % % % %
% Subsection: Antisymmetry
\subsection{Antisymmetry}
% Table
\begin{table}[h!]
\centering
\begin{tabular}{|lll|}
\hline
                        \textbf{Formula}                             & \textbf{Property name} & \textbf{Variable (Type)} \\ \hline
\rowcolor[HTML]{EFEFEF} x $\leq$ y \&\& y $\leq$ x $\implies$ x == y & antisymmetryLET        & x: Money                 \\
\rowcolor[HTML]{EFEFEF}                                              &                        & y: Money                 \\ 
                        x $\geq$ y \&\& y $\geq$ x $\implies$ x == y & antisymmetryGET        & x: Money                 \\
                                                                     &                        & y: Money                 \\ \hline
\end{tabular}
\caption{Antisymmetry on \textit{Money}}
\label{tbl:ch4_money_antisymmetry}
\end{table}
% Reasoning
The antisymmetric relation describes that whenever there is a relation from x to y and a relation from y to x, then x and y should be equal. The lower or equal then and greater or equal then relations fit in this category, as shown in \autoref{tbl:ch4_money_antisymmetry}. We can use these operations on the \textit{Money} type when both x and y use the same currency. This antisymmetric relation is also expected to hold, as \textit{Money} values should be equal when they are of the same currency and hold the same amount.

% % % % % % % % % % % % %
% Subsection: Transitivity
\subsection{Transitivity}
% Table
\begin{table}[h!]
\centering
\begin{tabular}{|lll|}
\hline
                         \textbf{Formula}                                 & \textbf{Property name}  & \textbf{Variable (Type)} \\ \hline
\rowcolor[HTML]{EFEFEF}  x == y \&\& y == z $\implies$ x == z             & transitiveEquality      & x: Money                 \\
\rowcolor[HTML]{EFEFEF}                                                   &                         & y: Money                 \\
\rowcolor[HTML]{EFEFEF}                                                   &                         & z: Money                 \\
                         x $<$ y \&\& y $<$ z $\implies$ x $<$ z          & transitiveInequalityLT  & x: Money                 \\
                                                                          &                         & y: Money                 \\
                                                                          &                         & z: Money                 \\
\rowcolor[HTML]{EFEFEF}  x $>$ y \&\& y $>$ z $\implies$ x $>$ z          & transitiveInequalityGT  & x: Money                 \\
\rowcolor[HTML]{EFEFEF}                                                   &                         & y: Money                 \\
\rowcolor[HTML]{EFEFEF}                                                   &                         & z: Money                 \\ 
                         x $\leq$ y \&\& y $\leq$ z $\implies$ x $\leq$ z & transitiveInequalityLET & x: Money                 \\
                                                                          &                         & y: Money                 \\
                                                                          &                         & z: Money                 \\
\rowcolor[HTML]{EFEFEF}  x $\geq$ y \&\& y $\geq$ z $\implies$ x $\geq$ z & transitiveInequalityGET & x: Money                 \\
\rowcolor[HTML]{EFEFEF}                                                   &                         & y: Money                 \\
\rowcolor[HTML]{EFEFEF}                                                   &                         & z: Money                 \\ \hline
\end{tabular}
\caption{Transitivity on \textit{Money}}
\label{tbl:ch4_money_transitivity}
\end{table}
% Reasoning
Operations can be done on the \textit{Money} types. The transitive properties \cite{raftery2011perspective} on the (in)equality operators should still hold on the \textit{Money} type as we can still compare the \textit{Money} values. It is important to note that either the currency of the values should be the same, or the conversion rate should be taken into account with these operations. 

% % % % % % % % % % % % %
% Subsection: Commutativity
\subsection{Commutativity}
\label{ssct:4_commutativity}
% Table
\begin{table}[h!]
\centering
\begin{tabular}{|lll|}
\hline
                        \textbf{Formula} & \textbf{Property name}               & \textbf{Variable (Type)} \\ \hline
\rowcolor[HTML]{EFEFEF} x + y == y + x   & commutativeAddition                  & x: Money                 \\
\rowcolor[HTML]{EFEFEF}                  &                                      & y: Money                 \\
                        x * y == y * x   & commutativeMultiplicationInteger1    & x: Integer               \\
                                         &                                      & y: Money                 \\
\rowcolor[HTML]{EFEFEF} x * y == y * x   & commutativeMultiplicationInteger2    & x: Money                 \\
\rowcolor[HTML]{EFEFEF}                  &                                      & y: Integer               \\
                        x * y == y * x   & commutativeMultiplicationPercentage1 & x: Percentage            \\
                                         &                                      & y: Money                 \\
\rowcolor[HTML]{EFEFEF} x * y == y * x   & commutativeMultiplicationpercentage2 & x: Money                 \\
\rowcolor[HTML]{EFEFEF}                  &                                      & y: Percentage            \\ \hline
\end{tabular}
\caption{Commutativity on \textit{Money}}
\label{tbl:ch4_money_commutativity}
\end{table}
% Reasoning
These properties are based on the commutative law \cite{baumgart1961axioms}. The result of an addition or multiplication does not vary when swapping the input variables. Because of the \textit{Money} type, we can only do addition on \textit{Money} values with other \textit{Money} values. For multiplication, there is no known value for multiplying two \textit{Money} variables. It is possible to multiply it by an \textit{Integer} or \textit{Percentage}. Also in this case, the order shouldn't matter if we would put the \textit{Money} value as first input parameter to multiplication or the other way around.

% % % % % % % % % % % % %
% Subsection: Anticommutativity
\subsection{Anticommutativity}
% Table
\begin{table}[h!]
\centering
\begin{tabular}{|lll|}
\hline
                        \textbf{Formula}  & \textbf{Property name} & \textbf{Variable (Type)} \\ \hline
\rowcolor[HTML]{EFEFEF} x - y == -(y - x) & anticommutativity      & x: Money                 \\
\rowcolor[HTML]{EFEFEF}                   &                        & y: Money                 \\ \hline
\end{tabular}
\caption{Anticommutativity on \textit{Money}}
\label{tbl:ch4_money_anticommutativity}
\end{table}
% Reasoning
In \autoref{ssct:4_commutativity} we described the commutative properties. Note that the operations only use addition and multiplication on this property. Subtraction is a operation that is anticommutative as swapping the order of the two argumates is negating the result. The anticommutative property thus negates the result of swapping the two arguments, intending to result in the actual value again, as shown in \autoref{tbl:ch4_money_anticommutativity}.

% % % % % % % % % % % % %
% Subsection: Associativity
\subsection{Associativity}
\label{ssct:4_associativity}
% Table
\begin{table}[h!]
\centering
\begin{tabular}{|lll|}
\hline
                        \textbf{Formula}           & \textbf{Property name}               & \textbf{Variable (Type)} \\ \hline
\rowcolor[HTML]{EFEFEF} (x + y) + z == x + (y + z) & associativeAddition                  & x: Money                 \\
\rowcolor[HTML]{EFEFEF}                            &                                      & y: Money                 \\
\rowcolor[HTML]{EFEFEF}                            &                                      & z: Money                 \\
                        (x * y) * z == x * (y * z) & associativeMultiplicationInteger1    & x: Integer               \\
                                                   &                                      & y: Integer               \\
                                                   &                                      & z: Money                 \\
\rowcolor[HTML]{EFEFEF} (x * y) * z == x * (y * z) & associativeMultiplicationInteger2    & x: Money                 \\
\rowcolor[HTML]{EFEFEF}                            &                                      & y: Integer               \\
\rowcolor[HTML]{EFEFEF}                            &                                      & z: Integer               \\
                        (x * y) * z == x * (y * z) & associativeMultiplicationPercentage1 & x: Money                 \\
                                                   &                                      & y: Percentage            \\
                                                   &                                      & z: Integer               \\
\rowcolor[HTML]{EFEFEF} (x * y) * z == x * (y * z) & associativeMultiplicationpercentage2 & x: Integer               \\
\rowcolor[HTML]{EFEFEF}                            &                                      & y: Money                 \\
\rowcolor[HTML]{EFEFEF}                            &                                      & z: Percentage            \\ \hline
\end{tabular}
\caption{Associativity on \textit{Money}}
\label{tbl:ch4_money_associativity}
\end{table}
% Reasoning
The law of associativity is known on addition and multiplication \cite{baumgart1961axioms}. It defines that the order in which certain operations are done, does not affect the result of the whole expression. As described in \autoref{ssct:4_commutativity} it is not possible to operate with multiplication with only \textit{Money} types. However, in Rebel the same properties should hold when using different types, as shown in \autoref{tbl:ch4_money_associativity}.

% % % % % % % % % % % % %
% Subsection: Non-associativity
\subsection{Non-associativity}
% Table
\begin{table}[h!]
\centering
\begin{tabular}{|lll|}
\hline
                        \textbf{Formula}           & \textbf{Property name} & \textbf{Variable (Type)} \\ \hline
\rowcolor[HTML]{EFEFEF} (x - y) - z != x - (y - z) & nonassociativity       & x: Money                 \\
\rowcolor[HTML]{EFEFEF}                            &                        & y: Money                 \\ \hline
\end{tabular}
\caption{Non-associativity on \textit{Money}}
\label{tbl:ch4_money_nonassociativity}
\end{table}
% Reasoning
% in tegenstelling tot steen is geen steen stoep
In contrast to associativity (\autoref{ssct:4_associativity}), non-associativity described that the order does affect the result of the whole expression. As we can see in \autoref{tbl:ch4_money_nonassociativity} subtraction is a relation where this property holds. An exception to this would be when each argument is zero.

% % % % % % % % % % % % %
% Subsection: Distributivity
\subsection{Distributivity}
% Table
\begin{table}[h!]
\centering
\begin{tabular}{|lll|}
\hline
                        \textbf{Formula}             & \textbf{Property name}  & \textbf{Variable (Type)} \\ \hline
\rowcolor[HTML]{EFEFEF} x * (y + z) == x * y + x * z & distributiveInteger1    & x: Money                 \\
\rowcolor[HTML]{EFEFEF}                              &                         & y: Integer               \\
\rowcolor[HTML]{EFEFEF}                              &                         & z: Integer               \\
                        (y + z) * x == y * x + z * x & distributiveInteger2    & x: Integer               \\
                                                     &                         & y: Money                 \\
                                                     &                         & z: Money                 \\
\rowcolor[HTML]{EFEFEF} x * (y + z) == x * y + x * z & distributivePercentage1 & x: Percentage            \\
\rowcolor[HTML]{EFEFEF}                              &                         & y: Money                 \\
\rowcolor[HTML]{EFEFEF}                              &                         & z: Money                 \\
                        (y + z) * x == y * x + z * x & distributivePercentage2 & x: Percentage            \\
                                                     &                         & y: Money                 \\
                                                     &                         & z: Money                 \\ \hline
\end{tabular}
\caption{Distributivity on \textit{Money}}
\label{tbl:ch4_money_distributivity}
\end{table}
% Reasoning
The law of distributivity is another well-known law \cite{baumgart1961axioms}. Unlike associativity, the order does matter here when using different operations. These operations can be used on \textit{Money} and since we can see \textit{Money} as a number, this property is also expected on this type. Note that it is not possible to multiply \textit{Money} types with each other, so the variable types are an important part in these properties as described in \autoref{tbl:ch4_money_distributivity}.

% % % % % % % % % % % % %
% Subsection: Identity
\subsection{Identity}
% Table
\begin{table}[h!]
\centering
\begin{tabular}{|lll|}
\hline
                        \textbf{Formula} & \textbf{Property name}  & \textbf{Variable (Type)} \\ \hline
\rowcolor[HTML]{EFEFEF} x + 0 == x       & additiveIdentity1       & x: Money                 \\
						0 + x == x       & additiveIdentity2       & x: Money                 \\
\rowcolor[HTML]{EFEFEF} x * 1 == x       & multiplicativeIdentity1 & x: Money                 \\
                        1 * x == x       & multiplicativeIdentity2 & x: Money                 \\ \hline
\end{tabular}
\caption{Identity on \textit{Money}}
\label{tbl:ch4_money_identity}
\end{table}
% Reasoning
The identity relation describes a function that returns the same value as the value that was given as input. For additive this entails the addition of zero to the input value and for multiplicative this entails multiplying the value by 1. Also the commutative property holds here, as the order does not matter in which this function is applied. Since it is not possible to just add 0 to a \textit{Money} value, the 0 showed in \autoref{tbl:ch4_money_identity} must be defined in a \textit{Money} format. Thus it must have the same currency as the parameter, with the amount of 0. For multiplication the \textit{Integer} type can be used.

% % % % % % % % % % % % %
% Subsection: Inverse
\subsection{Inverse}
% Table
\begin{table}[h!]
\centering
\begin{tabular}{|lll|}
\hline
                        \textbf{Formula} & \textbf{Property name} & \textbf{Variable (Type)} \\ \hline
\rowcolor[HTML]{EFEFEF} x + (-x) == 0    & additiveInverse1       & x: Money                 \\
                        (-x) + x == 0    & additiveInverse2       & x: Money                 \\ \hline
\end{tabular}
\caption{Inverse on \textit{Money}}
\label{tbl:ch4_money_inverse}
\end{table}
% Reasoning
The inverse relation describes for additivity that using addition with the input parameter and the negative of the input parameter, results in the value zero. Note that the operation is used on the \textit{Money} type, so the expected value is 0 with the same currency as the currency of the input parameter. Although the inverse relation could also be used with multiplication and division (defined as \code{x*(1/x) == 0}), it is not possible to use this definition in this project. As we cannot divide something with the \textit{Money} type, which is why we only define the inverse relation using addition on \textit{Money}.

% % % % % % % % % % % % %
% Subsection: Additivity
\subsection{Additivity}
% Table
\begin{table}[h!]
\centering
\begin{tabular}{|lll|}
\hline
                        \textbf{Formula}                             & \textbf{Property name} & \textbf{Variable (Type)} \\ \hline
\rowcolor[HTML]{EFEFEF} x == y $\implies$ x + z == y + z             & additive               & x: Money                 \\
\rowcolor[HTML]{EFEFEF}                                              &                        & y: Money                 \\
\rowcolor[HTML]{EFEFEF}                                              &                        & z: Money                 \\
                        x == y \&\& z == a $\implies$ x + z == y + a & additive4params        & x: Money                 \\
                                                                     &                        & y: Money                 \\
                                                                     &                        & z: Money                 \\ 
                                                                     &                        & a: Money                 \\ \hline
\end{tabular}
\caption{Additivity on \textit{Money}}
\label{tbl:ch4_money_additivity}
\end{table}
% Reasoning
Addition was earlier mentioned for the Commutativity (\autoref{ssct:4_commutativity}) and Associativity (\autoref{ssct:4_associativity}) properties. The properties mentioned here extend these by defining properties that are true when the input values are equal. When using the addition operator such that the resulting values on both sides remain the same, as shown in \autoref{tbl:ch4_money_symmetry}, it should not break the equality property on the resulting values.

% % % % % % % % % % % % %
% Subsection: Property of Zero
\subsection{Property of Zero}
% Table
\begin{table}[h!]
\centering
\begin{tabular}{|lll|}
\hline
                        \textbf{Formula} & \textbf{Property name}     & \textbf{Variable (Type)} \\ \hline
\rowcolor[HTML]{EFEFEF} x * 0 == 0       & multiplicativeZeroProperty1 & x: Money                 \\
                        0 * x == 0       & multiplicativeZeroProperty2 & x: Money                 \\ \hline
\end{tabular}
\caption{Property of Zero on \textit{Money}}
\label{tbl:ch4_money_propertyzero}
\end{table}
% Reasoning
The property of zero on multiplication states that if something is multiplied by zero, the result will always be zero. Since Rebel allows the use of multiplication on the \textit{Money} type, it's possible to multiply it by 0. Since the value of a \textit{Money} variable is based on a decimal number, this property states that the value will be exactly 0 (or 0.00 in the representation of a \textit{Money} value). But it should not contain any decimal number\unsure{Not sure if decimal is right here? - Fraction}. \todo{Note the order doesnt matter}

% % % % % % % % % % % % %
% Subsection: Division
\subsection{Division}
% Table
\begin{table}[h!]
\centering
\begin{tabular}{|lll|}
\hline
                        \textbf{Formula}                 & \textbf{Property name} & \textbf{Variable (Type)} \\ \hline
\rowcolor[HTML]{EFEFEF} x * y == z $\implies$ x == z / y & division1              & x: Money                 \\
\rowcolor[HTML]{EFEFEF}                                  &                        & y: Integer               \\ 
\rowcolor[HTML]{EFEFEF}                                  &                        & y: Money                 \\ 
                        x == z * y $\implies$ x / y == z & division2              & x: Money                 \\
                                                         &                        & y: Integer               \\
                                                         &                        & y: Money                 \\ \hline
\end{tabular}
\caption{Division on \textit{Money}}
\label{tbl:ch4_money_division}
\end{table}
% Reasoning
When using division with the \textit{Money} type, it is not possible to use a \textit{Money} value as denominator. However, a Money type can be dived by an Integer, thus we can define the division properties by using both the \textit{Money} and \textit{Integer} type. Note that the denominator cannot be zero, as division by zero is not possible. % Maybe theres a source for division by zero? :)

% % % % % % % % % % % % %
% Subsection: Trichotomy
\subsection{Trichotomy}
% Table
\begin{table}[h!]
\centering
\begin{tabular}{|lll|}
\hline
                        \textbf{Formula}             & \textbf{Property name} & \textbf{Variable (Type)} \\ \hline
\rowcolor[HTML]{EFEFEF} x $<$ y || x == y || x $>$ y & trichotomy             & x: Money                 \\
\rowcolor[HTML]{EFEFEF}                              &                        & y: Money                 \\ \hline
\end{tabular}
\caption{Trichotomy on \textit{Money}}
\label{tbl:ch4_money_trichotomy}
\end{table}
% Reasoning
The law of trichotomy defines that for every pair of arbitrary real numbers, exactly one of the relations $<$, ==, $>$ holds. We can define a property for this on money, as shown in \autoref{tbl:ch4_money_trichotomy}.



% Old content

%\change{Axiom is a specification of a property.}
%The properties that we test are based on the properties of Axioms in Algebra. We assume that any expression that Rebel allows in its syntax, should be a valid expression. \change{In syntax more is allowed. Syntax is first. Then type checker and then its valid. Maybe reasoning might adds more...} Rebel supports multiple types for variables, where some types can be used in conjunction with others. The property definitions can depend on the combination of types too.\change{Type system, just a term for it, Jargon. Also check other papers how these define it} For example when we have a value x of type `Money` and value y and z of type `Percentage`, then it is not the case that x*(y*z) equals (x*y)*z. But this property would hold if all the values x, y and z were of type `Integer`.
%\todo{Tabular representation with samples?}
% % % % % % % % % % % % % % % % % % % % % % % % % % % % % % % % % % % % %
% Section: Reflexitivity
%\section{Reflexitivity}
%A type is equal to itself

% % % % % % % % % % % % % % % % % % % % % % % % % % % % % % % % % % % % %
% Section: Associative addition and multiplication
%\section{Associative addition and multiplication}
%The expected result when operating with money types is that the total amount remains the same despite of the order. This is also a property in the Axioms of Algebra on Integers, which we can relate to the Money type. For example, if we first receive 50 euros, and then 40 euros, we have a total of 90 euros. But its also expected that we have 90 euros in case we first receive the 40 euros and then the 50 euros. Although this holds with the addition of money, it doesn't hold for subtraction. Furthermore we cannot multiply money with itself. We can however multiply between Money and an Integer, or with Money and Percentage. For these cases we also expect that the result of money multiplied by an int will be the same as multiplying the int with the money.\change{Why 'int' here? Kevin: read as Rebel int and Integer as scala int...}\\
%\\
% % % % % % % % % % % % % % % % % % % % % % % % % % % % % % % % % % % % %
% Section: Distributive multiplication
%\section{Distributive multiplication}
%etc...

% % % % % % % % % % % % % % % % % % % % % % % % % % % % % % % % % % % % %
% Chapter: Test mechanics
% % % % % % % % % % % % % % % % % % % % % % % % % % % % % % % % % % % % %
\chapter{Test mechanics}
\label{cpt:testmechanics}
\pinfo{Intro, describing contents of chapter}
In order to check whether these properties hold when using the generator, we
need to determine how the test framework should work. In this chapter we will
try to answer the following research question:
\begin{description}
  \item [RQ 2:] \rqTwo
\end{description}

We use property-based testing as testing technique. The aim of this project is
to test the implementation of the generator and trying to find yet unknown bugs
in it. Unfortunately we cannot test the properties right away on the generator,
but we aim to test the properties as automatic as possible. To check the
generator, we need to use the system that it generates. But in order to do this,
a valid \textit{Rebel} specification is required. In this chapter we describe
how the test framework is setup such that it can automatically check whether the
defined properties hold when using the generator.

% % % % % % % % % % % % % % % % % % % % % % % % % % % % % % % % % % % % %
% Section: The test framework
\section{The test framework}
\pinfo{Describing phases}
A \textit{Rebel} specification can be created with the property definitions.
Which can then be used to generate the test cases. The collection of resulting
test cases is the content of the test suite, which we can be run against the
generated system. We can divide the process into different phases. The goal of
the test framework is to combine most of the required phases such that each
defined property is being checked as automatic as possible. An overview of the
phases and the test framework is shown in \autoref{fig:testmechanics_overview}.
% Figure
\FloatBarrier
\begin{figure}[!ht]
%\frame{
	\includegraphics[width=\linewidth]{figures/testmechanics_overview}
%}
\caption{Overview of the test framework and the phases}
\label{fig:testmechanics_overview}
\centering
\end{figure}
\FloatBarrier
% End figure
The phases are defined as follows:
\def \tfPhaseOne{Create specification}
\def \tfPhaseTwo{Check \& build}
\def \tfPhaseThree{Generate system}
\def \tfPhaseFour{Generate test suite}
\def \tfPhaseFive{Run test suite}
\begin{enumerate}
  \item \tfPhaseOne{}
  \item \tfPhaseTwo{}
  \item \tfPhaseThree{}
  \item \tfPhaseFour{}
  \item \tfPhaseFive{}
\end{enumerate}
We will describe each phase in detail in the next sections. Additionally we will
define some evaluation criteria which will be used to evaluate the test
framework. The \textit{Reflexitivity} property will be used to demonstrate each
phase. More specifically: the case of \textit{Reflexivity} when using equality,
called \textit{ReflexiveEquality} throughout this project. The definition of
\textit{ReflexiveEquality} is shown in
\autoref{tbl:ch3_small_property_definition}.
% Table
\FloatBarrier
\begin{table}[!ht]
\centering
\begin{tabular}{ccc}
\hline
% \multicolumn{3}{c}{\textbf{ReflexiveEquality}} \\ \hline
\textbf{Formula} & \textbf{Variable} & \textbf{Type} \\ \hline
x == x & x & Money \\ \hline
\end{tabular}
\caption{Property definition of \textit{ReflexiveEquality}}
\label{tbl:ch3_small_property_definition}
\end{table}
\FloatBarrier
% End table

% % % % % % % % % % % % %
% Subsection: From property definitions to Rebel specification
\subsection{\tfPhaseOne{}}
\label{sct:3_prop_to_rebel}
\pinfo{Generator requires spec}
The generator requires a consistent \textit{Rebel} specification in order to generate a system. This means that we have to translate the properties (which are defined in \autoref{cpt:properties}) to a \textit{Rebel} specification. A rebel specification consist of a lifecycle definition together with its event definitions.\\
\\
\pinfo{Property to event translation}
A test case should be able to pass values as parameters to test a specific property for a certain amount of times. We can use the events for this in the \textit{Rebel} definition. An event describes a transition from one state to another and accepts parameters. Additionally, it can have pre- and postconditions, where the postconditions can state what happens when the transaction is being executed. In \autoref{lst:ch3_rebel_event_of_property} the event definition for the \textit{ReflexiveEquality} property written in \textit{Rebel} is shown.
% Listing
\FloatBarrier
\begin{sourcecode}[!ht]
\begin{lstlisting}[language=Rebel]
event reflexiveEquality(x: Money) {
    postconditions {
       new this.result == ( x == x );
    }
}
\end{lstlisting}
\caption{The event definition for the \textit{ReflexiveEquality} property.}
\label{lst:ch3_rebel_event_of_property}
\end{sourcecode}
\FloatBarrier
% End listing
\pinfo{Further explanation, parameters, result field}
The event name and the parameters are used to generate a test case from this event definition. To check whether the property was fulfilled given a certain tuple of parameters, we store the result in a data field called \textit{result}. The test cases can retrieve the value of this field, to determine the result. In case the result value is \textit{false} during testing, a bug has been found. (The \textit{new} keyword is used to state how the field changes when a transition is taking place.)\\
\\
\pinfo{Also need to define the specification itself}
Besides the event definition, we need to write the actual \textit{Rebel} specification to be able to generate a system from it. The specification describes the fields, the events it uses and the life cycle of the state machine. Since we are only interested in testing the events, we can hold the specification itself to a minimum. The life cycle consists of 2 states, the initial and final state. The transition between these states is the event we defined, \textit{ReflexiveEquality}. In \autoref{lst:ch3_rebel_specification_oneprop} a specification used for one property is shown. In the case of multiple properties, we can add these to the events block. In the life cycle, we can comma separate the transitions.
% Listing
\FloatBarrier
\begin{sourcecode}[!ht]
\begin{lstlisting}[language=Rebel]
module gen.specs_money.MoneyExample

import gen.specs_money.MoneyExampleLibrary

specification MoneyExample {
	fields {
        id: Integer @key
		result: Boolean
	}

	events {
		reflexiveEquality[]
	}

	lifeCycle {
		initial init -> result:	reflexiveEquality
		final result
	}
}
\end{lstlisting}
\caption{The event definition for the \textit{ReflexiveEquality} property.}
\label{lst:ch3_rebel_specification_oneprop}
\end{sourcecode}
\FloatBarrier
% End listing

% % % % % % % % % % % % %
% Subsection: Phase 2: Check & build
\subsection{\tfPhaseTwo{}}
\pinfo{Build specification}
Now that we have a specification, that specification can be checked and builded. Which results in an AST of the specification that the test framework can use to generate the tests. This is done by using the existing toolchain that is available for Rebel. The test framework itself is written in \textit{Rascal}.\\
\\
\pinfo{Building to AST}
Building the specification means that the specification is being checked and returns an AST of the specification when the specification is consistent. This is required in order to generate a system from it by using the generator, additonally the AST is used by the test framework to generate the test suite.

% % % % % % % % % % % % %
% Subsection: Phase 3: Generate system
\subsection{\tfPhaseThree{}}
\pinfo{SUT is the generated system}
The generator will be used to generate a system from the specification that we have created. This system which will be used to check each property and is called the SUT throughout this thesis. Note that the generated system is assumed to be runnable. As otherwise the test suite, that will be generated by the test framework (in the next phase), cannot be run against the generated system.

% % % % % % % % % % % % %
% Subsection: Phase 4: Generate test suite
\subsection{\tfPhaseFour{}}
\label{sct:3_tf_phase_four}
\pinfo{Init and generate}
The test suite requires some configuration to work with the generated system. The test framework first initializes the test suite, then generates the test cases.\\
\\
\pinfo{Test suite initialization}
The generated system uses Akka as messaging layer, the required configuration files for running the test suite on the generated system are added by the test framework. Furthermore the test suite is build up such that it first starts the SUT and then runs all the tests against the SUT. Thus when running the test suite (next phase), the SUT will automatically be started such that this does not require additional steps. Furthermore the initialization part can be found back in the source of this project. We do not cover this in detail here since there are no custom settings in there, rather its more default configuration that is only required to make the messaging layer work for the test suite.\\
\\
\pinfo{Test case generation based on event}
The test framework can traverse the AST and generate a test case for each event. A test case is generated by using the templating feature of Rascal, where we fill in event specific data as shown in \autoref{lst:ch3_rascal_testcase_template}. The resulting test case of \textit{ReflexiveEquality} is shown in \autoref{lst:ch3_generated_test_example}.
% Listing
\FloatBarrier
\begin{sourcecode}[!ht]
\begin{lstlisting}[language=Rascal]
public str snippetTestCase(str eventName, list[str] params, int tries) {
	return "\"work with <eventName>\" in {
	          generateRandomParamList(<convertParamsToList(params)>, <tries>).foreach {
	            data: List[Any] =\> {
	              checkAction(<eventName>(
	       	        <for (i <- [0..size(params)]) {>
                      // Iterate over params. Use getMappedType for the casting again
                      data(<i>).asInstanceOf[<getMappedTypeForParam(params[i])>]
                      // Add a comma if needed
                      <if (i != size(params)-1) {>,<}>
                    <}>
	                )
	              )
	            }
	          }
	        }";
}
\end{lstlisting}
\caption{Test case snippet}
\label{lst:ch3_rascal_testcase_template}
\end{sourcecode}
\FloatBarrier
% End listing

% Listing
\FloatBarrier
\begin{sourcecode}[!ht]
\begin{lstlisting}[language=Scala]
    "work with ReflexiveEquality" in {
       generateRandomParamList(List("Money"), 100).foreach {
         data: List[Any] => {
           checkAction( ReflexiveEquality(data(0).asInstanceOf[Money]) )
         }
       }
     }
\end{lstlisting}
\caption{An example of a generated test}
\label{lst:ch3_generated_test_example}
\end{sourcecode}
\FloatBarrier
% End listing
\pinfo{Explanation of a complete test file}
The functions \code{generateParamList()} and \code{checkAction()} are utility functions that are defined in the template that is used for a test file. The \code{generateRandomParamList()} method generates tuples of random values that are used as parameters. \code{checkAction()} is a method that executes the given event and checks whether the resulting value of the result field was \textit{true}. A test file consists of the utility functions and all of the snippets that were generated.

% % % % % % % % % % % % %
% Subsection: Phase 5: Run test suite
\subsection{\tfPhaseFive{}}
\pinfo{How to run the test suit + result}
The test suite can be run with \textit{SBT} by using \code{sbt test}. The log shows detailed information about the tests and shows a summary when the test suit has finished. When running the test framework with the specification that we created in \autoref{sct:3_prop_to_rebel} the test suite finishes successfully, as shown in \autoref{lst:ch3_log_testrun_success}.
% Listing
\FloatBarrier
\begin{sourcecode}[!ht]
\begin{lstlisting}[language=Log]
[info] MoneySpec
[info] - should work with ReflexiveEquality (3 seconds, 686 milliseconds)
[info] ScalaTest
[info] Run completed in 36 seconds, 957 milliseconds.
[info] Total number of tests run: 1
[info] Suites: completed 1, aborted 0
[info] Tests: succeeded 1, failed 0, canceled 0, ignored 0, pending 0
[info] All tests passed.
> Done testing
> ** Tests successful! **
\end{lstlisting}
\caption{Log output of the test suit concerning \textit{ReflexiveEquality}.}
\label{lst:ch3_log_testrun_success}
\end{sourcecode}
\FloatBarrier
% End listing
\pinfo{Notable start up time explanation}
Looking at the run time of this specific run, it shows us that the \textit{ReflexiveEquality} test case was executed within 4 seconds. While the whole test suit run took almost 37 seconds. This difference is due to the fact that the SUT first has to be started, as described in \autoref{sct:3_tf_phase_four}. The log clearly shows which test cases were run and whether these failed or not.\\
\\
\pinfo{Modify generator, demonstrate failing case}
Now that we have a working case, how does this work in case of a test failed? We can simulate a bug by modifying the generator that we use. Let's say that we have a translation error in the generator, such that the equality (==) operator would be translated to a not equal (!=) operator in the generated system. The results show a detailed stack trace of what went wrong along with the input values, such that the issue can be reproduced. \autoref{lst:ch3_log_testrun_failed} shows the output after modifying the generator.
% Listing
\FloatBarrier
\begin{sourcecode}[!ht]
\begin{lstlisting}[language=Log]
[info] MoneySpec
[info] - should work with ReflexiveEquality *** FAILED *** (1 second, 278 milliseconds)
[info]   java.lang.AssertionError: assertion failed: expected CurrentState(Result,Initialised(Data(None,Some(true)))), found CurrentState(Result,Initialised(Data(None,Some(false)))): With command: ReflexiveEquality(-940003591.28 EUR)
[info]   at scala.Predef$.assert(Predef.scala:170)
[info]   at akka.testkit.TestKitBase$class.expectMsg_internal(TestKit.scala:388)
[info]   at akka.testkit.TestKitBase$class.expectMsg(TestKit.scala:382)
[info]   at MoneySpec.expectMsg(MoneySpecSpec.scala:15)
[info]   at MoneySpec.checkAction(MoneySpecSpec.scala:86)
[info]   at MoneySpec$$anonfun$1$$anonfun$apply$mcV$sp$1$$anonfun$apply$mcV$sp$2.apply(MoneySpecSpec.scala:174)
[info]   at MoneySpec$$anonfun$1$$anonfun$apply$mcV$sp$1$$anonfun$apply$mcV$sp$2.apply(MoneySpecSpec.scala:173)
[info]   at scala.collection.immutable.List.foreach(List.scala:381)
[info]   at MoneySpec$$anonfun$1$$anonfun$apply$mcV$sp$1.apply$mcV$sp(MoneySpecSpec.scala:172)
[info]   at MoneySpec$$anonfun$1$$anonfun$apply$mcV$sp$1.apply(MoneySpecSpec.scala:172)
[info]   ...
[info] ScalaTest
[info] Run completed in 35 seconds, 883 milliseconds.
[info] Total number of tests run: 1
[info] Suites: completed 1, aborted 0
[info] Tests: succeeded 0, failed 1, canceled 0, ignored 0, pending 0
[info] *** 1 TEST FAILED ***
> Done testing
> ** Some tests failed! **
\end{lstlisting}
\caption{Log output after modifying the generator}
\label{lst:ch3_log_testrun_failed}
\end{sourcecode}
\FloatBarrier
% End listing

% % % % % % % % % % % % %
% Subsection: Test framework evaluation
\subsection{Test framework evaluation}
\pinfo{Evaluation points}
The tests are generated based on the defined properties. After running the test framework, we evaluate the results and check what can be improved. We define the following criteria to evaluate the test framework after each improvement:

% Coverage
\subsubsection{Coverage}
\pinfo{Tool used for determining the coverage}
The coverage of the components in the SUT that are aimed to be tested by the properties. To determine the coverage, we use an open-source library called \textit{Scoverage} \cite{siteScoverage2017}, which can create a report of the test coverage after running the tests. Since the SUT uses \textit{SBT} as build tool, we use the open-source plug-in \textit{sbt-scoverage}\footnote{https://github.com/scoverage/sbt-scoverage} to integrate this with \textit{SBT}.\\
\\
\pinfo{Same specification for comparing results}
For every evaluation, the same specification and generated system is used to determine the coverage. Note that the first experiment, for example, uses a smaller specification. While the second experiment separates the defined properties into two categories and added preconditions to one of the categories. In order to determine the coverage, the same specification will be used for both experiments, such that the SUT is equal and that the results from each experiment can be compared. In this example, the specifications of the second experiment will be used to determine the coverage of the first experiment.\\
\\
\pinfo{Which data exactly from reports}
The coverage report shows how many statements exist in the SUT and how many of those were covered. Additionally, it does the same for branches, which is the number of different execution paths that could be taken. Since we are not sure how these paths are determined, we will not use this criterion for evaluation. Instead, we will use the statement coverage and the total percentage of coverage. The coverage report also shows which parts of each statement have been executed, it shows green highlighting for covered parts and red highlighting for uncovered parts. The coverage highlighting for the \textit{ReflexiveEquality} property described in this chapter highlights everything green, meaning that the whole statement was executed, as shown in \autoref{fig:ch3_eval_e0_highlighting_reflexive-equality}.
% Figure
\FloatBarrier
\begin{figure}[!ht]
%\frame{
	\includegraphics[width=\linewidth]{figures/eval_e0_reflexive-equality}
%}
\caption{Test coverage example for \textit{ReflexiveEquality}}
\label{fig:ch3_eval_e0_highlighting_reflexive-equality}
\centering
\end{figure}
\FloatBarrier
% End figure
\pinfo{Property coverage}
The logic of a specification is defined in one Class in the generated system, which is called \textit{Logic} and prefixed by the specification name. We will only look at these classes to check to which extent the properties have been tested, using the highlighting that shows the coverage.\\
\\
\pinfo{Won't reach 100\%}
The generated system also contains some other logic that is more related to how it communicates with other instances when it is deployed, which is not something that we test. As a result, we will not be able to bring the test coverage to 100\%. However, all files in the generated system will be used to determine the overall coverage percentage. Since we use the same SUT to determine the coverage of the test suite on the SUT, the higher the coverage, the more complete it tests the defined properties in the generated system.

% Amount of bugs
\subsubsection{Amount of bugs}
\pinfo{Not a hard criteria, still using for indication}
The number of bugs found by an experiment also describes how effective the experiment was. Although this can not be a hard criteria, as it can vary per case. Consider that the system was already tested thoroughly, such that the bugs that this test suite would have found are already solved. This would mean that the amount of bugs found would remain 0, thus wouldn't have any effect as criteria. It is still an interesting part, as the amount of bugs found proofs that the test suite is able to find bugs. Because of this, we will report on this criteria and take it into account, but it will not be a critical criteria on determining whether one experiment was more successful than the other.

%\todo[inline]{Efficiency? Not yet.\\
%Can be done like getting coverage of libraries too, then substitute amount of properties and show that the same coverage can be received with less property tests. Thus less test cases, faster speed, same coverage. However, this would entail that the amount of bugs could go lower, as certain specific properties wont be tested if we require only coverage.}


% % % % % % % % % % % % % % % % % % % % % % % % % % % % % % % % % % % % %
% Section: Conclusion
\section{Conclusion}
\pinfo{Our approach, like QuickCheck}
The research question for this chapter lead as follows:\rqTwo{}. Existing approaches often require the SUT to be written in the same language. This was not possible when testing the generator in our case. The generator is being used to generate a system that is being used to test the generator. We use an approach that is similar to QuickCheck but using a \textit{Rebel} specification and the generated system to check whether the properties hold when using the generator.\\
\\
\pinfo{Combining all steps}
We demonstrated a full cycle based on one property, which indicated that this approach works to check a property. A full cycle consists of the following 5 phases:
\begin{enumerate}
\item \tfPhaseOne{}
\item \tfPhaseTwo{}
\item \tfPhaseThree{}
\item \tfPhaseFour{}
\item \tfPhaseFive{}
\end{enumerate}
The first step is done manually by translating the properties to a consistent \textit{Rebel} specification. The test framework is able to execute the other phases, which can be found in the \code{Main.rsc} file in the source code of this project.\\
\\
\pinfo{Larger specification for experiments}
For the experiments all the properties defined in \autoref{cpt:properties} will be used. This results in a bigger specification which can be used to test the generator automatically by using the test framework. After running the test framework, we evaluate it on the coverage and amount of bugs found metrics. The event definitions of each property defined in \autoref{cpt:properties} can be found in \autoref{app:a_event_definitions}.

% % % % % % % % % % % % % % % % % % % % % % % % % % % % % % % % % % % % %
% Section: Threats to validity
\section{Threats to validity}

\subsection*{Uncompilable system}
When the SUT is unable to compile, the test framework cannot proceed. As it cannot run the generated test suite against the SUT in that case. Although such errors could be detected by the test framework, it is out of scope for this thesis. IT is hard to argue whether the compilation error would be a bug or something else, as it can have many causes. However, when running the test framework this might still occur, which is a threat to this approach.

\subsection*{Accuracy}
\pinfo{Possibly incorrect results, due to Scoverage}
To evaluate the test framework we use coverage as a metric, which could have reported incorrect results. Scoverage is being used to determine the coverage and to generate a report from it. Since we are using random data as input, the results of the test coverage can fluctuate by small amounts in each run. However, we can still reason about the differences when there is a big difference between certain experiments. Additionally to the coverage we used the number of bugs that were found as another metric. This metric depends on which system the test suite is being run and if the system already fixed the bugs that the test suite would find. It is also the case that after fixing the bugs that were found earlier, this metric can be seen as unnecessary, as it would result in 0 then.

\subsection*{One system}
\pinfo{Only one generator}
Only one generator is being used throughout this thesis. However, it could be useful to make the test framework compatible with the other generators and generated systems too. This enables reasoning about the different implementations and its generators. Some changes are required to make the test framework compatible with these systems. But by doing so, every generated system for which a generator is built by ING can be checked based on the same properties, resulting in that the defined properties are checked thoroughly on every system and that inequalities can be detected between the different generators.\\
\\
\pinfo{Probably causing compile errors}
A threat in doing so is that one of the other generators might not support some translations of each expression that is used in the specification that we created. Thus this test framework can also be used to check whether every expression variant is taken into account by the generator. Unfortunately, an error in this translation would be blocking, in that it can lead to a generated system that is not able to compile. Resulting in that the test framework cannot proceed to run the test suite on the generated system. This could be used as a way to check the generators too. Although compilation errors were not the aim of the project, as compilation errors can have many causes, the test framework can still be used to detect those to a certain extent.

\subsection*{Whitebox implementation}
\pinfo{ScalaCheck, not testing generator then}
We chose to use property-based testing and implement the required functionality ourselfes, resulting in a white-box implementation. This means that we expect that our values generation is working correctly too. In case this isn't working correctly, this has to be fixed too.\\
\\
Another way how this could be done was to check how the custom types were generated to \textit{Scala}. And then generate a \textit{Scala} test project using the same types. Writing property tests for each type could achieve the same goal when it comes to checking the implementation of this component in the generated system. However, if we would follow this approach, we wouldn't use the generator to translate the \textit{Rebel} expressions to \textit{Scala}. This results in that the generator itself is still not being tested. With our approach, we test the generator and are able to find errors in the generator. Although we cannot conclude that the generator is implemented correctly if the generated test suit runs successful, rather we can conclude that the properties it checks for are satisfied.

% % % % % % % % % % % % % % % % % % % % % % % % % % % % % % % % % % % % %
% Chapter: Experiment 1
% % % % % % % % % % % % % % % % % % % % % % % % % % % % % % % % % % % % %
\chapter{Experiment 1: Using random input}
\label{cpt:experiment1}
\pinfo{Props to test cases, implication returns True}
The properties that we defined in \autoref{cpt:properties} are translated into
test cases as described in \autoref{cpt:testmechanics}. In this experiment, we
expect to find some bugs that were unknown before by using the test framework.
When we have triggered some bugs, an investigation is needed to check what the
cause is of that bug. Next, we can categorize the bugs found to come to an
answer to this research question:\rqThree{}

% % % % % % % % % % % % % % % % % % % % % % % % % % % % % % % % % % % % %
% Section: Method
\section{Method}
%\label{sec:initial_case}
\pinfo{Like QuickCheck, recall cycle shortly}
In the first experiment, each property will be tested 100 times with random
input values. This means that if the property holds for 100 tests, it is
reported to be successfully satisfying the property. This is a similar approach
compared to what \textit{QuickCheck} does when checking properties. Unlike
\textit{QuickCheck}, the test framework does not shrink the input values to come
with minimum values for which the case fails. Instead, it will just report the
values that were used when the property failed.

% % % % % % % % % % % % % % % % % % % % % % % % % % % % % % % % % % % % %
% Section: Results
\section{Results}
Two runs are being done to detect bugs in the generator in this experiment
because the first run terminated quickly. The test framework was unable to
proceed in testing every property in this run.

% % % % % % % % % % % % %
% Subsection: First run
\subsection{First run}
\pinfo{Termination, compile error due to library}
The first run results into a termination of the run due to a compile error in the generated system. Although we made the assumption that the generated system should be compilable, this error came from a property definition that was expected to hold, namely \textit{AssociativeMultiplicationInteger1} (\autoref{ssct:4_associativity}). Which is why we can consider this as an error that is found when using the test framework. The error describes that an overloaded method cannot be applied to the \textit{Money} type, as shown in \autoref{lst:ch5_firstrun_termination_log}.
% Listing
\FloatBarrier
\begin{sourcecode}[!ht]
\begin{lstlisting}[language=Log]
[error] MoneySpec.scala:316: overloaded method value * with alternatives:
[error]   (x: Double)Double <and>
[error]   (x: Float)Float <and>
[error]   (x: Long)Long <and>
[error]   (x: Int)Int <and>
[error]   (x: Char)Int <and>
[error]   (x: Short)Int <and>
[error]   (x: Byte)Int
[error]  cannot be applied to (squants.market.Money)
[error]           Initialised(Data(result = Some(((((x * y)) * z) == (x * ((y * z)))))))
[error]                                                                 ^
[error] MoneySpec.scala:441: overloaded method value * with alternatives:
[error]   (x: Double)Double <and>
[error]   (x: Float)Float <and>
[error]   (x: Long)Long <and>
[error]   (x: Int)Int <and>
[error]   (x: Char)Int <and>
[error]   (x: Short)Int <and>
[error]   (x: Byte)Int
[error]  cannot be applied to (squants.market.Money)
[error]             checkPostCondition((nextData.get.result.get == (((((x * y)) * z) == (x * ((y * z)))))), "new this.result == ( (x*y)*z == x*(y*z) )")
[error]                                                                                    ^
[error] two errors found
[error] (compile:compileIncremental) Compilation failed
[error] Total time: 79 s, completed 4-aug-2017 13:03:45
> Done testing
> ** Some tests failed! **
\end{lstlisting}
\caption{Log output first test run resulting in a termination.}
\label{lst:ch5_firstrun_termination_log}
\end{sourcecode}
\todo{Up arrow is incorrectly aligned in the listing due to layout}
\FloatBarrier
% End listing
\pinfo{Investigation, found property and var types}
The error log does not clearly indicate what exactly went wrong. It doesn't
show clearly which property is causing this error. Also, it does not describe
what the types of the variables were. Investigating the generated system reveals
that both errors were happening when dealing with the
\textit{AssociativeMultiplicationInteger1} property. This means that the
variables \textit{x}, \textit{y} and \textit{z} are of type \textit{Integer},
\textit{Integer}, \textit{Money} respectively, as described in
\autoref{ssct:4_associativity}. Temporarily disabling this property allows the
test framework to proceed further.

% % % % % % % % % % % % %
% Subsection: Second run
\subsection{Second run}
\pinfo{Failing tests, describing each}
After disabling the \textit{AssociativeMultiplicationInteger1} property, the
test framework was able to run completely. This results in 7 failing tests. For
each test, the input values for which the property doesn't hold are logged such
that the error can be reproduced. In
\autoref{tbl:experiment1_overview_second_run} an overview of the failing
properties, along with its input values (\textit{x}, \textit{y} and \textit{z})
are shown.
% Table
\FloatBarrier
\begin{table}[!ht]
\centering
\begin{tabular}{llll}
\hline
\textbf{Property name}               & \textbf{x}               & \textbf{y}        & \textbf{z}         \\ \hline
DistributivePercentage1              & 0.51                     & -311254801.77 EUR & -707194075.77 EUR  \\
DistributivePercentage2              & 0.93                     & 2089630160.75 EUR & -1316628389.49 EUR \\
DistributiveInt2                     & -883022216               & -298435082.93 EUR & 715725888.96 EUR   \\
AssociativeMultiplicationPercentage2 & 840296462                & 1771903729.60 EUR & 0.53               \\
DistributiveInt1                     & -1790274467.41 EUR       & 1691684272        & 1449321647         \\
AssociativeMultiplicationInteger2    & -1852801029.34 EUR       & -1309504561       & 1880170895         \\
AssociativeMultiplicationPercentage1 & -352883323.42 EUR        & 0.27              & 294211708          \\ \hline
\end{tabular}
\caption{Overview of failing tests along with its input values}
\label{tbl:experiment1_overview_second_run}
\end{table}
\FloatBarrier
% End table


% % % % % % % % % % % % % % % % % % % % % % % % % % % % % % % % % % % % %
% Section: Analysis
\section{Analysis}
For each failed test we investigate what went wrong. The first four tests
reveal precision problems when using the \textit{Money} type in calculations.
The latter three tests were also failing because of these precision problems.
However, these tests were also failing after the precision errors were fixed.
Thus, for the latter 3 tests, another version of the generated system was used,
which contains the fixes for the precision problems. This is done such that we
are able to reveal the other errors that these properties can reveal.

% % %
% Explaining failing cases.
% % %
% % %
% DistributivePercentage1
\subsubsection{DistributivePercentage1}
\label{ssct:ch5_distributivePercentage1}
This property uses a \textit{Percentage} value and two \textit{Money} values
for its tests. The values are named \textit{x}, \textit{y} and \textit{z}
respectively. To check this failing test, we check the results of the
intermediate calculations in the formula that is being used. In
\autoref{ch4_init_check_DistributivePercentage1} the values are shown for which
the test case failed, along with the intermediate calculations. The intermediate
calculations seem to be fine, as the results are almost the same when we compare
the results of the \textit{Scala} evaluation and the expected result. The
resulting left-hand side of the expression contains a precision error, which is
caused when multiplying a \textit{Percentage} (the \textit{x} variable) with a
\textit{Money} type (the result of \textit{y+z} in this case).
% Table
\FloatBarrier
\begin{table}[!ht]
\centering
\begin{tabular}{rll}
\hline
\textbf{Variable}      & \textbf{Value}                   & \textbf{Type}                \\ \hline
x                      & 0.51                             & Percentage                   \\
y                      & -311254801.77 EUR                & Money                        \\
z                      & -707194075.77 EUR                & Money                        \\ \hline
\textbf{Formula}       & \textbf{Scala result}            & \textbf{Expected result}     \\ \hline
x*(y+z) == (y*x)+(z*x) & false                            & true                         \\
\textbf{x*(y+z)}       & \textbf{-519408927.54539996 EUR} & \textbf{-519408927.5454 EUR} \\
(y*x)+(z*x)            & -519408927.5454 EUR              & -519408927.5454 EUR          \\
                       &                                  &                              \\
y+z                    & -1018448877.54 EUR               & -1018448877.54 EUR           \\
y*x                    & -158739948.9027 EUR              & -158739948.9027 EUR          \\
z*x                    & -360668978.6427 EUR              & -360668978.6427 EUR          \\ \hline
\end{tabular}
\caption{DistributivePercentage1: Precision error when multiplying a \textit{Percentage} with \textit{Money}}
\label{ch4_init_check_DistributivePercentage1}
\end{table}
\FloatBarrier
% End table

% % %
% DistributivePercentage2
\subsubsection{DistributivePercentage2}
This test case looks similar as \nameref{ssct:ch5_distributivePercentage1}. It
uses the same type of variables, but the expression is slightly different. In
\autoref{ch4_init_check_DistributivePercentage2} the result and the intermediate
calculations of a failing case are shown. What can be seen here is that the
precision error occurs when the \textit{Money} type is multiplied by the
\textit{Percentage} type. While with \nameref{ssct:ch5_distributivePercentage1}
it was the other way around.
% Table
\FloatBarrier
\begin{table}[!ht]
\centering
\begin{tabular}{rll}
\hline
\textbf{Variable}      & \textbf{Value}                   & \textbf{Type}                 \\ \hline
x                      & 0.93                             & Percentage                    \\
y                      & 2089630160.75 EUR                & Money                         \\
z                      & -1316628389.49 EUR               & Money                         \\ \hline
\textbf{Formula}       & \textbf{Scala result}            & \textbf{Expected result}      \\ \hline
(y+z)*x == (y*x)+(z*x) & false                            & true                          \\
(y+z)*x                & 718891647.2718 EUR               & 718891647.2718 EUR            \\
(y*x)+(z*x)            & 718891647.2718001 EUR            & 718891647.2718 EUR            \\
                       &                                  &                               \\
y+z                    & 773001771.26 EUR                 & 773001771.26 EUR              \\
\textbf{y*x}           & \textbf{1943356049.4975002 EUR}  & \textbf{1943356049.4975 EUR}  \\
\textbf{z*x}           & \textbf{-1224464402.2257001 EUR} & \textbf{-1224464402.2257 EUR} \\ \hline
\end{tabular}
\caption{DistributivePercentage1: Precision error when multiplying a \textit{Money} with \textit{Percentage}}
\label{ch4_init_check_DistributivePercentage2}
\end{table}
\FloatBarrier
% End table

% % %
% DistributiveInt2
\subsubsection{DistributiveInt2}
\label{ssct:ch5_distributiveInt2}
This case uses \textit{Integer} in conjunction with the \textit{Money} type.
Earlier cases showed that there was a precision error when using the
\textit{Percentage} and \textit{Money} types. Since the \textit{Percentage} type
is translated to a \textit{Double} in the generated system, it can be expected
that there would be precision problems occurring. As this is a known issue with
types that use floating-point arithmetic~\cite{goldberg1991every}. This case
reveals that a precision error also occurs when multiplying \textit{Money} with
an \textit{Integer}. In the intermediate calculations when investigating a
failing test with its values are shown in
\autoref{ch4_init_check_DistributiveInt2}. The last two rows, in boldface, show
that a precision error occurs when \textit{Money} is multiplied by an
\textit{Integer}.
% Table
\FloatBarrier
\begin{table}[!ht]
\centering
\begin{tabular}{rll}
\hline
\textbf{Variable}  & \textbf{Value}                   & \textbf{Type}                       \\ \hline
x                  & -883022216                       & Integer                             \\
y                  & -298435082.93 EUR                & Money                               \\
z                  & 715725888.96 EUR                 & Money                               \\ \hline
\textbf{Formula}   & \textbf{Scala result}            & \textbf{Expected result}            \\ \hline
(x*y)*z == x*(y*z) & false                            & true                                \\
(x*y)*z            & -368477052257036740 EUR          & -368477052257036762.48 EUR          \\
x*(y*z)            & -368477052257036796 EUR          & -368477052257036762.48 EUR          \\
                   &                                  &                                     \\
y+z                & 417290806.03 EUR                 & 417290806.03 EUR                    \\
\textbf{y*x}       & \textbf{263524808260992384 EUR}  & \textbf{263524808260992372.88 EUR}  \\
\textbf{z*x}       & \textbf{-632001860518029180 EUR} & \textbf{-632001860518029135.36 EUR} \\ \hline
\end{tabular}
\caption{DistributiveInt2: Precision error when multiplying \textit{Money} with an \textit{Integer}}
\label{ch4_init_check_DistributiveInt2}
\end{table}
\FloatBarrier
% End table

% % %
% AssociativeMultiplicationPercentage2
\subsubsection{AssociativeMultiplicationPercentage2}
The earlier cases already shown a precision error when using \textit{Double}
and \textit{Integer} in conjunction with \textit{Money}. This case triggers the
same problem, but also reveals that the same thing happens when multiplying an
\textit{Integer} with \textit{Money}. While with
\nameref{ssct:ch5_distributiveInt2} it was the other way around. The
intermediate calculations are shown in
\autoref{ch4_init_check_AssociativeMultiplicationPercentage2}, the calculation
of multiplying an \textit{Integer} with \textit{Money} is shown in boldface.
Additionally, this case shows that the small precision errors can cause a
noticeable difference, which lead to a difference of 130 EUR in this case (in
the \textit{Scala} results).
% Table
\FloatBarrier
\begin{table}[!ht]
\centering
\begin{tabular}{rll}
\hline
\textbf{Variable}  & \textbf{Value}                   & \textbf{Type}                      \\ \hline
x                  & 840296462                        & Integer                            \\
y                  & 1771903729.60 EUR                & Money                              \\
z                  & 0.53                             & Percentage                         \\ \hline
\textbf{Formula}   & \textbf{Scala result}            & \textbf{Expected result}           \\ \hline
(x*y)*z == x*(y*z) & false                            & true                               \\
(x*y)*z            & 789129950543366910 EUR           & 789129950543366877.856 EUR         \\
x*(y*z)            & 789129950543366780 EUR           & 789129950543366877.856 EUR         \\
                   &                                  &                                    \\
\textbf{x*y}       & \textbf{1488924434987484670 EUR} & \textbf{1488924434987484675.2 EUR} \\
y*z                & 939108976.688 EUR                & 939108976.688 EUR                  \\ \hline
\end{tabular}
\caption{AssociativeMultiplicationPercentage2: Precision error causing bigger differences}
\label{ch4_init_check_AssociativeMultiplicationPercentage2}
\end{table}
\FloatBarrier
% End table

% % %
% DistributiveInt1
\subsubsection{DistributiveInt1}
This case also uses three variables: \textit{x}, \textit{y} and \textit{z}.
Which are of type \textit{Money}, \textit{Integer} and \textit{Integer}
respectively. In \autoref{ch4_init_check_DistributiveInteger1} the different
values are shown of the calculation between \textit{Scala} and the expected
result. In boldface, it shows how the addition of two (positive) integers
results in a negative value. This is because the resulting value of the addition
would be bigger than the maximum value of what an \textit{Integer} type can
hold. This results occurrence is called integer overflow~\cite{brumley2007rich}.
The operation that is being done does not check or prevent this overflowing
behaviour. Although this could be expected when using the \textit{Integer} type,
it can lead other errors classes of vulnerabilities, including stack and heap
overflows~\cite{wang2009intscope}, when this is not being prevented.
% Table
\FloatBarrier
\begin{table}[!ht]
\centering
\begin{tabular}{rll}
\hline
\textbf{Variable}      & \textbf{Value}              & \textbf{Type}               \\ \hline
x                      & -1790274467.41 EUR          & Money                       \\
y                      & 1691684272                  & Integer                     \\
z                      & 1449321647                  & Integer                     \\ \hline
\textbf{Formula}       & \textbf{Scala result}       & \textbf{Expected result}    \\ \hline
x*(y+z) == (x*y)+(x*z) & false                       & true                        \\
x*(y+z)                & 2065907589620385223.57 EUR  & -5623262698769382599.79 EUR \\
(x*y)+(x*z)            & -5623262698769382599.79 EUR & -5623262698769382599.79 EUR \\
                       &                             &                             \\
\textbf{y+z}           & \textbf{-1153961377}        & \textbf{3141005919}         \\
x*y                    & -3028579159080673575.52 EUR & -3028579159080673575.52 EUR \\
x*z                    & -2594683539688709024.27 EUR & -2594683539688709024.27 EUR \\ \hline
\end{tabular}
\caption{DistributiveInteger1: \textit{Integer} overflows when using addition}
\label{ch4_init_check_DistributiveInteger1}
\end{table}
\FloatBarrier
% End table

% % %
% AssociativeMultiplicationInt2
\subsubsection{AssociativeMultiplicationInteger2}
% - Integer overflows, causing negative values etc.
For this case thrFor this case three variables are used: \textit{x}, \textit{y} and \textit{z},
which are of type \textit{Money}, \textit{Integer} and \textit{Integer}
respectively. In \autoref{ch4_init_check_AssociativeMultiplicationInteger2} the
values of a failing test case are shown, along with the intermediate formula
steps. On the left-hand side of the expression, we see the expected results,
while on the right-hand side there is a big difference between the
\textit{Scala} result and the expected result (shown in boldface in
\autoref{ch4_init_check_AssociativeMultiplicationInteger2}). The result value of
the operation would become smaller than the minimum value of what an
\textit{Integer} value can be. When this happens, an integer underflow occurs
which results in the value being ``wrapped'' to the maximum
value~\cite{brumley2007rich}, and then being used to calculate further. This
underflow results in an unexpected amount, which we can see back in the results.
The operation neither checks for underflowing an \textit{Integer} value nor does
it prevent it.
% Table
\FloatBarrier
\begin{table}[!ht]
\centering
\begin{tabular}{rll}
\hline
\textbf{Variable}  & \textbf{Value}                      & \textbf{Type}                                \\ \hline
x                  & -1852801029.34 EUR                  & Money                                        \\
y                  & -1309504561                         & Integer                                      \\
z                  & 1880170895                          & Integer                                      \\ \hline
\textbf{Formula}   & \textbf{Scala result}               & \textbf{Expected result}                     \\ \hline
(x*y)*z == x*(y*z) & false                               & true                                         \\
(x*y)*z            & 4561767263499657218201769467.30 EUR & 4561767263499657218201769467.30 EUR          \\
\textbf{x*(y*z)}   & \textbf{3877739486117270379.94 EUR} & \textbf{4561767263499657218201769467.30 EUR} \\
                   &                                     &                                              \\
x*y                & 2426251398546224819.74 EUR          & 2426251398546224819.74 EUR                   \\
y*z                & -2092906591                         & -2462092362461952095                         \\ \hline
\end{tabular}
\caption{AssociativeMultiplicationInteger2: \textit{Integer} underflows when using multiply}
\label{ch4_init_check_AssociativeMultiplicationInteger2}
\end{table}
\FloatBarrier
% End table

% % %
% AssociativeMultiplicationPercentage1
\subsubsection{AssociativeMultiplicationPercentage1}
% - Generating big double, precision can't be known for Squants!
In this case there are three variables: \textit{x}, \textit{y} and \textit{z},
which are of type \textit{Money}, \textit{Percentage} and \textit{Integer}
respectively. In Table \ref{ch4_init_check_AssociativeMultiplicationPercentage1}
the values and intermediate calculations are shown of a failing case, such that
we can reason about the results. The row in boldface shows a precision error
when comparing the results of \textit{Scala} and the expected result with each
other. This issue is caused by the \textit{Percentage} that is being used. In
the implementation, the \textit{Percentage} is being translated into a
\textit{Double} value. In this case, this \textit{Double} value is then being
multiplied with an \textit{Integer}. This results in a \textit{Double} value
containing a precision error, which is related to the problems with
floating-point arithmetic~\cite{goldberg1991every}.
% Table
\FloatBarrier
\begin{table}[!ht]
\centering
\begin{tabular}{rll}
\hline
\textbf{Variable}  & \textbf{Value}             & \textbf{Type}               \\ \hline
x                  & -352883323.42 EUR          & Money                       \\
y                  & 0.27                       & Percentage                  \\
z                  & 294211708                  & Integer                     \\ \hline
\textbf{Formula}   & \textbf{Scala result}      & \textbf{Expected result}    \\ \hline
(x*y)*z == x*(y*z) & false                      & true                        \\
(x*y)*z            & -28032049433190944 EUR     & -28032049433190942.3672 EUR \\
x*(y*z)            & -28032049433190948 EUR     & -28032049433190942.3672 EUR \\
                   &                            &                             \\
x*y                & -95278497.3234 EUR         & -95278497.3234 EUR          \\
\textbf{y*z}       & \textbf{79437161.16000001} & \textbf{79437161.16}        \\ \hline
\end{tabular}
\caption{AssociativeMultiplicationPercentage1: A precision error when using \textit{Percentage}}
\label{ch4_init_check_AssociativeMultiplicationPercentage1}
\end{table}
\FloatBarrier
% End table
Additionally, we can see a difference in the results on the left-hand and
right-hand side of the expression evaluation in \textit{Scala}. Whereas the
intermediate step for the left-hand side is calculated correctly. This also hints to
the bug in the \textit{Money} type which we already found when testing the
\textit{DistributiveInteger2} property. For the right-hand side, we cannot say
this immediately, as there is already an error in the intermediate step.\\
\\
This property revealed a precision error when the \textit{Percentage} type is
being used. The \textit{Percentage} is being translated to a \textit{Double}
value, causing operations with it to have precision errors. In this case the
\textit{Percentage} is being multiplied by an \textit{Integer}.

% % % % % % % % % % % % % % % % % % % % % % % % % % % % % % % % % % % % %
% Section: Evaluation criteria
\section{Evaluation criteria}

% Property coverage
\pinfo{Not tested implicative properties}
When looking at the coverage results, it is notable that the
condition for the if-clause of the implicative properties is often not being
triggered. Resulting in that it always returns \textit{true}, as this is how it
was specified in the specification (the else-clause of an implicative property).
This is caused by the random values that are being used as input. An example of
this is the \textit{TransitiveEquality} property
(\code{x == y \&\& y == z $\implies$ x == z}). When relying on random data,
there is a seldom chance that 3 values are equal to each other. Thus we could
optimize the random values such that the condition holds, such that we also test
these properties such that the if-clause is triggered. In
\autoref{fig:experiment1_coverage_implicative_property} the coverage for the
\textit{TransitiveEquality} property, green highlighting indicates the
statements that are executed, while red highlighting indicates statements that
were not checked at all.
% Figure
\FloatBarrier
\begin{figure}[!ht]
%\frame{
	\includegraphics[width=\linewidth]{figures/experiment1_coverage_implicative_property}
%}
\caption{Test coverage of an implicative property (\textit{TransitiveEquality})}
\label{fig:experiment1_coverage_implicative_property}
\centering
\end{figure}
\FloatBarrier
% End figure

% Total coverage
The first criteria to evaluate an experiment was to determine the test
coverage. The implicative properties are not covered when using random
values as input data. The other properties, which do not use implication, are
fully tested though. In \autoref{fig:experiment1_evaluation_result} the results of the coverage report are shown, with a total of 87,80\% coverage. The files with a name
ending with ``Logic'' contain the implementation of the properties. The other 12,20\% code is not tested, but this part is also not related to the implementation of the properties. Because of this, it is not required to achieve the full 100\% coverage. Note that the coverage does not include the coverage over the libraries that the system uses. Some implementation details are depended on the libraries that the generated system use, thus it can not be concluded that 87,80\% of the system is now tested.
% Figure
\FloatBarrier
\begin{figure}[!ht]
%\frame{
	\includegraphics[width=\linewidth]{figures/eval_experiment1}
%}
\caption{Test coverage report of the first experiment}
\label{fig:experiment1_evaluation_result}
\centering
\end{figure}
\FloatBarrier
% End figure
The branch coverage is reported to be 66,67\%. This is caused by the implicative properties that are being translated to \textit{if-else} statements, where the \textit{else} branch is translated to \textit{true}. Furthermore the coverage could be increased as not every part of the condition of implicative properties are being checked, as we have seen in \autoref{fig:experiment1_coverage_implicative_property}.

% % of bugs
\pinfo{\# of bugs, 8}
The second criterion that we defined is the number of bugs that we have found
by doing the experiment. By using this approach we found a total of 8 bugs. A
compilation error (1x), overflow/underflow errors (2x) and precision errors (5x). Many of these
precision errors originated from the library that is used for the \textit{Money}
type, for which we created an issue on
\textit{Github}\footnote{https://github.com/typelevel/squants/issues/265}. An
improvement of the test framework to also cover the implicative properties might
result in more bugs that can be found.

% % % % % % % % % % % % % % % % % % % % % % % % % % % % % % % % % % % % %
% Section: Conclusion
\section{Conclusion}
% 1: Squants using doubles internally. Small precision error can lead to bigger (unexpected) amounts as we've seen in AMP2.
% 	 - DistributivePercentage1: Multiplying Double with Money causes a precision error
%	 - DistributivePercentage2: Multiplying Money with Double causes a precision error
% 	 - DistributiveInt2: Multiplying Money with Integer
%    - AssociativeMultiplicationPercentage2: Multiplying Integer with Money. And with double. Causing big numbers to be losing precision anyway. Bug in Squants is the problem here though.
% 2: Operations on an Integer are over/underflowing, causing incorrect results.
%	 - AssociativeMultiplicationInt2: When using Multiply on Ints
%	 - DistributiveInt1: When using Addition on Ints
% 3: Percentage is double, causing big values to lose precision, which can cascade through the expressions
%	 - AssociativeMultiplicationPercentage1: Multiplying Percentage with Int, causing a big Double value with precision loss. Squants cannot guess the actual number anymore
\pinfo{Recap, found 8 bugs using this approach}
In this first experiment, we tested each property that was defined in
\autoref{cpt:properties} 100 times with using random values as input. First, the
test suite terminated due to a compilation error. After disabling the causing
property (temporarily), a total of 7 tests were failing. In this experiment, we
managed to find precision errors and overflow/underflow errors. Additionally, we
found a compilation error when using a property which was expected to hold. The
precision errors that were related to the \textit{Money} type were reported.
These bugs are fixed in the next version of the generator.\\
\\
\pinfo{Implicative properties not tested}
Although many properties were tested successfully, the test framework also
indicates that the implicative properties were satisfied. However, when looking
at the statement coverage of the implicative properties, we saw that the
if-clause is often not being. Meaning that it would call the else-clause which
simply returns true. When relying on random data, there is a seldom chance that
the if-clause is being triggered. Thus we could optimize the generated input
values such that these satisfy the condition for the if-clause.

% % % % % % % % % % % % % % % % % % % % % % % % % % % % % % % % % % % % %
% Section: Threats to validity
\section{Threats to validity}

\subsection*{Fixed amount of tries}
\pinfo{Fixed amount of tries not substantiated}
The 100 tries to check a property is a fixed number that is being used. But why
exactly this number and not a higher or lower number? It might be the case that
some errors are not triggered because of this fixed amount. Running more cases
might be revealing an additional error, or it might not. In case it doesn't, it
means that the test framework just requires more time to run the whole test suite,
while it does not have an effect on the results. 100 seems to be an amount that
works such that it consistently reports the same amount of failing tests
however, this has checked by running it the test runs multiple times, also with
using numbers like 300 or 50 for the amount. Also \textit{QuickCheck} uses this
amount to check a property. During this thesis, we stick with 100 as the number
of tries. Finding the optimal number of tries is left as future work, thus
remaining as a threat to validity in this approach.

\subsection*{Amount of test runs}
\pinfo{Multiple tries, each time 7 failed, assumed same causes}
The test framework has been executed several times for this experiment. With
each run, 7 tests were failing, which consistently were the same tests on every
run. Because of this, we assumed that the causes of these failures were the
same, as the same tests failed on each run. However, the causes might not have
been the same for these runs, thus remaining as a threat to validity.

\subsection*{Unfixed issue}
\pinfo{Compile error not fixed}
Unfortunately, the compilation error has not been fixed throughout this project
however, it is an open issue on
\textit{Github}\footnote{https://github.com/typelevel/squants/issues/281}. Some
precision errors originated from a library used in the generated system, called
\textit{Squants}~\cite{siteSquants2017}. An issue was created covering these precision
errors, which were fixed in the next release of that library. As for the
overflow and underflow errors, these occurred when using the \textit{Integer}
type in \textit{Rebel}. When using the \textit{Integer} type, this might have
been expected behaviour which causes this to happen. However, the generated
system does not check whether this happens, nor does it prevent this. We
consider the overflow and underflow errors as unexpected, as the \textit{Rebel}
language does not support other numeric types to hold a bigger value than an
\textit{Integer} supports.

% % % % % % % % % % % % % % % % % % % % % % % % % % % % % % % % % % % % %
% Chapter: Experiment 2
% % % % % % % % % % % % % % % % % % % % % % % % % % % % % % % % % % % % %
\chapter{Experiment 2: Smarter values generation}
\label{cpt:experiment2}
% % % % % % % % % % % % % % % % % % % % % % % % % % % % % % % % % % % % %
% Section: Smarter values generation
%\section{Smarter values generation}
%\label{sec:ch4_smarter_values_generation}
\pinfo{Context - result of experiment 1, what aim is now}
Some bugs were found by using random input values in the first experiment. However the implicative properties were not effectively checked in terms of triggering the if-clause when using random input values. This is what we aim to improve in this experiment, expecting to detect more bugs.

% % % % % % % % % % % % % % % % % % % % % % % % % % % % % % % % % % % % %
% Section: Method
\section{Method}
\pinfo{2 categories separation}
We can separate the properties that we have in 2 different categories: those using implication ($\implies$) and those that do not. The defined properties are being separated over two specifications according to which category these belong. We name these specifications \textit{MoneyExpressions} and \textit{MoneyConditionals}. For the \textit{MoneyConditionals} specification (the implicative properties), another way of generating the random values is more useful. The random input values are being optimized such that the condition of the if-clause of these properties are being satisfied. For the other specification (\textit{MoneyExpressions}), the earlier approach (random values as input) can still be used, we do not need to change this functionality for these cases.\\
\\
\pinfo{Adding preconditions}
In the \textit{MoneyConditionals} specification, the condition to trigger the if-clauses will be added to the preconditions of each the event definition, such that these can be used to generate the values matching this clause. The updated event definition of the \textit{Symmetric} property is shown in \autoref{lst:experiment2_updated_definition} for example. Where the preconditions have been added to the event definition.
% Listing
\FloatBarrier
\begin{sourcecode}[!ht]
\begin{lstlisting}[language=Rebel]
event symmetric(x: Money, y:Money) {
    preconditions {
        x == y;
    }
    postconditions {
       new this.result == ( (x == y) ? y == x : False );
    }
}
\end{lstlisting}
\caption{The updated event definition of the \textit{Symmetric} property}
\label{lst:experiment2_updated_definition}
\end{sourcecode}
\FloatBarrier
% End listing
\pinfo{Generating checks for preconditions}
When generating the test suite, the events are being traversed. In case an event with some preconditions is found, it generates a list of value tuples that satisfy the condition to trigger the if-clause. Which is different compared to the generated tests in \autoref{cpt:experiment1}.\\
\\
\pinfo{Diff1: custom generator}
The first difference is that it now uses our custom generator to determine the input values, instead of the built-in Java random generator. A list of tuples, containing values which satisfies the if-clause of the implication, are being generated. Our custom generator is a simple proof of concept in order to check if this will actually result in more failing tests. This custom generator basically consists of multiple methods which are being called based on the event name. In \autoref{lst:ch4_second_generating_values} this behavior is shown for the \textit{Symmetric} and \textit{Division1} event. The \textit{String} parameter of these methods is a way how we can pattern match on the event name in \textit{Rascal}. In case the event couldn't be handled, an exception is thrown.
% Listing
\FloatBarrier
\begin{sourcecode}[!ht]
\begin{lstlisting}[language=Rascal]
private list[Expr] genTestValueForEvent("Symmetric") {
    Expr moneyValue = genRandomMoney();
    return [moneyValue, moneyValue];
}
private list[Expr] genTestValueForEvent("Division1") {
    real moneyAmountX = genRandomDouble();
    real intAmountY = genRandomInteger();
    real moneyAmountZ = moneyAmountX * intAmountY;
    str currency = genRandomCurrency();
    return [convertToMoney(currency, moneyAmountX), converToExpr(intAmountY), convertToMoney(currency, moneyAmountZ)];
}
private default list[Expr] genTestValueForEvent(str eventName) {
    throw "genTestValueForEvent not implemented for event <eventName>";
}
\end{lstlisting}
\caption{Values generation for \textit{Symmetric} and \textit{Division1}, including the fall-back case.}
\label{lst:ch4_second_generating_values}
\end{sourcecode}
\FloatBarrier
% End listing
\pinfo{Not very dynamic}
This means that the way how we determine these values is basically hard-coded, requiring to have knowledge about the if-clause itself. Note that this doesn't make this approach very dynamic, but the result will consist of a list of tuples that satisfy the if-clause. These tuples will be used as input for the test case that will be generated.\\
\\
\pinfo{Mutating values with random operation}
However, the values that are generated now are fixed when we use them directly in a test case, which completely removes the randomness of the values when running the tests. It would be better to keep the randomness, such that the values are different on each run. To solve this problem, we mutate the values in the list such that the values are sort of random again. The tuples still have to satisfy the condition to trigger the if-clause, as this was the actual intention. So the second difference is that for each tuple in the list, we will generate a random operation and use that operation to mutate the values inside the tuple. To ensure that the tuple values still satisfy the condition of the if-clause, each value will be mutated by the same operation. In \autoref{lst:experiment2_second_resulting_test} an example of a generated test case is shown.
% Listing
\FloatBarrier
\begin{sourcecode}[!ht]
\begin{lstlisting}[language=Scala]
"work with Antisymmetry" in {
      Seq((USD(1593.62), USD(1593.62)), (USD(2869.78), USD(2869.78)),
          (EUR(4676.80), EUR(4676.80)), (USD(1850.29), USD(1850.29)),
          // ... // More values in the list
          (USD(9501.16), USD(9501.16)), (- EUR(149.67), - EUR(149.67)),
          (- EUR(159.67), - EUR(159.67)), (EUR(8015.77), EUR(8015.77)))
      .foreach {
        data:  (Money, Money) => {
          val randomOperation = genRandomOperation(genRandomOperator("Money", true), generateRandomMoney(data._1.currency), generateRandomInteger(true), generateRandomInteger(false), generateRandomPercentage(true), generateRandomPercentage(false), Random.nextInt(10))

          checkAction(Symmetry(
              randomOperation(data._1),
              randomOperation(data._2)
              )
          )
        }
      }
    }
\end{lstlisting}
\caption{Resulting test case with semi-random values. Omitted some input tuples for readability.}
\label{lst:experiment2_second_resulting_test}
\end{sourcecode}
\FloatBarrier
% End listing
\pinfo{Small explanation about the new test case}
The list of values are generated by using our custom generator, the amount of tuples in the list can be defined when generating the test suit. A method \code{genRandomOperation()} has been added to the template, which is used to mutate the fixed values in the list. After all the \code{checkAction()} method is being called to check the result of the test.\\
\\
\pinfo{Also: else now returns false}
Now that the input values for the implication events should always satisfy the condition of the if-clause, we can also update the specification such that the else-clause of the expression always returns \textit{False}. This results in a failing case again in case the precondition was not met. When this happens, it could indicate that there's a problem with either our custom generator, or in the generator.

% % % % % % % % % % % % % % % % % % % % % % % % % % % % % % % % % % % % %
% Section: Results
\section{Results}
\pinfo{Failing tests: Division}
Running the test suit with these changes results in 2 additional failing tests compared to the first experiment (\autoref{cpt:experiment1}). An overview of the failing properties and the used input values are shown in \autoref{tbl:experiment2_overview_first_run}. The log of the test run reports that the precondition was not met when using these input values, as shown in \autoref{lst:experiment2_log_first_run}.
% Table
\FloatBarrier
\begin{table}[!ht]
\centering
\begin{tabular}{llll}
\hline
\textbf{Property name} & \textbf{x}    & \textbf{y} & \textbf{z} \\ \hline
Division1              & -16729.90 USD & 830        & -20.16     \\
Division2              & -44.68 USD    & 870        & -38870.47  \\ \hline
\end{tabular}
\caption{Failing tests overview along with its input values}
\label{tbl:experiment2_overview_first_run}
\end{table}
\FloatBarrier
% End table

% Listing
\FloatBarrier
\begin{sourcecode}[!ht]
\begin{lstlisting}[language=Log]
[info] MoneyConditionals
[info] - should work with Additive4params (7 seconds, 224 milliseconds)
[info] - should work with AntisymmetryLET (5 seconds, 493 milliseconds)
[info] - should work with Symmetric (5 seconds, 344 milliseconds)
[info] - should work with Division2 *** FAILED *** (23 milliseconds)
[info]   java.lang.AssertionError: assertion failed: expected CommandSuccess(Division2(-16729.90 USD,830,-20.16 USD)), found CommandFailed(NonEmptyList(PreConditionFailed(x == z*y)))
[info] - should work with Division1 *** FAILED *** (127 milliseconds)
[info]   java.lang.AssertionError: assertion failed: expected CommandSuccess(Division1(-44.68 USD,870,-38870.47 USD)), found CommandFailed(NonEmptyList(PreConditionFailed(x*y == z)))
// ...
\end{lstlisting}
\caption{Precondition failed error in \textit{Division1} and \textit{Division2}.}
\label{lst:experiment2_log_first_run}
\end{sourcecode}
\FloatBarrier
% End listing

% % % % % % % % % % % % % % % % % % % % % % % % % % % % % % % % % % % % %
% Section: Analysis
\section{Analysis}
The values used in the test case should be correct, since we generated these values such that they satisfy the condition of the if-clause and thus they should satisfy the preconditions. Note that the conditions of the if-clause were added as preconditions in the \textit{MoneyConditionals} specification, which causes the error. As the \textit{PreConditionFailed} error is thrown by the system when the input values do not satisfy the preconditions.\\
\\
\pinfo{Describe precision error happening at first}
For \textit{Division1} it states that the condition \code{x*y == z} failed. The
values used for \textit{x}, \textit{y} and \textit{z} were \textit{-44.68 USD},
\textit{870} and \textit{-38870.47 USD} respectively. The result of
\textit{x * y} = \textit{-44.68 USD * 870} = \textit{-38871.60 USD}. This should
be equal to \textit{z}, in fact, the input of \textit{z} was slightly different,
\textit{-38870.47 USD}.\\
\\
Remember that the input values are being mutated by a
random operation that we have added to the test cases. This difference is caused
by the precision error when operating with the \textit{Money} type, which was
found in \autoref{cpt:experiment1}. The random operation that was done was
causing this behaviour. The same goes for the error with \textit{Division2},
where \code{x == z*y} should hold. The values of \textit{x}, \textit{y} and
\textit{z} are \textit{-16729.90 USD}, \textit{830}, \textit{-20.16 USD}
respectively. The result of \textit{z * y} = \textit{-16732.80 USD}, which is
not equal to \textit{-16729.90} USD.\\
\\
\pinfo{Next (when fixed precision), division problem}
The first experiment already described the precision problem and how it could be fixed. To solve this problem, we modify the generator such that the precision error is fixed when generating the system. Then the test framework is being executed again to check whether both tests are succeeding. This resulted in the same amount of tests that were failing, which means that we found a different case now. In \autoref{tbl:experiment2_overview_second_run} an overview of the used input values are shown\footnote{The decimals have been truncated for readability, \autoref{lst:experiment2_log_second_run} shows the exact values}. The log reported that, one case still fails on the precondition check, while the other case just reports values for which the result is \textit{false}, as shown in \autoref{lst:experiment2_log_second_run}.
% Table
\FloatBarrier
\begin{table}[!ht]
\centering
\begin{tabular}{llll}
\hline
\textbf{Property name} & \textbf{x}                               & \textbf{y} & \textbf{z}                               \\ \hline
% Division1              & 1.504347826086956521739130434782609 USD  & -779       & -1171.886956521739130434782608695652 USD \\
Division1              & 1.5043478260... USD     & -779       & -1171.8869565217... USD \\
% Division2              & -3328.825454545454545454545454545455 USD & -129       & 25.80484848484848484848484848484848 USD  \\ \hline
Division2              & -3328.8254545454... USD & -129       & 25.8048484848... USD    \\ \hline
\end{tabular}
\caption{Failing tests overview, after fixing precision errors}
\label{tbl:experiment2_overview_second_run}
\end{table}
\FloatBarrier
% End table

% Listing
\FloatBarrier
\begin{sourcecode}[!ht]
\begin{lstlisting}[language=Log]
[info] MoneyConditionalsSpec:
[info] MoneyConditionals
[info] - should work with Additive4params (7 seconds, 24 milliseconds)
[info] - should work with AntisymmetryLET (3 seconds, 66 milliseconds)
[info] - should work with Symmetric (4 seconds, 361 milliseconds)
[info] - should work with Division2 *** FAILED *** (670 milliseconds)
[info]   java.lang.AssertionError: assertion failed: expected CommandSuccess(Division2(-3328.825454545454545454545454545455 USD,-129,25.80484848484848484848484848484848 USD)), found CommandFailed(NonEmptyList(PreConditionFailed(x == z*y)))
[info] - should work with Division1 *** FAILED *** (316 milliseconds)
[info]   java.lang.AssertionError: assertion failed: expected CurrentState(Result,Initialised(Data(None,Some(true)))), found CurrentState(Result,Initialised(Data(None,Some(false)))): With command: Division1(1.504347826086956521739130434782609 USD,-779,-1171.886956521739130434782608695652 USD)
// ...
\end{lstlisting}
\caption{Precondition failed error in \textit{Division1} and \textit{Division2}.}
\label{lst:experiment2_log_second_run}
\end{sourcecode}
\FloatBarrier
% End listing
\pinfo{Division2 triggers division problem}
The test concerning \textit{Division2} shows that the precondition check fails. If we look at the input values, it can be seen that the \textit{Money} values are a fractional number. As it contains many decimals and it rounds up at the end. When operating with this rounded value, the resulting value is also slightly different. As the generated system is implemented such that the preconditions are being checked first, the \textit{PreConditionFailed} exception is thrown. This leads to the issue of the division problem \todo{Source?} in which a number cannot be equally divided. In \autoref{cpt:properties}, we defined that the precision in this case should be of X \todo{Define X} decimals. However, the test framework does not specifically check for this precision yet.\\
\\
\pinfo{Division1 triggers difference in rounding problem}
% Division1 is defined as \code{x*y == z $\implies$ x == z/y}.
When looking at \textit{Division1}, we see another case as the input values passed the precondition checks. This indicates that the values satisfy the condition to trigger the if-clause of the property. However, the result of the if-clause returns \textit{false}, showing us that the property does not hold when using these input values. Thus a case has been found for which the \textit{Division1} property doesn't hold. The investigation of the intermediate calculation steps are shown in TX. Note that the \textit{Division1} property is defined as \code{x*y == z $\implies$ x == z/y}.

% Table
\FloatBarrier
\begin{table}[!ht]
\centering
\begin{tabular}{rll}
\hline
\textbf{Variable}  & \textbf{Value}                                    & \textbf{Type}                                        \\ \hline
X                  & 1.504347826086956521739130434782609 USD           & Money                                                \\
Y                  & -779                                              & Integer                                              \\
Z                  & -1171.886956521739130434782608695652 USD          & Money                                                \\ \hline
\textbf{Formula}   & \textbf{Scala result}                             & \textbf{Expected result}                             \\ \hline
x*y == z           & true                                              & false                                                \\
x == z/y           & false                                             & false                                                \\
                   &                                                   &                                                      \\
x*y                & -1171.886956521739130434782608695652 USD          & -1171.886956521739130434782608695652\textbf{411} USD \\
z/y                & 1.504347826086956521739130434782608 USD           & 1.504347826086956521739130434782608 USD              \\ \hline
\end{tabular}
\caption{Division1: Difference in rounding}
\label{tbl:experiment2_division1_rounding_difference}
\end{table}
\FloatBarrier
% End table

In the results we can see that the expected values do not match the property either. Although in \textit{Scala} the first expression is considered true. Since the expected results also return false for the intermediate calculations, the input values might not fully satisfy the condition to trigger the if-clause. Which could be an implementation error in our calues generator. However, its notable that in \textit{Scala} the condition is considered to hold, which triggered this case. This indicates that there is a rounding error happening in the system, which triggered this case.\\
\\
Unfortunately, we are unable to trace back how the input values used for this tests were exactly determined. As these are build up by using random generated values and then mutating these by a random operation. Nevertheless, the results show that there is also an unexpected rounding going on when executing \textit{x*y} in \textit{Scala}. As the expected value contains some additional decimals compared to the result from \textit{Scala}.

% % % % % % % % % % % % % % % % % % % % % % % % % % % % % % % % % % % % %
% Section: Evaluation criteria
\section{Evaluation criteria}
\pinfo{Property coverage}
When looking at the coverage report concerning a specific property, it can be seen that the else-clause of the implication is not being triggered any more. In \autoref{fig:experiment2_eval_e2_highlighting_transitive-equality} the coverage of \textit{TransitiveEquality} is shown, note that only the else condition (which was translated to \textit{false}) is not triggered by the test suite. This was also the intention of the modification used in this experiment, as the if-clause is actually what we wanted to check in this experiment.
% Figure
\FloatBarrier
\begin{figure}[!ht]
%\frame{
	\includegraphics[width=\linewidth]{figures/eval_e2_transitive-equality}
%}
\caption{Test coverage for \textit{TransitiveEquality} in second experiment}
\label{fig:experiment2_eval_e2_highlighting_transitive-equality}
\centering
\end{figure}
\FloatBarrier
% End figure

% Evaluation criteria
\pinfo{74\% on conditionals. Others remain the same - image}
The expectation was that the test suite could be improved, such that the test coverage on the SUT would become higher. In the first experiment we found that the properties using implication were not tested thoroughly. In this experiment these properties are triggering the if-clause of the properties using implication, thus we expect the test coverage to be higher when looking on the coverage of the properties. This is also follows from the results, as we can see in \autoref{fig:experiment2_eval_e2}, the test coverage when looking at the logic file of the implicative properties is 74\%.
% Figure
\FloatBarrier
\begin{figure}[!ht]
%\frame{
	\includegraphics[width=\linewidth]{figures/eval_experiment2}
%}
\caption{Test coverage report of the first experiment}
\label{fig:experiment2_eval_e2}
\centering
\end{figure}
\FloatBarrier
% End figure
\pinfo{Coverage, 85\% (overall), better}
The total test coverage on the SUT is reported to be 85\%. As we have discussed in the evaluation of the first experiment, we do not expect to reach the 100\% coverage, as there are certain components in the generated system which we do not test with this approach. Also, considering that the else-clause is not being triggered of the implicative properties, the test coverage will never become 100\%. Which is not a problem in that sense, as we don't intent to test the else clause, we are more interested in the result of the if-clause of these implicative properties.\\
\\
\pinfo{\# of bugs, 2 more}
The other criteria we use was the amount of bugs that we have found. Using this approach 2 more tests were failing compared to the first experiment in \autoref{cpt:experiment1}. The bug found was when using division with the \textit{Money} type, which is expected to be rounded off at the Xth decimal \todo{Define X}. However, as we have seen in this experiment, the generated system does not take this into account. We can categorize this bug into one category, rounding errors. These are different from precision errors as the precision errors are caused by having an incorrectly calculated value, where rounding errors are basically not working with the rounding method that is defined.

% % % % % % % % % % % % % % % % % % % % % % % % % % % % % % % % % % % % %
% Section: Conclusion
\section{Conclusion}
In this experiment we generated the input values such that the condition of the properties using implication are satisfied. This revealed 2 additional failing cases. One is related to the division problem and one is indicating a rounding error. Although the latter one might also be caused because of incorrectly generated values. The division problem occurs when there is no even division of a number. For example: when dividing 1 by 3. The generated system tries to hold the exact value, which triggers this situation. It is reasonable that the system tries to hold the exact value, however, the rounding method is not taken into account here. Furthermore, it is not clear from the specification how this rounding should be done.\\
\\
When defining the properties in \autoref{cpt:properties}, we said that a value should be precise. When rounding is needed, it must have X decimals \todo{Define x}. However, the test framework does not yet take this rule into account when testing these properties.

% % % % % % % % % % % % % % % % % % % % % % % % % % % % % % % % % % % % %
% Section: Threats to validity
\section{Threats to validity}

% Incorrect value generation
\subsection*{Incorrect value generation}
We have implemented a custom value generator to generate values for each test case. Furthermore, a random operation is being done on these values to make these random again. There could be an error in the implementation that incorrect values are being created, which are expected to be correct. When this is the case, the traceability of how the values were created is hard. This might affect the results or make some errors hard to trace back. We have seen this in this experiment.

% Detecting precision
\subsection*{Detecting precision}
In this chapter we triggered the division problem and found that this can cause problems in the generator. The value generation could be modified such that this case isn't being triggered anymore. But on the other hand, its important to know what the expected result would be in this case. A possibility is to define this on the Rebel language, or to make such rounding and precision rules part of the specification. Currently it is unclear what should happen in this situation. We concluded that the generator doesn't take the rules on precision and rounding, that we have defined in \autoref{cpt:properties}, are not taken into account. And that it currently uses a lower precision. Such a high precision is perhaps not expected for Rebel specifications, leaving it as a threat for this approach.
% Not testing for X decimals precision currently. We could generate values such that this isnt being triggered. However, maybe we want to do so. Or maybe its better to define the allocation or precision in Rebel, such that it is part of the specification. Currently this is unclear what should happen with division.

% Dynamicallity
\subsection*{Dynamicallity}
\todo{Not sure if 'dynamicallity' is a correct word for this}
\pinfo{Not very dynamic}
The implicative properties are now being tested such that the condition of the if-clause is being satisfied. However, the values generator that is being used for this is not very dynamic. As it simply checks for the event name and throws an exception in case this is not defined for the property yet. This means that adding new property definitions to the test framework, requires a modification to the value generator in case of an implicative property. This makes the test framework less dynamic when adding new properties that should be tested on the generator.\\
\\
\pinfo{Fix: using the preconditions}
To fix this it would be better to generate the values by interpreting the preconditions such that random values can be determined based on a certain condition. Since the \textit{Rebel} toolchain already makes use of a bounded model checker to check a specification, this could be used to simply translate an expression and retrieve values for which the condition holds.\\
\\
\pinfo{Checked using Z3, but returns same values all the time}
We have looked into this, by using the Z3 solver. However, the solver always returns the same number when executing it multiple times. Which means that the 100 values that we would ask from the generator, will be exactly the same. A workaround would be to then add the number that was received earlier as another constraint, such that 100 unique values are being retrieved. But the problem still remains, as executing the same script multiple times results in the same values. When changing the seed of the random generator that is being used, it will return different values. In order to make the test framework execute this behaviour, the value generator has to be changed to integrate with the solver. Additionally, this could have a huge effect on time increase that the test framework needs to succesfully finish.\\
\\
\pinfo{Other possibilities, future work}
There are other solvers available too, or other methods to generate values that match the condition. It would be usefull to make the test framework more dynamic when such properties are being used. However, this is left as future work.

% Implicative properties
\subsection*{Implicative properties effectiveness}
\pinfo{Uneffective? Checking more though}
The use of implicative properties might not be as effective as using properties that do not. If the properties could be rewritten such that random values could be used to check the same thing, the implicative properties might be unneccessary. On the other hand, more functionalities from the generator are being used, and thus being tested, by this approach. Which wouldn't be the case when the implicative properties are being removed. If-statements and preconditions were not being used in the first experiment.



% Not extra coverage maybe? If we evaluate that different than earlier.

% % % % % % % % % % % % % % % % % % % % % % % % % % % % % % % % % % % % %
% Chapter: Experiment 3
% % % % % % % % % % % % % % % % % % % % % % % % % % % % % % % % % % % % %
\chapter{Experiment 3: Improving the value generation}
\label{cpt:experiment3}
In the second experiment (\autoref{cpt:experiment2}) it was found that the
property definition for using division with the \textit{Monye} type wasn't clear
enough in that it couldn't be satisfied the way how it was specified. In this
experiment we aim to improve this. We update the existing definition such that
it can now be correctly.\\
\\
Another result from the second experiment was that the value generation was not
very dynamic when new properties were being added. It should be possible to add
additional properties to the specification, which are then being tested
automatically by the test framework. To do this, the value generator will be
updated too, such that it uses the preconditions in the event definitions to
determine the parameter values. Additional properties can then be added to test
the generator.\\
\\
In this experiment we will use an updated version of the generator. In this
version the precision errors that were found in the first experiment are fixed,
such that we can focus on possible additional errors that might occur.

% % % % % % % % % % % % % % % % % % % % % % % % % % % % % % % % % % % % %
% Section: Method
\section{Method}
The property definition of division will be updated for this experiment, because
the first definition did not take the division problem into account. There was
also no definition for rounding the value, as the value was expected to hold the
exact value. We will update the definition by using a round function. By using
this, the fixed properties can be defined which can then be tested by the test
framework. These are fixed because these property definitions turned out to be
incomplete or incorrect in the second experiment (\autoref{cpt:experiment2}).\\
\\
Besides updating the property definitions that were using division, additional
property definitions are being added to test more of the generator. However, as
we have seen in the second experiment, the values generator was not very
dynamic. Which resulted in that it had to be modified in case an implicative
property is being added to the specification. This is the second thing that
should be improved, such that additional (implicative) properties do not require
modifications to the values generator.

% % % % % % % % % % % % %
% Subsection: Updating property definitions
\subsection{Updating property definitions}
\todo{Source for division - rounding?}
\todo{Rephrase}
\pinfo{Round method}
Initially there were only 2 property definitions that were using division, which both are being updated. In order to check for the values such that the property definition can be used, we implement a \code{round()} method in the \textit{Rebel} specification. This \textit{round} method is intended to return a value which can be used to define the expected behaviour when using division with the \textit{Money} type.\\
\\
\pinfo{Defining, updating and adding more}
In addition to updating the existing properties that are using division, more properties can be added. For example, properties of inequality when using division, as the ones that were defined for
division only used equality. Also other properties like the subtraction property can
be added and more definitions for multiplication and additivity
can be added to the existing list of property definitions that we defined for \textit{Rebel}. The
additional property definitions for division, multiplication, additivity and
subtraction fall under the ``Properties of equality and inequality'' category
and are defined in \autoref{ssct:properties_definitions_additionalproperties},
along with the updated definitions for the existing property definitions that
use division. \todo{Add added properties by name as list}\\
\\
\pinfo{Round method implementation + why}
The properties using division are now using the \code{round()} method in its d
efinition. \textit{Rebel} does not provide a way to round a value, which is why
we need to define the function in the specification. In \textit{Rebel} a
function is defined as an expression that is being executed whenever the
function is being called. Unfortunately, there is no way in \textit{Rebel} to
define the \textit{Scala} implementation of this \textit{round} function. As a
workaround, we define the implementation as a \textit{String} and modify the
generator such that the function's implementation is the content of the
\textit{String} (removing the quotes). The \textit{round} method rounds the
\textit{Money} value to a maximum of 4 decimals. The fifth decimal is being
rounded by using ``HALF_UP'' technique
\todo{Maybe a source or so? Or common knowledge?}. The function implementation
in the \textit{Rebel} specification is shown in
\autoref{lst:experiment3_rebel_round_implementation}.
% Listing
\FloatBarrier
\begin{sourcecode}[!ht]
\begin{lstlisting}[language=Rebel]
function round(money: Money): Money =
    "money.currency(money.amount.setScale(4, RoundingMode.HALF_UP))";
\end{lstlisting}
\caption{The updated event definition of the \textit{Symmetric} property}
\label{lst:experiment3_rebel_round_implementation}
\end{sourcecode}
\FloatBarrier
% End listing

% % % % % % % % % % % % %
% Subsection: Improving dynamicallity
\subsection{Improving dynamicallity}
\pinfo{Using preconditions}
The additional properties that are being added also using implication in their definitions. In the second experiment (\autoref{cpt:experiment2}), the value generator was not dynamic enough in that it requires modifications for each implicative property that is being added. In this experiment we aim to improve this, by using the defined preconditions to determine the input values.\\
\\
A custom generator is being created that uses the preconditions to determine the input values. Note that this can be seen as an update to the generator that was created in \autoref{cpt:experiment2}. This value generator parses the preconditions and intends to generate values based on these conditions. Since the expressions inside the preconditions might become quite complex, we focus on a limited version of it, while still satisfying the requirements that are needed for the properties that have been defined in \autoref{cpt:properties}.\\
\\
Most properties are using single variables in its precondition statements. Some are using expressions on the left-hand or right-hand side of a statement, but not on both sides. The value generator that we implement will not support using expressions on both the left-hand and right-hand side, as generating values matching the condition can become complex. Instead, it requires to only have a variable on one side and either an expression or a variable on the other side.\\
\\
The code to generate the tuples of input values is shown in \autoref{lst:experiment3_value_generation_code}. We can separate this process into the following steps:
\def \valueGeneratorStepOne{Initialize value generation data}
\def \valueGeneratorStepTwo{Traverse and handle statements}
\def \valueGeneratorStepThree{Generate values for yet unassigned variables}
\def \valueGeneratorStepFour{Add values to resulting list}
\begin{enumerate}
  \item \valueGeneratorStepOne
  \item \valueGeneratorStepTwo
  \item \valueGeneratorStepThree
  \item \valueGeneratorStepFour
\end{enumerate}
% Listing
\FloatBarrier
\begin{sourcecode}[!ht]
\begin{lstlisting}[language=Rascal]
public list[list[Expr]] genValues(str eventName, Preconditions? preconditions, list[Parameter] transitionParams, int amount) {
    println("\> Generating values for event <eventName>");

    list[list[Expr]] valueList = [];
    for (int i <- [0..amount]) {
        calculatedParams = (); // Clear old data
        paramGenData = ("<p.name>" : <p.tipe, -9999.00, 9999.00, true> | p <- transitionParams);

        // Calculating
        for(/Statement s <- preconditions) {
            <lhs, rhs, operator> = extractStatementData(s.expr);
            handleConditionStatement(operator, lhs, rhs);
        }

        // Check whether all are determined, if not, determine those using the randomValueProps data
        for (Parameter p <- transitionParams, !calculatedParams["<p.name>"]?) {
            calculatedParams["<p.name>"] = calculateExpression(p.name);
        }

        // Add to list
        valueList += [[getExprForVar("<p.name>") | p <- transitionParams]];
    }
    return valueList;
}
\end{lstlisting}
\caption{The updated event definition of the \textit{Symmetric} property}
\label{lst:experiment3_value_generation_code}
\end{sourcecode}
\FloatBarrier
% End listing

In the following sections we describe each step in detail.

% Initialize value generation data for each variable
\subsubsection{1. \valueGeneratorStepOne}
To generate a single value, some data is being hold to keep track of the conditions to which a certain value should when it is being generated. These conditions are minimum value, maximum value and whether the zero value is allowed. Additionally the result type of the variable is stored, used when generating the final value. This data is stored in a tuple and called \textit{RandomValueProps} by using an \textit{alias} in \textit{Rascal}. This definition is shown in \autoref{lst:experiment3_alias_definition_randomvalueprops}.
% Listing
\FloatBarrier
\begin{sourcecode}[!ht]
\begin{lstlisting}[language=Rascal]
// Minimum: including. So: if 0, then 0 can be a result value when determining it random.
// Maximum: including. So: if 10, then 10.00 is max result value when determining random.
alias RandomValueProps = tuple[Type tipe, real min, real max, bool allowZero];
\end{lstlisting}
\caption{The updated event definition of the \textit{Symmetric} property}
\label{lst:experiment3_alias_definition_randomvalueprops}
\end{sourcecode}
\FloatBarrier
% End listing
The value generator initializes the \textit{RandomValueProps} for each input variable (\autoref{lst:experiment3_value_generation_code},~Line~7). Setting the minimum and maximum value to a default value and allowing zero by default.

% Traverse statements in the precondition block and handle these based on the expression
\subsubsection{2. \valueGeneratorStepTwo}
Each statement in the preconditions block is being checked (\autoref{lst:experiment3_value_generation_code},~Line~9 -13). The \textit{handleStatement()} method handles each statement. The actions done by this method depend on the operators used in the statement that is being handled.\\
\\
In case of an expression that only contains variables and uses equality (for example, \textit{x == y}), the value generator assigns a random value to \textit{x} and assigns the same value to \textit{y}. In case of inequality (\textit{x > y}), the value generator also assigns a random value to \textit{x}, but adjusts the minimum or maximum bounds of the \textit{y} value such that it satisfies the condition.\\
\\
In case there is an expression on one side (for example, \textit{x * y == z}), the expression will be evaluated first. In this case, random values will be assigned to \textit{x} and \textit{y}. Next, the expression can be evaluated and the result of that is being assigned to \textit{z}. The same is done with inequality relations, but then the minimum or maximum bounds are being set based on the operator.\\
\\
Having expressions on both the left-hand and the right-hand side of the expression is unsupported. The \textit{handleStatement()} method will throw an error in case this happens.

% Generate values for yet unassigned variables
\subsubsection{3. \valueGeneratorStepThree}
When handling each expression, some variables already get an assigned value. However, some variables might only have their \textit{RandomValueProps} updated but do not have an assigned value yet. In this step these variables, that do not have a value yet, are being assigned a value based on their \textit{RandomValueProps} (\autoref{lst:experiment3_value_generation_code},~Line~15-18).

% Add values to resulting list
\subsubsection{4. \valueGeneratorStepFour}
The values that have been determined are being added to the list of generated input values (\autoref{lst:experiment3_value_generation_code},~Line~21). In the end the list is being returned, containing all the generated input values that match the preconditions of the event.

% % % % % % % % % % % % % % % % % % % % % % % % % % % % % % % % % % % % %
% Section: Results
\section{Results}
% - Run, tests succeed (hopefully)
Running the test framework with the addition of the additional properties (\autoref{ssct:properties_definitions_additionalproperties}), results in no additional failing tests compared to the first experiment. The log is shown in \_\_\_. Remember that the test framework is run with the generator in which the precision problems related to the \textit{Money} type are fixed (otherwise there would be failing tests, due to the precision problems).
\todo{Add log}
\\
\\

% % % % % % % % % % % % % % % % % % % % % % % % % % % % % % % % % % % % %
% Section: Analysis
\section{Analysis}
% - Test succeeding provided rounding
% - Case of rounding down, it fails (bigger number shows difference)
% - > Thus probably division does this too usually. FLooring causes difference
The tests are succeeding due to the implementation of the \textit{round()} method. It rounds the value of \textit{Money} to 4 decimals using the ``HALF\_UP'' rounding mode. The number 4 is chosen here to ensure that the property definitions should hold up to a precision of 4 decimals when using the round method. However, this can be modified in case a bigger precision is expected. Changing the number to 6 or 8 decimals do not change the results when looking at the amount of tests failing \todo{Check: is this true?}. However, changing the rounding mode would affect the results. When using ``DOWN'' as rounding mode, the tests that use the \textit{round()} method are failing\todo{Check: All division tests or just 'some'?}. This can be seen in \_\_\_, where ``DOWN'' is being used as rounding mode.\\
\todo{Add log}
\\
We stay with he implementation of ``HALF\_UP'' as rounding mode, as this is more in line with our expectations. When using the ``DOWN'' rounding mode, the resulting value is sometimes off with a difference value of 0.001, which is caused by this rounding mode.
\\
% - Adding properties didn't require a change in the value generator anymore
For this experiment, additional property definitions have been added to test the generator. The value generator now determines the values based on the preconditions of the properties. Thus the additional properties used in this experiment did not require more effort than defining these in the \textit{Rebel} specification. Compared to the second experiment (\autoref{cpt:experiment2}) this is a huge improvement, as it does not require the developer to modify the value generator in case implicative properties are being added.

% % % % % % % % % % % % % % % % % % % % % % % % % % % % % % % % % % % % %
% Section: Evaluation criteria
\section{Evaluation criteria}
% - Same coverage
% - No extra bugs found
%   > but increased dynamicallity. We have shown that additional properties can now be added to check, without needing to update the test framework
...

% % % % % % % % % % % % % % % % % % % % % % % % % % % % % % % % % % % % %
% Section: Threats to validity
\section{Threats to validity}
% - Implementation error in our generator. However, we generate random values every run. (unlike Z3)
% > Running multiple times might give a fault-positive. High probability this is caused by the random operation that is done on the input values. This is a threat that came forth from experiment 2, and still exists here. - (Using > or < with 0 0 0 can happen because of this)

% - Existing approaches are available too, we only checked the Z3. But SageMath or others are possibilities too. However, this requires us to make such a tool compatible with the test framework. Our own implementation has shown to work for the properties we have defined. In case of more complex properties, it might be more useful to use another tool to determine these values.
...
% More reliable solving. Checked but not working as expected. Which is why we did our own generator
\subsubsection{Using existing solvers}
\pinfo{BMC possibility, tested. But not working}
Existing solvers could be used to determine a certain number of input values
that satisfy the preconditions. Since the \textit{Rebel} toolchain already makes
use of a bounded model checker to check a specification, this could be used to
translate an expression and retrieve values for which the condition holds.\\
\\
\pinfo{Checked using Z3, but returns same values all the time}
We have looked into this, by using the \textit{Z3} solver. However, the solver
always returns the same number when executing it multiple times. Which means
that the 100 values that we would ask from the generator, will be exactly the
same. A workaround would be to then add the number that was received earlier as
additional constraint, such that 100 unique values are being retrieved. But the
problem still remains, as executing the same script multiple times will result in
the same values. Another possibility would be to change the seed of the random generator that is being
used to generate the values, resulting in different values, however, then a random seed should be used each time the solver is being run to make the generated values unique. In order to integrate the solver with the test framework, the value generator has to be changed.\\
\\
An update to the value generator is required anyway to determine the input values based on the preconditions. In addition, some existing examples to check a \textit{Rebel} specification already take up some time. Which are probably related to the translations that have to be done and actually running the solver. We have not measured the exact duration of each step in this case, but we expect that generating 100 random input values by using this approach requires some time. Resulting in a huge run time increase for the test framework.\\
\\
Other solvers might work better for this approach, but these still require an update to the values generation part of the test framework to integrate with such solvers.


%
%\pinfo{Other possibilities, future work}
%There are other solvers available too, or other methods to generate values that
%match the condition. It would be useful to make the test framework more dynamic
%when such properties are being used.

% % % % % % % % % % % % % % % % % % % % % % % % % % % % % % % % % % % % %
% Chapter: Discussion
% % % % % % % % % % % % % % % % % % % % % % % % % % % % % % % % % % % % %
\chapter{Discussion}
\label{cpt:discussion}
In this chapter we discuss the research questions.

% % % % % % % % % % % % % % % % % % % % % % % % % % % % % % % % % % % % %
% RQ 1: Which properties
\section{RQ 1: \rqOne{}}
There were no existing properties defined for \textit{Rebel} that were expected
to hold. We have defined many properties for \textit{Rebel}. With the focus on
the \textit{Money} type, which is considered the most important type for a
bank.\\
\\
Many properties have been defined, but are these all the properties? Is each
definition correct? Some properties might be considered incorrect or overlapping
with others. We have seen this with the defintion of \textit{division} that was
incorrect initially, which was encountered in the second experiment. In the
third experiment the property definitions for \textit{division} have been
updated. Additionally, extra properties were added to the list for the third
experiment, \textit{additivity}, \textit{subtraction} and
\textit{multiplication}. This showed how additional properties could be added to
the test framework to test the generator. In case of incorrect, missing or
overlapping property definitions, the current set of property definitions could
be updated as we have done with the third experiment. There can be many
combinations among the different types of \textit{Rebel} and the supported
operators. Therefore, the set of properties that we have defined in
\autoref{cpt:properties} is not the complete set of expected properties in
\textit{Rebel}.\\
\\
Each property definition is defined as an expression, the test framework is
currently limited to checking the generator on these kind of properties. We can also
think of other properties that should hold in \textit{Rebel}. For example, the
behaviour of sync block: what are the properties of the sync block in
\textit{Rebel}? Can these be described in a \textit{Rebel} specification? If so,
the current test framework would not support that definition because it can
currently only cope with expression-based properties.

% % % % % % % % % % % % % % % % % % % % % % % % % % % % % % % % % % % % %
% RQ 2: How we test
\section{RQ 2: \rqTwo{}}
We have described a way how the generator can be tested by using property-based
testing. In order to check the generator, a \textit{Rebel} specification is
created containing the property definitions. This specification is then used by
the test framework to generate tests and run these tests against the generated
system.\\
\\
The initial version of the test framework used random values to test each property. The first experiment showed that this doesn't work for the implicative properties. The if-clause of these implicative properties were not being triggered when using random values. In the second experiment the test framework was improved. The input values where determined such that the condition of the implicative property was being satisfied. Additionally, these values were being mutated by a random operation, to keep the randomness of the input values. This improvement to the test framework revealed some incorrect property definitions. The implementation of how the input values were determined was not very dynamic. Because every time an implicative property was being added to the specification, an update was required to the test framework.\\
\\
In the third experiment the input values were determined in a more dynamic way, by using the preconditions to determine the input values. This requires a property definition to define its preconditions. As a result, it was not required to update the test framework when implicative properties were being added. This is shown in the third experiment.\\
\\
We evaluated the test framework in each experiment by looking at the test coverage and the number of bugs that have been detected. The coverage details showed that in the first experiment the if-clause of the implicative properties was not being triggered. Which lead to an improved version in the second and third experiment. Furthermore the branch coverage reported a consistent 50\% over the second and third experiment, which is caused by the implicative properties. The else-clause is not being tested, but that is also not the intention of that property.\\
\\
The overall test coverage for each experiment was roughly the same (87.80\%, 88.87\% and 88.28\%). This shows that despite of the improvements made to the test framework, the test coverage did not change. This is caused by the fact that each experiment is different:
\begin{itemize}
  \item The first experiment used random input values for each property. The else-clause of the implicative properties were defined as \code{true}, resulting that each implicative property was considered correct.
  \item In the second experiment the else-clause of the implicative properties were set to \code{false}. The input values are generated such that these satisfy the condition of the if-clause. These values are also being mutated when running the test, to keep the actual values random.
  \item In the third experiment some incorrectly defined properties were updated and additional properties have been added.
\end{itemize}
The test coverage of the first experiment would be considerably lower in case the else-clause of the implicative properties would be \code{false}. This would result in that every test concerning an implicative property would fail. Although the result of this would then be that there are many false-positives, because it just returns \textit{false} in its definition.\\
\\
The test coverage over the second and third experiment are roughly the same (only 0.59\% difference). But the third experiment tested 17 more properties than the second experiment. This shows that adding additional properties does not have much impact on the test coverage. The additional properties are being checked automatically by the test framework because these properties are defined in the \textit{Rebel} specification.

% % % % % % % % % % % % % % % % % % % % % % % % % % % % % % % % % % % % %
% RQ 3: Number and kind of bugs
\section{RQ 3: \rqThree{}}
Multiple bugs were found using property-based testing to check the generator.
The generator failed to satisfy a total of 9 properties that we have defined.
Some properties triggered different kind of bugs. The bugs that were found can
be separated into the following categories:
\begin{description}
  \item[~~~~Compilation errors:] Errors that make the generated system unable to compile, resulting that the generated system cannot be used.
  \item[~~~~Overflow/underflow errors:] Errors happening because of a limit that has been reached on specific types.
  \item[~~~~Precision errors:] Errors causing an unexpected outcome value when being calculated.
\end{description}

\subsection*{Compilation errors}
\pinfo{Compilation error, Squants issue}
In the second experiment, the test framework was initially being terminated because of a
compilation error. Although one assumption was that the generated system should
be able to compile, another assumption was that the specification was
consistent. The specification containing all the properties is consistent,
as \textit{Rebel} did not report any syntactic or semantic errors with the type
checker. The test framework is thus able to find such compilation errors. However,
there can be many more compilation errors for which we do not check, which is
also out of the scope of this thesis. The cause of this error was actually
caused by an implementation error in an open-source library that the generated
system used, called \textit{Squants}~\cite{siteSquants2017}. To fix this, we
created a \textit{Github}
issue\footnote{https://github.com/typelevel/squants/issues/281} describing the
problem. So that this can be fixed in the next release of the library.

\subsection*{Overflow/underflow errors}
\pinfo{Overflow/underflow errors, discussion and unclear definition}
The overflow/underflow errors are caused because of the use of the
\textit{Integer} type. On one hand, this could be prevented by checking the
operations beforehand for overflow errors. On the other hand, this could be the
expected behaviour when an \textit{Integer} is being used in \textit{Rebel}. As
\textit{Integers} are known have such limits that are also dependent on the
platform the application is run~\cite{wang2009intscope}. However, in
\textit{Rebel} there is currently no other type that can be used to hold a
bigger number. For example in \textit{Java} there is \textit{Long} for a larger
number, or \textit{BigDecimal} for even bigger numbers. This would mean that
\textit{Rebel} does not support big numbers, or that a custom type must be used
for this. Considering that \textit{Rebel} does not provide another type for
bigger numbers, the \textit{Integer} type is considered to also hold bigger
numbers. Since the specification is about banking products and it probably could
happen that a big number is needed. After all, we cannot know this for sure, as
\textit{Rebel} does not provide a specification yet of each of type in
\textit{Rebel}.

\subsection*{Precision errors}
\pinfo{Precision errors, Squants issue}
As we have seen, the \textit{Money} precision errors both occurred when using
\textit{Percentage} values as well as when using \textit{Integer} values to
operate with the \textit{Money} type. Since we were able to reproduce the issue
in a clean \textit{REPL} environment, the problem existed in the open-source
library, called \textit{Squants}, that was used for the \textit{Money} type. In
order to solve this problem, we created an issue on
\textit{Github}\footnote{https://github.com/typelevel/squants/issues/265}
related to the precision problems on the \textit{Money} type. A contributor
responded and fixed the issue within a day, the change will be included in the
next version of the library (1.4). So it is required to update this library in
order to let these tests in our test suite pass.

% Advice?
% To think of: Where there some things encountered during the project, which might require attention, but haven't or won't be thoroughly tested with this approach. Or do some additions cover these?
% > Squants
%	- Some bugs, maybe there are more in there
% 	- Snapshot release unavailable for some time? (1.4.0-SNAPSHOT was not reachable, had to compile it ourselves)

% % % % % % % % % % % % % % % % % % % % % % % % % % % % % % % % % % % % %
% Chapter: Conclusion
% % % % % % % % % % % % % % % % % % % % % % % % % % % % % % % % % % % % %
\chapter{Conclusion}
\label{chp:conclusion}
% Summary
In this thesis, we have shown a way how the generator can be tested by using property-based testing. This is done by generating tests based on the \textit{Rebel} specification and making use of the generator to generate the tests. The \textit{Rebel} specification is build up based on a set of defined properties of \textit{Rebel}. These property definitions were not defined earlier, thus we defined many expected properties in \textit{Rebel}.\\
\\
With test framework we have found some bugs in the generated system that were unknown before. This proves that this approach already worked to identify some problems in the generator that were not known before. Additionally, we contributed to an open-source library called \textit{Squants}, by issuing two reports of bugs that existed in the library.\\
\\
% Research questions
% Answer main research question
To answer the main research question, we defined and answered the three sub research questions, which have been discussed in \autoref{cpt:discussion}. The main research question was as follows:
\begin{quote}
\rqMain
\end{quote}
A \textit{Rebel} specification is created from the defined properties. Next, the existing generator is being used to generate the generated system and to translate the properties (in the \textit{Rebel} specification) to test cases. When running the test framework, we found some errors by using the generated system. Additionally, we were able to detect a compilation error and incorrectly translated formulas.\\
\\
We conclude that, by using this approach, the generator can be checked on whether it satisfies the defined properties. The generated system is used to determine the result.

% % % % % % % % % % % % % % % % % % % % % % % % % % % % % % % % % % % % %
% Section: Future work
\section{Future work}
% Multi VM testing, specify sync properties, can test with multi JVM testing etc. Sources > future work

\subsection*{Complete property definitions}
\pinfo{Full definitions for Rebel}
In this thesis, we defined some properties on the \textit{Rebel} language,
which we consider to hold when using \textit{Rebel}. This list of property
definitions on \textit{Rebel} is not complete, there can be many more properties
and there are other types available in \textit{Rebel} which we did not cover.
These should be added to have a complete set of property definitions for the
\textit{Rebel} language. This is left as future work. When additional properties
are known, these properties can be added to the test framework such that these
will also be tested on the generator.

\subsection*{Values generation by using solvers}
\pinfo{Current value generation limitations}
The current value generation works for the properties that we defined throughout this thesis. When additional properties are being added, the value generator might or might not have to be updated. This depends on what preconditions have to be parsed for that property and whether the current implementation supports these operations. As described, the current implementation has some limitations. It does not support expressions that are using different operators or that use expressions on both the left-hand and right-hand side (expressions here means non-literals or variables, but a combination of operators and literals/variables).\\
\\
\pinfo{Other tools possible}
Another way how the value generator could work is to use some tools to determine the random values. As we have discussed in \autoref{cpt:experiment3}, the SMT solver would theoratically be an option. But in our attempts, this resulted in having the same input values every time the test framework is run, meaning that the randomness of the values is lost in the first part. To counter this problem, other solvers might be more useful to implement this behaviour, such as \textit{SageMath}~\cite{siteSageMath2017}. \textit{SageMath} can parse expressions and determine the conditions to which each variable in the expression should hold. When using this, it would mean that the value generator has to be modified such that it integrates with this tool. Another approach could be used for this too, such as concolic testing~\cite{sen2006cute} to determine the input values randomly. % Maybe there are some QuickCheck ports that do this already for some languages?

\subsection*{Multiple generators}
\pinfo{Only one, support for more could be added}
Throughout this thesis, we have only tested the Scala/Akka generator which is
developed by \textit{ING}. Since there are more generators available, the test
framework can be improved such that it is
compatible with the other systems that can be generated by using one of the
other generators that are available within \textit{ING}. By doing this, the same
property definitions can be tested on different kind of generated systems. This
can be used to detect inequalities among the generated system.

\subsection*{Mutation testing}
A way to evaluate the effectiveness of the test framework would be to use
mutation testing. The mutation coverage could be used to measure how
effective the test framework would be (the number of mutants created and the
number of killed mutants). Unfortunately, there is currently a limited support
for mutation testing for \textit{Scala} systems, as it cannot meaningfully
mutate \textit{Scala} code~\cite{siteSbtPit2017}.\\
\\
\textit{Scala} compiles to \textit{Java} bytecode, mutating on bytecode level
could be used to solve the problem of limited support. However, this could lead
to false-positives, meaning that some mutants might not be relevant in that the
modifications would not affect the implementation. This should then be taken
into account.

% Ultimately, as last
\subsection*{Properties based on type checker}
\pinfo{Generating properties based on type checker}
\textit{Rebel} contains a type checker, which is able to determine exactly
which operations are supported by using certain combinations of operators and
types. This could be used to automatically generate the property specifications
in \textit{Rebel} based on whether a type supports certain operations that are
required for the property. For example, if \textit{Percentage} supports
addition, the properties for \textit{additivity}, \textit{associativeAddition}
and \textit{commutativeAddition} can be generated based on this rule.\\
\\
Considering that a map
of all operations can be created and a map of all existing types, each
combination can be tested against the type checker. If the type checker an
expression as correct, the event definition for the property can
be generated. Although this might result in many definitions that are being
tested, with a possible overlap between them, it would be a way to automatically
generate the properties that should be tested. This would assume that the type
checker does the right thing, which would be a threat for this approach. This
might also cause more compilation errors in the generated system. Nevertheless,
it can make it more dynamic when a new type could be added to \textit{Rebel}.

\subsection*{Other components}
The current setup of the test framework is intended to be used to test a certain
component of the generated system. Namely, the \textit{Rebel} types and the
operations among these. More properties can be added to check every type that
\textit{Rebel} supports. The test framework could also be extended such that it
can also test other components of the system. Such as the sync block
definitions, defining actions that should happen synchronously. Or performance
measures when interacting with the data in the system, which uses a database
implementation. Currently, such components are not being tested, while these
components are also important for the bank. This is left as future work.\\
\\
Although the test framework can be extended to also test other components of
the generated system, it might not be possible to check each and every component
by using this approach. For example, it is probably not able to check how
multiple generated systems would integrate with each other and if this is done
correctly. The generated system is configured to work with multiple
distributions of the system, but this is not being tested yet.

% % % % % % % % % % % % % % % % % % % % % % % % % % % % % % % % % % % % %
% Chapter: Bibliography
% % % % % % % % % % % % % % % % % % % % % % % % % % % % % % % % % % % % %
%{%\tiny
%\bibliographystyle{alphaurl}
\bibliographystyle{ieeetr}
\bibliography{thesis}
%}

% % % % % % % % % % % % % % % % % % % % % % % % % % % % % % % % % % % % %
% Chapter: Appendices
% % % % % % % % % % % % % % % % % % % % % % % % % % % % % % % % % % % % %
\appendix
% % % % % % % % % % % % % % % % % % % % % % % % % % % % % % % % % % % % %
% Section: Appendix I: Property definitions of Rebel in Rebel
\chapter{Property definitions in Rebel}
\label{app:a_event_definitions}
% Listing
\begin{sourcecode}[!ht]
\begin{lstlisting}[language=Rebel]
event reflexiveEquality(x: Money) {
    postconditions {
       new this.result == ( x == x );
    }
}

event reflexiveInequalityLET(x: Money) {
    postconditions {
       new this.result == ( x <= x );
    }
}

event reflexiveInequalityGET(x: Money) {
    postconditions {
       new this.result == ( x >= x );
    }
}

event symmetric(x: Money, y:Money) {
    postconditions {
       new this.result == ( (x == y) ? y == x : True );
    }
}

event transitiveEquality(x: Money, y: Money, z: Money) {
    postconditions {
       new this.result == ( (x == y && y == z) ? x == z : True );
    }
}

event transitiveInequalityLT(x: Money, y: Money, z: Money) {
    postconditions {
       new this.result == ( (x < y && y < z) ? x < z : True );
    }
}

event transitiveInequalityGT(x: Money, y: Money, z: Money) {
    postconditions {
       new this.result == ( (x > y && y > z) ? x > z : True );
    }
}
\end{lstlisting}
\caption{The property definitions as \textit{Rebel} specification}
\end{sourcecode}
\FloatBarrier
% End listing
%
% Listing
\begin{sourcecode}[!ht]
\begin{lstlisting}[language=Rebel]
event transitiveInequalityLET(x: Money, y: Money, z: Money) {
    postconditions {
       new this.result == ( (x <= y && y <= z) ? x <= z : True );
    }
}

event transitiveInequalityGET(x: Money, y: Money, z: Money) {
    postconditions {
       new this.result == ( (x >= y && y >= z) ? x >= z : True );
    }
}

event additive(x: Money, y: Money, z: Money) {
    postconditions {
         new this.result == ( (x==y) ? x+z == y+z : True );
    }
}

event additive4params(x: Money, y: Money, z: Money, a: Money) {
    postconditions {
        new this.result == ( (x == y && z == a) ? x+z == y+a : True );
    }
}

event commutativeAddition(x: Money, y: Money) {
    postconditions {
        new this.result == ( x+y == y+x );
    }
}

event commutativeMultiplicationInteger1(x: Integer, y: Money) {
    postconditions {
        new this.result == ( x*y == y*x );
    }
}

event commutativeMultiplicationInteger2(x: Money, y: Integer) {
    postconditions {
        new this.result == ( x*y == y*x );
    }
}

event commutativeMultiplicationPercentage1(x: Percentage, y: Money) {
    postconditions {
        new this.result == ( x*y == y*x );
    }
}

event commutativeMultiplicationPercentage2(x: Money, y: Percentage) {
    postconditions {
        new this.result == ( x*y == y*x );
    }
}

event associativeAddition(x: Money, y: Money, z: Money) {
    postconditions {
        new this.result == ( (x+y)+z == x+(y+z) );
    }
}

event associativeMultiplicationInteger1(x: Integer, y: Integer, z: Money) {
    postconditions {
        new this.result == ( (x*y)*z == x*(y*z) );
    }
}
\end{lstlisting}
\caption{The property definitions as \textit{Rebel} specification (continued)}
\end{sourcecode}
\FloatBarrier
% End listing
%
% Listing
\begin{sourcecode}[!ht]
\begin{lstlisting}[language=Rebel]
event associativeMultiplicationInteger2(x: Money, y: Integer, z: Integer) {
    postconditions {
        new this.result == ( (x*y)*z == x*(y*z) );
    }
}

event associativeMultiplicationPercentage1(x: Money, y: Percentage, z: Integer) {
    postconditions {
        new this.result == ( (x*y)*z == x*(y*z) );
    }
}

event associativeMultiplicationPercentage2(x: Integer, y: Money, z: Percentage) {
    postconditions {
        new this.result == ( (x*y)*z == x*(y*z) );
    }
}

event distributiveInteger1(x: Money, y: Integer, z: Integer) {
    postconditions {
        new this.result == ( x*(y+z) == x*y + x*z );
    }
}

event distributiveInteger2(x: Integer, y: Money, z: Money) {
    postconditions {
        new this.result == ( (y+z)*x == y*x + z*x );
    }
}

event distributivePercentage1(x: Percentage, y: Money, z: Money) {
    postconditions {
        new this.result == ( x*(y+z) == x*y + x*z );
    }
}

event distributivePercentage2(x: Percentage, y: Money, z: Money) {
    postconditions {
        new this.result == ( (y+z)*x == y*x + z*x );
    }
}

event additiveIdentity1(x: Money) {
    postconditions {
        new this.result == ( x + EUR 0.00 == x );
    }
}

event additiveIdentity2(x: Money) {
    postconditions {
        new this.result == ( EUR 0.00 + x == x );
    }
}

event multiplicativeIdentity1(x: Money) {
    postconditions {
        new this.result == ( x*1 == x );
    }
}
event multiplicativeIdentity2(x: Money) {
    postconditions {
        new this.result == ( 1*x == x );
    }
}
\end{lstlisting}
\caption{The property definitions as \textit{Rebel} specification (continued)}
\end{sourcecode}
\FloatBarrier
% End listing
%
% Listing
\begin{sourcecode}[!ht]
\begin{lstlisting}[language=Rebel]
event additiveInverse1(x: Money) {
    postconditions {
        new this.result == ( x+(-x) == EUR 0.00 );
    }
}

event additiveInverse2(x: Money) {
    postconditions {
        new this.result == ( (-x)+x == EUR 0.00 );
    }
}

event antisymmetryLET(x: Money, y: Money) {
    postconditions {
        new this.result == ( (x <= y && y <= x) ? x == y : True );
    }
}

event antisymmetryGET(x: Money, y: Money) {
    postconditions {
        new this.result == ( (x >= y && y >= x) ? x == y : True );
    }
}

event division1(x: Money, y: Integer, z: Money) {
    postconditions {
        new this.result == ( (x*y == z) ? (x == z/y) : True );
    }
}

event division2(x: Money, y: Integer, z: Money) {
    postconditions {
        new this.result == ( (x == z*y) ? (x/y == z) : True );
    }
}

event multiplicativeZeroProperty1(x: Money) {
    postconditions {
        new this.result == ( x*0 == EUR 0.00 );
    }
}

event multiplicativeZeroProperty2(x: Money) {
    postconditions {
        new this.result == ( 0*x == EUR 0.00 );
    }
}

event anticommutativity(x: Money, y: Money) {
    postconditions {
        new this.result == ( x-y == -(y-x) );
    }
}

event nonassociativity(x: Money, y: Money, z: Money) {
    postconditions {
        new this.result == ( (x-y)-z != x-(y-z) );
    }
}

event trichotomy(x: Money, y: Money) {
    postconditions {
        new this.result == ( x < y || x == y || x > y );
    }
}
\end{lstlisting}
\caption{The property definitions as \textit{Rebel} specification (continued)}
\end{sourcecode}
\FloatBarrier
% End listing

% % % % % % % % % % % % % % % % % % % % % % % % % % % % % % % % % % % % %
% Section: Appendix II:
\chapter{Additional property definitions}
\label{app:b_event_definitions_additional}
% Listing
\begin{sourcecode}[!ht]
\begin{lstlisting}[language=Rebel]
event divisionEquality1(x: Money, y: Integer, z: Money) {
    preconditions {
        round((x*y)) == round((z));
    }
    postconditions {
        new this.result == ( (round(x*y) == round(z)) ? (round(x) == round(z/y)) : False );
    }
}

event divisionEquality2(x: Money, y: Integer, z: Money) {
    preconditions {
        round((x)) == round((z*y));
    }
    postconditions {
        new this.result == ( (round(x) == round(z*y)) ? (round(x/y) == round(z)) : False );
    }
}

event divisionEquality3(x: Money, y: Money, z: Integer) {
    preconditions {
        x == y;
        z != 0;
    }
    postconditions {
        new this.result == ( (x == y && z != 0) ? (x/z == y/z) : False );
    }
}

event divisionInequalityLT1(x: Money, y: Money, z: Integer) {
    preconditions {
        x < y;
        z > 0;
    }
    postconditions {
        new this.result == ( (x < y && z > 0) ? (x/z < y/z) : False );
    }
}
\end{lstlisting}
\caption{Additional property definitions as \textit{Rebel} specification}
\end{sourcecode}
\FloatBarrier
% End listing
%
% Listing
\begin{sourcecode}[!ht]
\begin{lstlisting}[language=Rebel]
event divisionInequalityLT2(x: Money, y: Money, z: Integer) {
    preconditions {
        x < y;
        z < 0;
    }
    postconditions {
        new this.result == ( (x < y && z < 0) ? (x/z > y/z) : False );
    }
}

event divisionInequalityLGT1(x: Money, y: Money, z: Integer) {
    preconditions {
        x > y;
        z > 0;
    }
    postconditions {
        new this.result == ( (x > y && z > 0) ? (x/z > y/z) : False );
    }
}

event divisionInequalityLGT2(x: Money, y: Money, z: Integer) {
    preconditions {
        x > y;
        z < 0;
    }
    postconditions {
        new this.result == ( (x > y && z < 0) ? (x/z < y/z) : False );
    }
}

event additiveEquality(x: Money, y: Money, z: Money) {
    preconditions {
        x == y;
    }
    postconditions {
         new this.result == ( (x==y) ? x+z == y+z : False );
    }
}

event additiveEquality4params(x: Money, y: Money, z: Money, a: Money) {
    preconditions {
        x == y;
        z == a;
    }
    postconditions {
        new this.result == ( (x == y && z == a) ? x+z == y+a : False );
    }
}

event additiveInequalityLT(x: Money, y: Money, z: Money) {
    preconditions {
        x < y;
    }
    postconditions {
         new this.result == ( (x<y) ? x+z < y+z : False );
    }
}

event additiveInequalityGT(x: Money, y: Money, z: Money) {
    preconditions {
        x > y;
    }
    postconditions {
         new this.result == ( (x>y) ? x+z > y+z : False );
    }
}
\end{lstlisting}
\caption{Additional property definitions as \textit{Rebel} specification (continued)}
\end{sourcecode}
\FloatBarrier
% End listing
%
% Listing
\begin{sourcecode}[!ht]
\begin{lstlisting}[language=Rebel]
event subtractiveEquality(x: Money, y: Money, z: Money) {
    preconditions {
        x == y;
    }
    postconditions {
        new this.result == ( (x==y) ? (x-z == y-z) : False );
    }
}

event subtractiveInequalityLT(x: Money, y: Money, z: Money) {
    preconditions {
        x < y;
    }
    postconditions {
        new this.result == ( (x<y) ? (x-z < y-z) : False );
    }
}

event subtractiveInequalityGT(x: Money, y: Money, z: Money) {
    preconditions {
        x > y;
    }
    postconditions {
        new this.result == ( (x>y) ? (x-z > y-z) : False );
    }
}

event multiplicativeEquality(x: Money, y: Money, z: Integer) {
    preconditions {
        x == y;
    }
    postconditions {
        new this.result == ( (x==y) ? (x*z == y*z) : False );
    }
}

event multiplicativeInequalityLT1(x: Money, y: Money, z: Integer) {
    preconditions {
        x < y;
        z > 0;
    }
    postconditions {
        new this.result == ( (x<y && z > 0) ? (x*z < y*z) : False );
    }
}

event multiplicativeInequalityLT2(x: Money, y: Money, z: Integer) {
    preconditions {
        x < y;
        z < 0;
    }
    postconditions {
        new this.result == ( (x<y && z < 0) ? (x*z > y*z) : False );
    }
}

event multiplicativeInequalityGT1(x: Money, y: Money, z: Integer) {
    preconditions {
        x > y;
        z > 0;
    }
    postconditions {
        new this.result == ( (x>y && z > 0) ? (x*z > y*z) : False );
    }
}
\end{lstlisting}
\caption{Additional property definitions as \textit{Rebel} specification (continued)}
\end{sourcecode}
\FloatBarrier
% End listing
%
% Listing
\begin{sourcecode}[!ht]
\begin{lstlisting}[language=Rebel]
event multiplicativeInequalityGT2(x: Money, y: Money, z: Integer) {
    preconditions {
        x > y;
        z < 0;
    }
    postconditions {
        new this.result == ( (x>y && z < 0) ? (x*z < y*z) : False );
    }
}
\end{lstlisting}
\caption{Additional property definitions as \textit{Rebel} specification (continued)}
\end{sourcecode}
\FloatBarrier
% End listing


\end{document}


% % % % % % % % % % % % % % % % % % % % % % % % % % % % % % % %
% Possible further improvements (TODO's):
% - Background: Rebel Keller source for labeled transition system
% - Test mechanics: Also define efficiency? Not yet, requires many changes and rerunning is all again.
% - Properties: Also describe percentage? Not yet, requires defining, reasoning, and running it all again. And possibly updating evaluations too
